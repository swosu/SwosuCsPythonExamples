\chapter{Measuring the Cost of Recursion --- Teacher's Commentary}

\section{Teaching Overview}
This chapter moves from wonder to measurement. Students have seen recursion bloom; now they must quantify its cost. As an instructor, your role is to turn curiosity into data.

Key goal: help students feel the \textit{weight} of recursion. The Tracker class turns invisible stack frames into visible numbers.

\section{Pedagogical Setup}
Before class:
\begin{itemize}
  \item Have students run the code \texttt{measure\_fibonacci()} from Chapter 4 of the student workbook.
  \item Ask them to predict: which will grow faster, calls or additions? Which chart will be exponential?
  \item Display the recursive vs. iterative plots side by side.
\end{itemize}

Teaching tip: Pause at each plot. Ask: \textit{What does the shape tell you? What story does this data whisper?}

\section{Instructor Notes on Code}
The \texttt{DataTracker} class is not about performance optimization; it’s about cognitive visibility.
Encourage students to:
\begin{itemize}
  \item Trace which lines increment counters.
  \item Modify it to count multiplications or return depths.
  \item Compare run-to-run variation.
\end{itemize}

Then connect the dots to Big-O notation — it’s not abstract now; it’s empirical.

\section{Common Pitfalls}
\begin{itemize}
  \item Students forget to reset counters between runs.
  \item They compare recursive vs iterative without realizing base cases differ.
  \item They confuse stack growth with data growth — emphasize call count vs. memory use.
\end{itemize}

\section{Extension Ideas}
\begin{itemize}
  \item Challenge them to add memoization and measure again.
  \item Introduce timing for each $n$ and graph log-scale axes to reveal asymptotic behavior.
  \item Ask: “How could we verify this with induction?”
\end{itemize}

\section{Reflection Prompts}
\begin{quote}
Recursion is the art of calling yourself until you learn something new each time.
\end{quote}

\begin{itemize}
  \item What does the graph of recursion \emph{feel} like?
  \item When does elegance become inefficiency?
  \item Can you love a slow algorithm if it teaches you something fast?
\end{itemize}

\section{Instructor Reflection}
This chapter transforms recursion from poetry into physics. By measuring time, calls, and assignments, you bridge emotion and evidence.

Encourage students to end the session by answering one final question:
\textit{If your code were a living thing, what would it remember between calls?}

