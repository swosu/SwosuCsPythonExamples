\chapter{Foundations of Induction and Recursion}

This chapter introduces the central ideas of \textbf{mathematical induction} and \textbf{recursion} as presented in Chapter 5 of Kenneth Rosen's \emph{Discrete Mathematics and Its Applications}.

\section{The Big Picture}

Mathematical induction and recursion are two sides of the same elegant coin. Induction is how we \emph{prove} things about a process that repeats. Recursion is how we \emph{define} that process.

We use induction to reason that what works for one step will work for the next. We use recursion to build structures or compute results by defining a problem in terms of smaller instances of itself.

\section{Key Ideas from Rosen’s Chapter 5}

\begin{itemize}
  \item \textbf{Basis Step:} Prove that a statement is true for an initial value (usually $n=0$ or $n=1$).
  \item \textbf{Inductive Step:} Assume it is true for $n=k$ (the \emph{inductive hypothesis}) and prove it for $n=k+1$.
  \item \textbf{Strong Induction:} Sometimes we assume it’s true for \emph{all} previous cases up to $k$ to prove it for $k+1$.
  \item \textbf{Recursive Definitions:} A way to define sets, sequences, or functions in terms of themselves.
  \item \textbf{Structural Induction:} A generalization used for recursively defined structures like trees or expressions.
\end{itemize}

\section{Why This Matters}

Induction teaches us to trust the domino effect: if one falls and the rule is consistent, they all fall. Recursion lets us \emph{build the dominoes} themselves.

They are the grammar and logic behind everything from factorial functions to sorting algorithms to proofs of algorithmic correctness.

\section{Example: The Factorial Function}

The factorial of $n$, written $n!$, is defined recursively:
\[
  n! = 
  \begin{cases}
    1, & n = 0 \\
    n \cdot (n-1)!, & n > 0
  \end{cases}
\]

We can prove by induction that $n! \ge 2^{n-1}$ for all $n \ge 1$.

\textbf{Proof (sketch):}
\begin{itemize}
  \item \textbf{Base case:} When $n = 1$, we have $1! = 1 \ge 2^{0} = 1$\,\checkmark
  \item \textbf{Inductive step:} Assume $k! \ge 2^{k-1}$ for some integer $k \ge 1$. Then
  \[
      (k+1)! = (k+1)\,k! \ge (k+1)2^{k-1} \ge 2^{k}.
  \]
  Therefore, the inequality holds for $k + 1$, completing the induction.
\end{itemize}

\section{A Student Challenge}

\begin{quote}
“Induction is not a leap of faith — it’s a method of climbing an infinite ladder, one rung at a time.”
\end{quote}

\textbf{Challenge:} Write your own recursive function in Python that computes $n!$, and then write a proof by induction showing why it works for all $n \ge 0$.

\section{Checkpoint Questions}
\begin{enumerate}
  \item What are the two main steps of a proof by induction?
  \item How is recursion related to induction?
  \item Give a real-world example of a recursive process.
  \item Can every recursive definition be proven correct using induction?
\end{enumerate}

