\documentclass[12pt]{article}
\usepackage[margin=1in]{geometry}
\usepackage{amsmath,amssymb}
\usepackage{array,tabularx,multicol}
\usepackage{enumitem}
\usepackage{booktabs}
\setlength{\parskip}{0.6em}
\setlength{\parindent}{0pt}

\begin{document}

{\large \textbf{Discrete Structures \quad Chapter 4.6 — Cryptography}}

\hrule
\vspace{0.6em}

\textbf{Example 1: Caesar Cipher ($k=3$)}

\textbf{Question.}  
Encrypt the message \texttt{MEET YOU IN THE PARK} using the Caesar cipher with shift $k=3$.

\textbf{Step 1 — Letters $\rightarrow$ numbers.}

We use zero-based numbering: A=0, B=1, …, Z=25.  
\[
\text{MEET YOU IN THE PARK} \quad \Rightarrow \quad 
12, 4, 4, 19, \ 24, 14, 20, \ 8, 13, \ 19, 7, 4, \ 15, 0, 17, 10
\]

\textbf{Step 2 — Apply $f(p) = (p+3) \bmod 26$.}

Add 3 to each number, wrapping around if the result exceeds 25:
\[
(12+3)=15, \ (4+3)=7, \ (4+3)=7, \ (19+3)=22,
\]
\[
(24+3)=27 \equiv 1,\ (14+3)=17,\ (20+3)=23,
\]
\[
(8+3)=11,\ (13+3)=16,\ (19+3)=22,\ (7+3)=10,\ (4+3)=7,
\]
\[
(15+3)=18,\ (0+3)=3,\ (17+3)=20,\ (10+3)=13.
\]

\textbf{Step 3 — Numbers $\rightarrow$ letters.}

Convert the ciphertext numbers back to letters:
\[
15,7,7,22,\ 1,17,23,\ 11,16,\ 22,10,7,\ 18,3,20,13
\]
\[
\Rightarrow \text{PHHW BRX LQ WKH SDUN}
\]

\textbf{Final Answer.}  
The encrypted message is:
\[
\boxed{\texttt{PHHW BRX LQ WKH SDUN}}
\]
\emph{(Translation: “MEET YOU IN THE PARK” shifted +3.)}

\hrule
\vspace{1em}

\textbf{Quick Reflection.}  
The Caesar cipher uses modular arithmetic in $\mathbb{Z}_{26}$ so letters “wrap around” after Z. The function $f(p)=(p+k)\bmod 26$ keeps all results in 0–25.

---

\section*{Practice Solutions}

\textbf{P1 — Encrypt (easy).}  
Use $k=5$ to encrypt: \texttt{DOGS AND CATS}.

Step 1 — Convert to numbers:
\[
3,14,6,18,\ 0,13,3,\ 2,0,19,18
\]

Step 2 — Add $5$ mod 26:
\[
8,19,11,23,\ 5,18,8,\ 7,5,24,23
\]

Step 3 — Back to letters:
\[
\boxed{\texttt{ITLX FSI HFYX}}
\]

---

\textbf{P2 — Decrypt (easy).}  
Decrypt \texttt{YMNX NX FQ YJXY} that was made with $k=5$.

We reverse the shift: $c - 5 \pmod{26}$.

\[
\text{Y}=24\to19=\text{T},\ \text{M}=12\to7=\text{H},\ \text{N}=13\to8=\text{I},\ \text{X}=23\to18=\text{S}
\]
\[
\Rightarrow \boxed{\texttt{THIS IS AN TEST}}
\]
So the message is “THIS IS AN TEST.” (It should probably read “THIS IS A TEST.”)

---

\textbf{P3 — Crack the shift (harder).}  
Ciphertext: \texttt{L ORYH PDWKP!}

Try guessing common English patterns.

\texttt{ORYH} looks like “LOVE,” and the one-letter word “L” likely corresponds to “I.”

That suggests a shift of $k=3$ backward (since L $\to$ I is $-3$).

Decrypting with $k=3$:
\[
\texttt{L ORYH PDWKP!} \Rightarrow \boxed{\texttt{I LOVE MATH!}}
\]

---

\section*{Summary of Key Takeaways}
\begin{itemize}
  \item The Caesar cipher is modular addition in $\mathbb{Z}_{26}$.
  \item Encryption: $E_k(p) = (p + k) \bmod 26$
  \item Decryption: $D_k(c) = (c - k) \bmod 26$
  \item If you can add or subtract mod 26, you can encrypt or decrypt.
  \item This cipher is historically important but easily broken by frequency analysis or brute force (26 possibilities).
\end{itemize}

\hrule
\bigskip

\textbf{Going Deeper.}  
You can extend this same math to more complex ciphers:
\[
f(p) = (a\cdot p + b)\bmod 26
\]
where $a$ must have a multiplicative inverse mod 26. This leads directly into the \emph{Affine Cipher}—our next example.

\end{document}

