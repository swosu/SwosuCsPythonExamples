\documentclass[12pt]{article}
\usepackage[margin=1in]{geometry}
\usepackage{amsmath,amssymb}
\usepackage{setspace}
\setstretch{1.2}
\setlength{\parskip}{0.8em}
\setlength{\parindent}{0pt}

\begin{document}
{\large \textbf{Discrete Structures \quad Chapter 4.6 — Cryptography}}

\hrule
\vspace{0.6em}

\section*{Example 12 — The Tale of the Two Keys: Understanding SSH and ED25519}

\subsection*{Scene 1 — The Birth of a Key Pair}

Once upon a command line, a developer typed:

\begin{verbatim}
$ ssh-keygen -t ed25519 -C "your_email@example.com"
\end{verbatim}

and with that, two keys were born.

\begin{itemize}
  \item The \textbf{private key} (usually stored as \texttt{~/.ssh/id\_ed25519}) — a secret mathematical number you must never share.
  \item The \textbf{public key} (stored as \texttt{~/.ssh/id\_ed25519.pub}) — a mathematically related value that anyone may see.
\end{itemize}

These two keys are linked by a one-way mathematical bond:  
from the private key you can derive the public key,  
but from the public key alone, you cannot feasibly find the private key.

\textbf{In mathematical terms:}  
The private key is a random number $k$ (an integer modulo a large prime number),  
and the public key is $K = k \cdot B$,  
where $B$ is a special base point on an elliptic curve called \emph{Curve25519}.

The multiplication $\cdot$ here isn’t ordinary arithmetic — it’s \emph{elliptic curve point multiplication}.  
It’s easy to go forward (compute $K$ from $k$),  
but practically impossible to go backward (find $k$ given $K$).

This one-way direction is what makes the system secure.

\bigskip
\hrule
\vspace{0.5em}

\subsection*{Scene 2 — The Meeting with GitHub}

When you upload your \texttt{id\_ed25519.pub} to GitHub,  
you’re essentially saying:

\begin{quote}
“GitHub, if anyone ever proves they know the private number $k$ that goes with this public key $K$,  
please trust them — that’s me.”
\end{quote}

GitHub doesn’t need to store your private key or a password.  
It just saves your public key $K$ and waits for you to prove your identity through math.

\bigskip
\hrule
\vspace{0.5em}

\subsection*{Scene 3 — The Silent Handshake}

Later, when you try to connect:

\begin{verbatim}
$ git push origin main
\end{verbatim}

your computer (the \emph{client}) sends GitHub (the \emph{server}) a friendly request:
“Hey, I’d like to log in using my public key $K$.”

GitHub checks if it recognizes that $K$ from your account,  
and if so, it sends back a one-time random challenge —  
a random number $r$ that must be signed.

Your computer uses your private key $k$ to produce a mathematical signature $\sigma$  
that proves “I know $k$” without ever revealing $k$ itself.

GitHub then uses your public key $K$ to check the signature:
\[
\text{Verify}(\sigma, r, K) \Rightarrow \text{True or False}.
\]
If the math checks out, GitHub knows it’s really you.

\bigskip
\textbf{No passwords were typed. No secrets were sent. Only proof.}

This is called \textbf{public-key authentication},  
and the math behind it (in ED25519) is built on the secure foundation of elliptic curve cryptography.

\bigskip
\hrule
\vspace{0.5em}

\subsection*{Scene 4 — Under the Hood: ED25519’s Elegant Math}

At its heart, ED25519 uses a specific elliptic curve over a finite field $\mathbb{F}_p$  
with $p = 2^{255} - 19$.  
The curve equation looks like this:
\[
-x^2 + y^2 = 1 + d x^2 y^2
\]
where $d = -121665/121666$.

All computations are done \emph{modulo} $p$.

\textbf{Signing process (simplified):}
\begin{enumerate}
  \item Your private key $k$ is used to generate a random value $r$.
  \item Compute a public value $R = r \cdot B$.
  \item Hash together $(R, K, \text{message})$ to get a challenge $h$.
  \item Compute a response $s = r + h \cdot k \pmod q$.
  \item Your signature is the pair $(R, s)$.
\end{enumerate}

\textbf{Verification process:}
\[
sB \stackrel{?}{=} R + hK.
\]
If that equality holds, the signature is valid —  
proving you knew the private key $k$ that corresponds to the public key $K$,  
without ever revealing $k$ itself.

\bigskip
\hrule
\vspace{0.5em}

\subsection*{Scene 5 — Trust, Speed, and Beauty}

Why ED25519? Because it’s:
\begin{itemize}
  \item \textbf{Fast.} Operations on Curve25519 are designed for efficiency on modern CPUs.
  \item \textbf{Secure.} 255-bit keys are strong enough for decades of safety.
  \item \textbf{Simple.} The algorithm avoids messy mathematical corner cases that plagued older elliptic curves.
  \item \textbf{Deterministic.} The same private key always yields the same signature, eliminating random side-channel leaks.
\end{itemize}

\textbf{In short:}  
ED25519 is the elegant modern successor to RSA — leaner, faster, and just as trustworthy.

\bigskip
\hrule
\vspace{0.5em}

\subsection*{Scene 6 — What Really Happens When You Push to GitHub}

When you type:
\begin{verbatim}
git push origin main
\end{verbatim}

a full cryptographic handshake unfolds in milliseconds:

\begin{enumerate}
  \item GitHub sends a challenge (random data).
  \item Your SSH client signs that challenge using your private key.
  \item GitHub verifies the signature with your public key.
  \item The secure session is established — encrypted, authenticated, and trusted.
\end{enumerate}

Every push, pull, or fetch after that uses the same session’s encrypted channel.  
No passwords. No eavesdropping. Just math.

\bigskip
\hrule
\vspace{0.5em}

\subsection*{Epilogue — The Moral of the Story}

SSH with ED25519 is not just about convenience —  
it’s about \textbf{trust through proof, not secrecy.}

\begin{quote}
Your private key whispers: “I know something only I could know.”  
Your public key declares: “Check my math, and you’ll believe me.”
\end{quote}

Together they make it possible for developers around the world  
to collaborate securely — building bridges of trust made entirely of numbers.

\bigskip
\hrule
\vspace{0.5em}

\subsection*{Optional Extension for the Curious}

If you’d like to visualize what’s really happening:
\begin{itemize}
  \item Generate a key: \texttt{ssh-keygen -t ed25519}
  \item Print your public key: \texttt{cat ~/.ssh/id\_ed25519.pub}
  \item Run: \texttt{ssh -vT git@github.com}
\end{itemize}

You’ll see each step of the cryptographic dialogue.  
It’s math in motion — security as symphony.

\bigskip
\textbf{Takeaway:}  
When you use SSH with GitHub, you are not “logging in.”  
You are \emph{proving who you are with math}.

\end{document}

