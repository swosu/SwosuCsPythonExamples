\documentclass[12pt]{article}
\usepackage[margin=1in]{geometry}
\usepackage{amsmath,amssymb,booktabs,array,tabularx}
\usepackage{enumitem}
\setlength{\parskip}{0.65em}
\setlength{\parindent}{0pt}

% A roomy writing-space box
\newcommand{\workbox}[1]{%
  \vspace{0.25em}\fbox{\parbox{\dimexpr\textwidth-2\fboxsep-2\fboxrule\relax}{\vspace{#1}}}%
}

\begin{document}
{\large \textbf{Discrete Structures \qquad Chapter 4.6 — Cryptography}}

\hrule
\vspace{0.6em}

\section*{Example 9 (Worksheet) — RSA Decryption (work the process, not just the answer)}

\textbf{Problem.} We receive the ciphertext blocks \texttt{0981 0461} produced by the RSA cryptosystem from Example 8. The public key was \((n,e)=(2537,13)\) with \(n=43\cdot59\). Decrypt the message.

\subsection*{The big idea (the why)}
RSA works in blocks. If a block \(c\) was encrypted with \(c \equiv m^e \pmod{n}\), then anyone who knows the \emph{private} exponent \(d\) (the inverse of \(e\bmod\phi(n)\)) can recover the plaintext block via
\[
m \equiv c^d \pmod{n}.
\]
This is fast thanks to \emph{repeated squaring}. After recovering each numeric block \(m\), translate back to letters using two digits per letter: \(A=00,\dots, Z=25\). Leading zeros matter!

\subsection*{Step 1 — Compute the private exponent \(d\)}
\[
\phi(n)=(p-1)(q-1)=42\cdot58=2436,\qquad \text{find }d\text{ with }13d\equiv 1\pmod{2436}.
\]
Extended Euclid gives \(d=937\) (indeed \(13\cdot 937=12181=1+5\cdot2436\)).

\subsection*{Step 2 — Decrypt each block with repeated squaring}

\textbf{Block 1:} \(c=0981\Rightarrow c=981\).
\[
m \equiv 981^{937}\pmod{2537},\qquad 937=512+256+128+32+8+1.
\]
Squares mod \(2537\):
\[
\begin{array}{r|rrrrrrrrr}
\text{power} & 1 & 2 & 4 & 8 & 16 & 32 & 64 & 128 & 256 & 512\\\hline
981^{\text{power}} \bmod 2537 & 
981 & 838 & 1922 & 1325 & 450 & 441 & 322 & 2472 & 1688 & 293
\end{array}
\]
Multiply only the needed entries (powers \(1,8,32,128,256,512\)):
\[
\begin{aligned}
r&\leftarrow 1\\
r&\cdot 981 \equiv 981\\
r&\cdot 1325 \equiv  981\cdot1325 \equiv  1717\\
r&\cdot 441 \equiv  1717\cdot441 \equiv  1251\\
r&\cdot 2472 \equiv  1251\cdot2472 \equiv  282\\
r&\cdot 1688 \equiv  282\cdot1688 \equiv  1292\\
r&\cdot 293 \equiv  1292\cdot293 \equiv  \boxed{704}
\end{aligned}
\]
So \(m_1=0704\Rightarrow \texttt{07 04}=\texttt{H E}\).

\medskip
\textbf{Block 2:} \(c=0461\Rightarrow c=461\).
\[
m \equiv 461^{937}\pmod{2537},\qquad 937=512+256+128+32+8+1.
\]
Squares mod \(2537\):
\[
\begin{array}{r|rrrrrrrrr}
\text{power} & 1 & 2 & 4 & 8 & 16 & 32 & 64 & 128 & 256 & 512\\\hline
461^{\text{power}} \bmod 2537 & 
461 & 1950 & 2074 & 1261 & 1959 & 1737 & 676 & 316 & 913 & 1433
\end{array}
\]
Multiply the needed entries (powers \(1,8,32,128,256,512\)):
\[
\begin{aligned}
r&\leftarrow 1\\
r&\cdot 461 \equiv 461\\
r&\cdot 1261 \equiv  461\cdot1261 \equiv  1327\\
r&\cdot 1737 \equiv  1327\cdot1737 \equiv  1559\\
r&\cdot 316 \equiv  1559\cdot316 \equiv  1122\\
r&\cdot 913 \equiv  1122\cdot913 \equiv  82\\
r&\cdot 1433 \equiv  82\cdot1433 \equiv \boxed{1115}
\end{aligned}
\]
So \(m_2=1115\Rightarrow \texttt{11 15}=\texttt{L P}\).

\subsection*{Step 3 — Read the plaintext}
Blocks \(0704\ 1115\) translate to \(\boxed{\texttt{HELP}}\).

\subsection*{Tips, tricks, and common pitfalls}
\begin{itemize}[leftmargin=1.2em]
\item Keep the two-digit mapping straight: \(A=00,\ldots,J=09,\ldots,Z=25\). Leading zeros are part of the block!
\item Choose the block size so that each four–digit block \(m\) is \(\boldsymbol{< n}\). Here \(2N=4\) works because \(2525 < 2537 < 252525\).
\item When doing repeated squaring, build a small table of \(c^{1}, c^{2}, c^{4}, \dots\) and then multiply only the powers that add up to \(d\).
\item Arithmetic gets easier if you reduce \emph{often}. Every product should be brought back modulo \(n\) immediately.
\end{itemize}

\bigskip
\hrule
\vspace{0.4em}
\section*{Practice — Your Turn (use \(n=2537,\ e=13,\ d=937\))}
Use the same key as above. Show your exponentiation steps and \emph{keep} leading zeros when converting back to letters.

\textbf{Problem A (easier).} Decrypt the single block \texttt{2081}. What two letters do you get?  
\emph{Hint:} compute \(2081^{937}\bmod 2537\) and then split the result as \(\_\_\ \_\_\).
\workbox{2.2cm}

\textbf{Problem B (similar).} Decrypt the two–block ciphertext \texttt{2081 2182}.  
\emph{Reminder:} convert each four–digit block separately, then map back to letters.
\workbox{2.6cm}

\textbf{Problem C (harder).} The ciphertext \texttt{0981 0724 1774} was made with the same key.  
\begin{itemize}[leftmargin=1.2em, topsep=0.2em,itemsep=0.2em]
\item Decrypt all three blocks.
\item Translate to letters. If the last block ends with a padding letter \texttt{X}, circle it.
\end{itemize}
\workbox{3.2cm}

\bigskip
\textbf{Reflection.} In one or two sentences, explain why knowing \(e\) and \(n\) does \emph{not} make decryption easy, but knowing \(d\) does.
\workbox{1.6cm}

\end{document}

