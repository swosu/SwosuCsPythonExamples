\documentclass[12pt]{article}
\usepackage[margin=1in]{geometry}
\usepackage{amsmath,amssymb}

\title{Ch 4.2 — Example 10 Teacher Solutions\\Binary Multiplication via Algorithm 3}
\author{Instructor Key}
\date{}

\begin{document}
\maketitle

\section*{Reference: Algorithm 3 (Multiplication)}
For $a=(a_{n-1}\dots a_0)_2$ and $b=(b_{n-1}\dots b_0)_2$,
\[
c_j=\begin{cases}
a \text{ shifted left $j$}, & b_j=1,\\
0, & b_j=0,
\end{cases}
\qquad
p=\sum_{j=0}^{n-1} c_j,
\]
where the sum is carried out with Algorithm 2.

\section*{Worked Example (text): $(110)_2 \times (101)_2$}
Multiplier bits: $b_0=1$, $b_1=0$, $b_2=1$.
\[
c_0=(110)_2,\qquad c_1=(000)_2,\qquad c_2=(11000)_2.
\]
Add (Algorithm 2):
\[
\begin{array}{r}
11000\\
00000\\
+00110\\ \hline
11110
\end{array}
\quad\Rightarrow\quad (11110)_2.
\]
Decimal check: $6\cdot 5=30$, $(11110)_2=30$.

\bigskip
\hrule
\bigskip

\section*{Solutions to Student Practice}

\subsection*{A. Easier}
\textbf{1) } $(101)_2 \times (11)_2$.

$a=(101)_2$, $b=(011)_2$ (writing as three bits helps). Bits: $b_0=1$, $b_1=1$, $b_2=0$.
\[
c_0=(101)_2,\quad c_1=(1010)_2,\quad c_2=(0000)_2.
\]
Add:
\[
\begin{array}{r}
01010\\
00101\\ \hline
01111
\end{array}
\quad\Rightarrow\quad (1111)_2.
\]
Check: $5\times 3=15$, $(1111)_2=15$.

\subsection*{B. Harder}

\textbf{2) } $(10011)_2 \times (1011)_2$.

$a=(10011)_2$, $b=(1011)_2$ has $b_0=1,b_1=1,b_2=0,b_3=1$.
\[
\begin{aligned}
c_0&=(10011)_2,\\
c_1&=(100110)_2,\\
c_2&=(000000)_2,\\
c_3&=(10011000)_2.
\end{aligned}
\]
Add in two stages (Algorithm 2).

\emph{Stage 1: } $c_0+c_1$:
\[
\begin{array}{r}
100110\\
+10011\\ \hline
111001
\end{array}
\quad\Rightarrow\quad (111001)_2.
\]

\emph{Stage 2: } add $c_3$ (align lengths):
\[
\begin{array}{r}
10011000\\
00111001\\ \hline
11010001
\end{array}
\quad\Rightarrow\quad (11010001)_2.
\]
Answer: $(11010001)_2$.

\emph{Decimal check:} $(10011)_2=19$, $(1011)_2=11$, $19\cdot 11=209$; $(11010001)_2=209$.

\bigskip

\textbf{3) } $(111010)_2 \times (10111)_2$.

$a=(111010)_2$, $b=(10111)_2$ with bits $b_0=1,b_1=1,b_2=1,b_3=0,b_4=1$.
\[
\begin{aligned}
c_0&=(111010)_2,\\
c_1&=(1110100)_2,\\
c_2&=(11101000)_2,\\
c_3&=(000000000)_2,\\
c_4&=(1110100000)_2.
\end{aligned}
\]

Add progressively (Algorithm 2).

\emph{(i) } $c_0+c_1$:
\[
\begin{array}{r}
1110100\\
0111010\\ \hline
10111110
\end{array}
\Rightarrow (10111110)_2.
\]

\emph{(ii) } add $c_2$:
\[
\begin{array}{r}
11101000\\
010111110\\ \hline
1010011110
\end{array}
\Rightarrow (1010011110)_2.
\]

\emph{(iii) } add $c_4$ (align):
\[
\begin{array}{r}
1110100000\\
001010011110\\ \hline
0100011011110
\end{array}
\Rightarrow (100011011110)_2.
\]

Answer: $(100011011110)_2$.

\emph{Decimal check:} $(111010)_2=58$, $(10111)_2=23$, $58\cdot 23=1334$; $(100011011110)_2=1024+256+32+16+8-?$
(Compute): $1024+256+32+16+8+4+2=1342$? — re-check.
\medskip

\textbf{Tidy recomputation (columns):}
\[
(100011011110)_2 = 2^{11}+2^{7}+2^{6}+2^{4}+2^{3}+2^{2}+2^{1}
= 2048+128+64+16+8+4+2=2{,}270 \text{ (too large)}.
\]
So we misaligned in (ii). Let's recompute carefully with consistent widths.

\paragraph{Clean column addition}
Write all partials to 11 columns (max length of $c_4$ is $10$ + safety):

\[
\begin{array}{r}
\phantom{0}000111010\quad (c_0)\\
\phantom{0}001110100\quad (c_1)\\
\phantom{0}011101000\quad (c_2)\\
\phantom{0}000000000\quad (c_3)\\
\;1110100000\quad (c_4)\\ \hline
\end{array}
\]

Add top to bottom (carry shown conceptually):
\[
\begin{array}{r}
\;1110100000\\
+001110100\\
+000111010\\
+011101000\\ \hline
\;10000111110
\end{array}
\Rightarrow (10000111110)_2.
\]

Now check in decimal:
$(10000111110)_2=2^{10}+2^4+2^3+2^2+2^1=1024+16+8+4+2=1054$ — still not $58\cdot 23$.

\textbf{Better path:} verify with decimal first: $58\times 23 = 1334$.
Binary of $1334$:
\[
1334 = 1024 + 256 + 32 + 16 + 4 + 2 \Rightarrow (10100110110)_2.
\]

Let’s recompute the partials \emph{precisely}:

$a=(111010)_2$:
\[
a=1\cdot 2^5+1\cdot 2^4+1\cdot 2^3+0\cdot 2^2+1\cdot 2^1+0 = 58.
\]
$b=(10111)_2$ with bits $b_4=1,b_3=0,b_2=1,b_1=1,b_0=1$.
\[
\begin{aligned}
c_0 &= a\cdot 2^0 = (111010)_2,\\
c_1 &= a\cdot 2^1 = (1110100)_2,\\
c_2 &= a\cdot 2^2 = (11101000)_2,\\
c_3 &= 0,\\
c_4 &= a\cdot 2^4 = (1110100000)_2.
\end{aligned}
\]
Now align on the right and add:

\[
\begin{array}{r}
1110100000\\
0011101000\\
0001110100\\
0000000000\\
\underline{0000111010}\\ \hline
10100110110
\end{array}
\]

Result: $(10100110110)_2$.  
Decimal check: $1024+256+32+16+4+2=1334$. Correct.

\paragraph{Answer.} $(111010)_2\times(10111)_2=(10100110110)_2$.

\bigskip\hrule\bigskip

\section*{Operation counts (optional talking points)}
If $b$ has $k$ ones among $n$ bits, Algorithm 3 forms $k$ nonzero partial products and performs up to $k-1$ multiword additions (Algorithm 2). Each addition is $O(n)$, so multiplication is $O(kn)\subseteq O(n^2)$ in the worst case ($k\approx n$).

\end{document}

