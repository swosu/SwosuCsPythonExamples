\documentclass[12pt]{article}
\usepackage[margin=1in]{geometry}
\usepackage{amsmath,amssymb}
\usepackage{setspace}
\setstretch{1.25}
\setlength{\parskip}{0.7em}
\setlength{\parindent}{0pt}

\begin{document}
{\large \textbf{Discrete Structures \quad Chapter 4.6 — Cryptography}}

\hrule
\vspace{0.6em}

\section*{Example 13 — A Tiny Toy Model of Public–Private Keys}

\subsection*{Goal}
To see, step by step, how a public key and private key can work together to lock and unlock information.  
We’ll use numbers so small you can compute them in your head.  
This is not secure — it’s a sandbox for understanding the flow.

\bigskip
\hrule
\vspace{0.5em}

\subsection*{Scene 1 — Choosing a “World” (the modulus)}

Let’s build a tiny arithmetic world where everything wraps around at 33.  
We’ll say all operations are done “mod 33.”

That means:  
\[
a \equiv b \pmod{33}
\quad\text{if they differ by a multiple of 33.}
\]

So, for instance, $40 \equiv 7 \pmod{33}$ because $40-7=33$.

\bigskip
\hrule
\vspace{0.5em}

\subsection*{Scene 2 — The Secret and the Public Keys}

We pick two numbers that work as opposites in this modular world.  
We want one number $e$ for “encrypt” and one number $d$ for “decrypt,”  
so that doing both in a row brings us back to the original message $m$.

We need:
\[
(m^e)^d \equiv m \pmod{33}.
\]

For this to happen, $e$ and $d$ must be inverses with respect to the totient of 33.  
(We’ll skip the deep theory — we’ll just choose a pair that works.)

Let’s pick:

\[
e = 3, \qquad d = 7.
\]

We can check that these behave nicely because:
\[
3\times 7 = 21 \equiv 1 \pmod{20},
\]
and 20 is the totient of 33 (that’s a fancy way of saying there are 20 numbers less than 33 that don’t share a factor with 33).

Perfect! They are modular inverses.

\textbf{Private key:} $d=7$  
\textbf{Public key:} $(e=3,\ n=33)$

\bigskip
\hrule
\vspace{0.5em}

\subsection*{Scene 3 — Sending a Secret Message}

Suppose Bob wants to send Alice the number $m=4$ (representing a small message).

He uses Alice’s public key $(e,n)=(3,33)$ to encrypt it:

\[
c \equiv m^e \pmod{33} = 4^3 \bmod 33.
\]

Compute:  
$4^3 = 64,$ and $64 \bmod 33 = 64 - 33 = 31.$

So the ciphertext is:
\[
c = 31.
\]

Bob sends $31$ to Alice.

\bigskip
\hrule
\vspace{0.5em}

\subsection*{Scene 4 — Unlocking the Secret}

Alice uses her private key $d=7$ to decrypt:

\[
m' \equiv c^d \pmod{33} = 31^7 \bmod 33.
\]

That looks nasty, but in modular arithmetic patterns repeat fast.  
Let’s compute powers of 31 mod 33:

\[
31^1 \equiv 31, \quad
31^2 \equiv 31\cdot31 = 961 \equiv 4, \quad
31^3 \equiv 4\cdot31 = 124 \equiv 25,\\
31^4 \equiv 25\cdot31 = 775 \equiv 13, \quad
31^5 \equiv 13\cdot31 = 403 \equiv 7,\\
31^6 \equiv 7\cdot31 = 217 \equiv 19, \quad
31^7 \equiv 19\cdot31 = 589 \equiv 4.
\]

So:
\[
m' = 4.
\]

It worked! Alice got back the original message.  
She never revealed $d$, and Bob never knew it — he only knew $e$ and $n$.

\bigskip
\hrule
\vspace{0.5em}

\subsection*{Scene 5 — Reversing the Flow (Digital Signature)}

Now Alice wants to prove that a message came from her.  
She uses her \textbf{private key first}, and anyone can check it with her \textbf{public key}.

She signs $m=5$ by computing:

\[
s \equiv m^d \pmod{33} = 5^7 \bmod 33.
\]

Compute $5^7 = 78{,}125$.  
Divide by 33: $33\times2367 = 78{,}111$, remainder 14.  
So $s=14$.

Alice sends $(m=5, s=14)$.

\textbf{Verification:}  
Anyone with her public key $(e=3,n=33)$ checks:

\[
s^e \equiv 14^3 \bmod 33 = 2744 \bmod 33.
\]

Compute: $33\times83=2739$, remainder 5.

So $s^e \equiv 5 \pmod{33}$ — it matches $m$.  
The signature checks out!

\bigskip
\hrule
\vspace{0.5em}

\subsection*{Scene 6 — What We Learned}

\begin{itemize}
  \item The \textbf{public key} lets anyone lock a box that only the private key can open.
  \item The \textbf{private key} can sign something that anyone can verify with the public key.
  \item The keys are linked by a one-way relationship: $d$ and $e$ are modular inverses mod $\varphi(n)$.
  \item Real systems like RSA or ED25519 do this same dance — just with gigantic primes and far more sophisticated math.
\end{itemize}

\bigskip
\textbf{Takeaway:}  
Encryption and signatures are two directions of the same beautiful math.  
Public locks, private keys, and modular worlds —  
all making trust possible in the land of numbers.

\end{document}

