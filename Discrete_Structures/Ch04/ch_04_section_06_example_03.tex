\documentclass[12pt]{article}
\usepackage[margin=1in]{geometry}
\usepackage{amsmath}
\usepackage{amssymb}
\usepackage{enumitem}
\usepackage{xcolor}
\usepackage{tcolorbox}

\title{\textbf{Caesar Cipher Decryption} \\ Student Worksheet}
\date{}

\begin{document}
\maketitle

\section*{Understanding Decryption}

Previously, we learned how to \textbf{encrypt} messages using the Caesar cipher. Now we'll learn to \textbf{decrypt} them—convert the secret message back to the original!

The key insight: \textbf{Decryption is the reverse of encryption.}

\begin{itemize}[leftmargin=*]
\item \textbf{Encryption:} We shifted letters \emph{forward} by $k$ positions using $f(p) = (p + k) \bmod 26$
\item \textbf{Decryption:} We shift letters \emph{backward} by $k$ positions using $f(p) = (p - k) \bmod 26$
\end{itemize}

\vspace{0.3cm}

\begin{tcolorbox}[colback=blue!5!white,colframe=blue!50!black,title=\textbf{Key Concept: Negative Numbers and Mod}]
When we subtract and get a negative number, we need to ``wrap around'' the other direction. For example, if we try to go back 7 from the letter E (position 4), we get $4 - 7 = -3$. 

To handle this, we compute: $-3 \bmod 26 = 23$ (which is the letter X).

\textbf{Quick trick:} If you get a negative number, just add 26 to make it positive!
$$-3 + 26 = 23$$
\end{tcolorbox}

\vspace{0.5cm}

\section*{Example 3: Worked Solution}

\textbf{Question:} Decrypt the ciphertext message ``LEWLYPLUJL PZ H NYLHA HSOHOLY'' that was encrypted with the shift cipher with shift $k = 7$.

\vspace{0.3cm}

\noindent\textbf{\textcolor{blue}{Solution:}}

\textbf{Step 1: Convert letters to numbers}

We use our standard A=0, B=1, C=2, \ldots, Z=25 system. Let's convert the ciphertext:

\begin{itemize}[leftmargin=*]
\item \textbf{LEWLYPLUJL:} L=11, E=4, W=22, L=11, Y=24, P=15, L=11, U=20, J=9, L=11
\item \textbf{PZ:} P=15, Z=25
\item \textbf{H:} H=7
\item \textbf{NYLHA:} N=13, Y=24, L=11, H=7, A=0
\item \textbf{HSOHOLY:} A=0, L=11, H=7, O=14, L=11, Y=24
\end{itemize}

\vspace{0.2cm}
\noindent Our number sequence is:
\begin{center}
\begin{tabular}{cccccccccc}
11 & 4 & 22 & 11 & 24 & 15 & 11 & 20 & 9 & 11 \\
\end{tabular}
\end{center}
\begin{center}
\begin{tabular}{cc}
15 & 25 \\
\end{tabular}
\qquad
\begin{tabular}{c}
7 \\
\end{tabular}
\qquad
\begin{tabular}{ccccc}
13 & 24 & 11 & 7 & 0 \\
\end{tabular}
\qquad
\begin{tabular}{cccccc}
0 & 11 & 7 & 9 & 14 & 11 & 24
\end{tabular}
\end{center}

\vspace{0.3cm}

\textbf{Step 2: Apply the decryption function}

We apply $f(p) = (p - 7) \bmod 26$ to shift backward by 7. Let's work through each number:

\begin{align*}
f(11) &= (11 - 7) \bmod 26 = 4 \bmod 26 = 4 \\
f(4) &= (4 - 7) \bmod 26 = -3 \bmod 26 = 23 \quad \text{(add 26: } -3 + 26 = 23\text{)} \\
f(22) &= (22 - 7) \bmod 26 = 15 \bmod 26 = 15 \\
f(11) &= (11 - 7) \bmod 26 = 4 \bmod 26 = 4 \\
f(24) &= (24 - 7) \bmod 26 = 17 \bmod 26 = 17 \\
f(15) &= (15 - 7) \bmod 26 = 8 \bmod 26 = 8 \\
f(11) &= (11 - 7) \bmod 26 = 4 \bmod 26 = 4 \\
f(20) &= (20 - 7) \bmod 26 = 13 \bmod 26 = 13 \\
f(9) &= (9 - 7) \bmod 26 = 2 \bmod 26 = 2 \\
f(11) &= (11 - 7) \bmod 26 = 4 \bmod 26 = 4
\end{align*}

Continuing for the remaining letters:
\begin{align*}
f(15) &= 8, \quad f(25) = 18, \quad f(7) = 0 \\
f(13) &= 6, \quad f(24) = 17, \quad f(11) = 4, \quad f(7) = 0, \quad f(0) = 19 \\
f(0) &= 19, \quad f(11) = 4, \quad f(7) = 0, \quad f(9) = 2, \quad f(14) = 7, \quad f(11) = 4, \quad f(24) = 17
\end{align*}

\vspace{0.2cm}
\noindent Our decrypted numbers are:
\begin{center}
\begin{tabular}{cccccccccc}
4 & 23 & 15 & 4 & 17 & 8 & 4 & 13 & 2 & 4 \\
\end{tabular}
\end{center}
\begin{center}
\begin{tabular}{cc}
8 & 18 \\
\end{tabular}
\qquad
\begin{tabular}{c}
0 \\
\end{tabular}
\qquad
\begin{tabular}{ccccc}
6 & 17 & 4 & 0 & 19 \\
\end{tabular}
\qquad
\begin{tabular}{ccccccc}
19 & 4 & 0 & 2 & 7 & 4 & 17
\end{tabular}
\end{center}

\vspace{0.3cm}

\begin{tcolorbox}[colback=green!5!white,colframe=green!50!black,title=\textbf{Pro Tip: Handling Negative Results}]
Whenever $(p - k)$ gives you a negative number:
\begin{enumerate}
\item Notice it's negative
\item Add 26 to make it positive
\item That's your answer!
\end{enumerate}
Example: $(4 - 7) = -3$, so $-3 + 26 = 23$
\end{tcolorbox}

\vspace{0.3cm}

\textbf{Step 3: Convert numbers back to letters}

Using A=0, B=1, \ldots, Z=25:

\begin{itemize}[leftmargin=*]
\item 4=E, 23=X, 15=P, 4=E, 17=R, 8=I, 4=E, 13=N, 2=C, 4=E
\item 8=I, 18=S
\item 0=A
\item 6=G, 17=R, 4=E, 0=A, 19=T
\item 19=T, 4=E, 0=A, 2=C, 7=H, 4=E, 17=R
\end{itemize}

\vspace{0.3cm}
\noindent\textbf{Final Answer:} The decrypted message is \fbox{\textbf{EXPERIENCE IS A GREAT TEACHER}}

\vspace{0.3cm}

\begin{tcolorbox}[colback=purple!5!white,colframe=purple!50!black,title=\textbf{Why This Works}]
If someone encrypted a message by shifting forward 7, we decrypt by shifting backward 7. It's like walking 7 steps forward, then 7 steps back—you end up where you started!
\end{tcolorbox}

\vspace{0.5cm}

\hrule

\vspace{0.5cm}

\section*{Practice Problems}

\subsection*{Problem A (Easier Warm-up)}

Decrypt the ciphertext ``\textbf{FDW}'' that was encrypted with shift $k=3$.

\vspace{0.3cm}
\noindent\textit{Hint: This is a short message. Remember to subtract 3 from each letter's position. If you get negative numbers, add 26!}

\vspace{5cm}

\subsection*{Problem B (Standard Practice)}

Decrypt the ciphertext ``\textbf{MJQQT BTWQI}'' that was encrypted with shift $k=5$.

\vspace{0.3cm}
\noindent\textit{Hint: You encrypted this message in the previous worksheet! Now decrypt it to get back the original message.}

\vspace{6cm}

\subsection*{Problem C (Challenge)}

Decrypt the ciphertext ``\textbf{EJKKR ZRUOJ}'' that was encrypted with shift $k=5$.

\vspace{0.3cm}
\noindent\textit{Challenge: Some of these letters will give negative results when you subtract 5. Practice your wrapping-around skills!}

\vspace{6cm}

\end{document}
