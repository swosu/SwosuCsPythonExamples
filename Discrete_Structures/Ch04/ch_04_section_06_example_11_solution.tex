\documentclass[12pt]{article}
\usepackage[margin=1in]{geometry}
\usepackage{amsmath,amssymb}
\setlength{\parskip}{0.6em}
\setlength{\parindent}{0pt}

\newcommand{\Z}{\mathbb{Z}}

\begin{document}
{\large \textbf{Discrete Structures \quad Chapter 4.6 — Cryptography (Solutions)}}

\hrule
\vspace{0.6em}

\section*{Example 11 (Solutions) — RSA as a Partially Homomorphic System}

\subsection*{Conceptual Recap}
RSA encryption is given by:
\[
E_{(n,e)}(M) = M^e \bmod n.
\]
A cryptosystem is called \emph{homomorphic} if operations on ciphertexts correspond to operations on plaintexts.

For RSA:
\[
E(M_1)\cdot E(M_2) \equiv (M_1^e)(M_2^e) \equiv (M_1M_2)^e \equiv E(M_1M_2)\ (\bmod n).
\]
Thus, RSA is \textbf{multiplicatively homomorphic.}  
However, since
\[
E(M_1) + E(M_2) \neq E(M_1 + M_2),
\]
RSA is not additively homomorphic.  
Therefore, we describe RSA as \textbf{partially homomorphic.}

\bigskip
\hrule
\vspace{0.5em}

\section*{Worked Example}

We’ll use a small RSA system for clarity:  
\[
n = 77 = 7 \times 11, \quad e = 7, \quad d = 43.
\]

\textbf{Plaintexts:} $M_1 = 5$, $M_2 = 9$.

\textbf{Step 1. Encrypt each plaintext.}
\[
E(5) = 5^7 \bmod 77.
\]

Compute:
\[
5^2 = 25,\quad 5^4 = 25^2 = 625 \equiv 625 - 8\cdot77 = 625 - 616 = 9,
\]
\[
5^7 = 5^4 \cdot 5^2 \cdot 5 = 9 \cdot 25 \cdot 5 = 1125 \equiv 36 \pmod{77}.
\]

So \(E(5)=36.\)

Similarly,
\[
E(9) = 9^7 \bmod 77.
\]

Compute:
\[
9^2 = 81 \equiv 4,\quad 9^4 = 4^2 = 16,
\]
\[
9^7 = 9^4 \cdot 9^2 \cdot 9 = 16 \cdot 4 \cdot 9 = 576 \equiv 71 \pmod{77}.
\]

So \(E(9)=71.\)

\textbf{Step 2. Multiply ciphertexts.}
\[
E(5)\cdot E(9) \bmod 77 = 36\cdot71 = 2556 \equiv 15 \pmod{77}.
\]

\textbf{Step 3. Encrypt the product plaintext.}
\[
E(5\cdot9) = E(45) = 45^7 \bmod 77.
\]

Compute with successive squaring:
\[
45^2 = 2025 \equiv 2025 - 26\cdot77 = 2025 - 2002 = 23,
\]
\[
45^4 = 23^2 = 529 \equiv 529 - 6\cdot77 = 529 - 462 = 67,
\]
\[
45^7 = 45^4 \cdot 45^2 \cdot 45 = 67 \cdot 23 \cdot 45 = 69285.
\]
Reduce modulo 77:
\[
69285 \div 77 = 900\text{ remainder }15.
\]
So \(E(45) = 15.\)

\textbf{They match!}  
Hence RSA preserves multiplication under encryption.

\bigskip
\hrule
\vspace{0.5em}

\section*{Practice Problem Solutions}

\subsection*{Problem A (Easier)}

\textbf{Given:} $(n,e)=(77,7)$, plaintexts $M_1=2$, $M_2=3$.

\textbf{Compute:}
\[
E(2)=2^7\bmod77=128\bmod77=51,
\]
\[
E(3)=3^7\bmod77=2187\bmod77=31.
\]

Now multiply:
\[
E(2)\cdot E(3)\bmod77=51\cdot31=1581\bmod77=45.
\]

Direct encryption of product:
\[
E(2\cdot3)=E(6)=6^7\bmod77=279936\bmod77=45.
\]
\textbf{Verification complete. ✅}

\bigskip

\subsection*{Problem B (Similar Difficulty)}

\textbf{Given:} $(n,e)=(2537,13)$ and $M_1=14$, $M_2=15$.  
We’ll verify \(E(M_1)\cdot E(M_2)\equiv E(M_1M_2)\).

Compute quickly (by calculator or modular exponentiation):
\[
E(14)=14^{13}\bmod2537=1043,\quad E(15)=15^{13}\bmod2537=2059.
\]
Multiply:
\[
E(14)\cdot E(15)\bmod2537 = 1043\cdot2059=2143537\bmod2537=1763.
\]
Now check direct encryption:
\[
E(14\cdot15)=E(210)=210^{13}\bmod2537=1763.
\]
\textbf{They agree! ✅}

RSA works multiplicatively even for larger $n$.

\bigskip

\subsection*{Problem C (Harder Challenge — Discussion)}

\textbf{1. Why RSA cannot be additively homomorphic:}  
Because modular exponentiation distributes over multiplication, not addition.  
\[
(M_1+M_2)^e \neq M_1^e + M_2^e \ (\bmod n).
\]
Exponentiation turns addition into a nonlinear operation — there’s no way to extract $M_1+M_2$ from $M_1^e$ and $M_2^e$ without decryption.

\textbf{2. Implications for cloud computing:}  
Since addition (and general operations) can’t be done directly on ciphertexts, RSA can’t power secure, fully remote computation on encrypted data.  
You’d need to decrypt first — which breaks confidentiality.

\textbf{3. The importance of Gentry’s breakthrough (2009):}  
Craig Gentry’s Fully Homomorphic Encryption (FHE) allowed \emph{any} computation — additions and multiplications — directly on ciphertexts.  
This was revolutionary: it meant that a cloud service could compute on encrypted data without ever seeing the plaintext.  
His work, based on lattice cryptography, earned him both the ACM Grace Murray Hopper Award and a MacArthur Fellowship.

\bigskip
\hrule
\vspace{0.5em}

\section*{Teaching Reflections and Extensions}

\begin{itemize}
  \item \textbf{Connection to Algebra:}  
    Homomorphism in math means “structure-preserving map.” RSA literally preserves multiplication under encryption — a bridge between abstract algebra and applied security.

  \item \textbf{Security Note:}  
    While the homomorphic property is elegant, it also makes RSA vulnerable to certain attacks if used without padding (e.g., chosen-ciphertext attacks).  
    In real-world applications, RSA is always combined with secure padding like OAEP to prevent misuse.

  \item \textbf{Historical Insight:}  
    The dream of computing on encrypted data started with these “partial” properties.  
    Gentry’s 2009 thesis made that dream real, launching a new field of post-quantum cryptography.

  \item \textbf{Encouragement for Students:}  
    If you’ve followed this far — congratulations! You’ve just touched the frontier where algebra, computer science, and cybersecurity meet.  
    Homomorphic encryption is one of the most exciting frontiers in modern cryptography.
\end{itemize}

\end{document}

