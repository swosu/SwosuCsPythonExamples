\documentclass[12pt]{article}
\usepackage[margin=1in]{geometry}
\usepackage{amsmath,amssymb}
\setlength{\parskip}{0.6em}
\setlength{\parindent}{0pt}

\begin{document}
{\large \textbf{Discrete Structures \quad Chapter 4.6 — Cryptography}}

\hrule
\vspace{0.6em}

\section*{Solutions — Example 4 Affine Cipher}

\subsection*{Example Walk-Through}
\(K\to10,\; f(p)=(7p+3)\bmod26.\)
\[
f(10) = (7\cdot10 + 3)\bmod26 = 73\bmod26 = 21.
\]
\(21\) corresponds to \(V\).  
\boxed{K \to V}

---

\subsection*{Problem A}
\(C=2.\)
\[
f(2) = (3\cdot2 + 1)\bmod26 = 7.
\]
\(7\rightarrow H.\)
\boxed{C\to H}

\subsection*{Problem B}
\(H=7.\)
\[
f(7) = (5\cdot7 + 7)\bmod26 = 42\bmod26 = 16.
\]
\(16\rightarrow Q.\)
\boxed{H\to Q}

\subsection*{Problem C}
Encrypt \texttt{DOG} with \(f(p)=(11p+8)\bmod26.\)

\[
\begin{array}{rclcl}
D&=&3 &\Rightarrow& (11\cdot3+8)\bmod26 = 41\bmod26=15\rightarrow P\\
O&=&14&\Rightarrow& (11\cdot14+8)\bmod26 =162\bmod26=6\rightarrow G\\
G&=&6 &\Rightarrow& (11\cdot6+8)\bmod26 =74\bmod26=22\rightarrow W
\end{array}
\]
\boxed{\texttt{DOG}\rightarrow\texttt{PGW}}

\subsection*{Reflection Answer}
If \(a\) shares a factor with 26, then some letters collapse to the same output (no unique
inverse), making decryption impossible.  Only when \(\gcd(a,26)=1\) does the cipher remain
bijective and reversible.

\end{document}

