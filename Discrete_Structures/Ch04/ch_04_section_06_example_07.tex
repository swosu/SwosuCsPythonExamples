\documentclass[12pt]{article}
\usepackage[margin=1in]{geometry}
\usepackage{amsmath,amssymb}
\usepackage{enumitem}
\usepackage{setspace}
\setlength{\parskip}{0.6em}
\setlength{\parindent}{0pt}

% Working space box
\newcommand{\workbox}[1]{\fbox{\parbox{\dimexpr\textwidth-2\fboxsep-2\fboxrule\relax}{\vspace{#1}}}}

\begin{document}
{\large \textbf{Discrete Structures \quad Chapter 4.6 — Cryptography}}

\hrule
\vspace{0.5em}

\section*{Example 7 (Worksheet) — Shift Ciphers as a Cryptosystem}

\textbf{Goal.} Describe the family of shift ciphers in the formal language of a cryptosystem.

\subsection*{The Big Idea: What’s a Cryptosystem?}
A \textbf{cryptosystem} is a mathematical framework describing how messages are encrypted and decrypted.  
Formally, it’s written as a 5-tuple:
\[
(\mathcal{P},\ \mathcal{C},\ \mathcal{K},\ \mathcal{E},\ \mathcal{D})
\]
where each symbol represents a part of the encryption ecosystem:

\begin{itemize}[leftmargin=2em]
  \item $\mathcal{P}$ – the set of possible \emph{plaintexts}
  \item $\mathcal{C}$ – the set of possible \emph{ciphertexts}
  \item $\mathcal{K}$ – the \emph{keyspace}, all keys that can be used
  \item $\mathcal{E}$ – the set of \emph{encryption functions}
  \item $\mathcal{D}$ – the set of \emph{decryption functions}
\end{itemize}

The golden rule of any cryptosystem is:
\[
D_k(E_k(p)) = p \quad \text{for every plaintext } p.
\]
That means: decrypting an encrypted message must always give you back the original.

\subsection*{Step 1 — Translate the Language of Letters into Math}
Each letter of the alphabet is assigned a number in $\mathbb{Z}_{26}$ (the integers 0–25 mod 26).

\[
\text{A}=0, \ \text{B}=1, \ \ldots, \ \text{Z}=25
\]

A message like \texttt{HELLO} becomes \([7,4,11,11,14]\).

\subsection*{Step 2 — Define the Shift Cipher Functions}
To encrypt, we \emph{add} a fixed key $k$ mod 26:
\[
E_k(p) = (p + k) \bmod 26.
\]
To decrypt, we \emph{subtract} the same $k$ mod 26:
\[
D_k(c) = (c - k) \bmod 26.
\]

\subsection*{Step 3 — Describe the Family of Shift Ciphers as a Cryptosystem}
Putting it all together:

\[
\begin{aligned}
\mathcal{P} &= \mathcal{C} = \text{all strings of elements in } \mathbb{Z}_{26}, \\
\mathcal{K} &= \mathbb{Z}_{26}, \\
\mathcal{E} &= \{\,E_k(p) = (p + k) \bmod 26 \mid k \in \mathbb{Z}_{26}\,\}, \\
\mathcal{D} &= \{\,D_k(c) = (c - k) \bmod 26 \mid k \in \mathbb{Z}_{26}\,\}.
\end{aligned}
\]

This means each possible shift $k$ defines one member of the family of shift ciphers.

\subsection*{Step 4 — Check the “Undo” Property}
To verify that encryption and decryption work as a matched pair:
\[
D_k(E_k(p)) = (p + k - k) \bmod 26 = p.
\]
So every message can be perfectly recovered.

\subsection*{Tips \& Common Pitfalls}
\begin{itemize}[leftmargin=1.25em]
  \item Don’t confuse the “keyspace” $\mathcal{K}$ with a single key $k$. The keyspace is the entire set of possible shifts.
  \item Forgetting to take mod 26 is a very common mistake.
  \item A shift cipher is \emph{not secure} — only 26 possible keys! We study it to understand the structure of more complex systems.
\end{itemize}

\bigskip
\hrule
\vspace{0.5em}

\section*{Practice — Your Turn!}

\textbf{Problem A (Easier).}  
For a shift cipher with \(k = 5\), write down \(E_k(p)\) and \(D_k(c)\).  
Explain in your own words what “mod 26” ensures.  
\workbox{2cm}

\textbf{Problem B (Similar).}  
Let \(p = 19\) (the letter T) and \(k = 7\).  
Compute \(E_k(p)\) and translate it back into a letter.  
Then apply \(D_k\) to check that you get back T.  
\workbox{2.2cm}

\textbf{Problem C (Harder).}  
Write the complete 5-tuple \((\mathcal{P},\mathcal{C},\mathcal{K},\mathcal{E},\mathcal{D})\) for a system that works on
uppercase English letters and digits (0–9).  
What changes?  
\workbox{2.6cm}

\bigskip
\textbf{Reflection.}  
How does writing cryptography in formal notation help us build new systems in the future?  
\workbox{1.4cm}

\end{document}

