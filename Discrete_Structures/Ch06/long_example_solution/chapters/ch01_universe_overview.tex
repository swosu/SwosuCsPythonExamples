\chapter{Tiny Tactics Arena: Universe Overview and Design Brief}

\section{Story Hook: Welcome to the Arena}

You and your friend have accidentally invented a tactics game.

It started as a joke in the Discord:
\begin{quote}
  ``What if we made a tiny turn-based arena where every character is
  wildly unbalanced, but we use math to pretend it is fair?''
\end{quote}

Two hours later, there is:
\begin{itemize}
  \item a roster of heroes (some cool, some cursed),
  \item a pile of artifacts with questionable side effects,
  \item a skill tree that looks like it was designed at 3am,
  \item and a spreadsheet named \texttt{balance\_vFINAL\_final\_ACTUAL.xlsx}.
\end{itemize}

This chapter is where we stop guessing and actually \emph{specify} the universe
we are playing in. We will give it a name:

\begin{center}
  \textbf{Tiny Tactics Arena}
\end{center}

and we will turn the chaos into symbols, parameters, and precise questions.
If someone reads only this chapter, they should understand:
\begin{itemize}
  \item what objects exist in our world (heroes, artifacts, skills, matches),
  \item what can be customized by a player,
  \item where the math lives (permutations, combinations, stars-and-bars,
        probability),
  \item and what code we will eventually write to explore and analyze the game.
\end{itemize}


\section{Entities and Parameters}

At the heart of Tiny Tactics Arena are a few types of objects.

\subsection*{Heroes}

We have a roster of heroes. Each hero has:
\begin{itemize}
  \item a name (for flavor: ``Nova'', ``Glitch'', ``Tanktop Tim''),
  \item a \emph{role} (e.g., damage, support, tank),
  \item a \emph{skill tree} they can invest points into.
\end{itemize}

We will treat the roster size as a parameter:
\[
  H = \text{number of available heroes in the full roster}.
\]

\subsection*{Teams}

A player enters a match with a team of fixed size:
\[
  T = \text{number of heroes on a team}.
\]

For example, in a ``3v3'' mode, $T = 3$.

Sometimes we will care only which heroes are on the team (as a \emph{set} of size $T$).
Other times, we will care about the \emph{order} in which they act (initiative order).

\subsection*{Artifacts}

Artifacts are items that can be equipped before a match:
\begin{itemize}
  \item Each artifact has a name and effect
        (``+2 damage'', ``double heal'', ``50\% chance to explode'').
  \item A player can bring up to $A_{\max}$ artifacts into a match.
\end{itemize}

We let:
\[
  A = \text{number of distinct artifact types in the game},
\]
and
\[
  A_{\max} = \text{maximum number of artifacts a team can equip}.
\]

We will consider simple rules first:
\begin{itemize}
  \item At most one copy of each artifact.
  \item Order of artifacts does not matter (you either bring it or you do not).
\end{itemize}

\subsection*{Skill Points}

Each hero has a small skill tree. To keep things combinatorially friendly, we will say:
\begin{itemize}
  \item A hero has $k$ skill lines (for example: attack, defense, utility).
  \item Each hero gets $S$ total skill points to distribute across those lines.
\end{itemize}

We let:
\[
  k = \text{number of skill lines for a hero}, \qquad
  S = \text{total skill points available for that hero}.
\]

The skill tree, for counting purposes, is just a way to distribute $S$ identical points into
$k$ labeled buckets (lines).


\subsection*{Matches and Randomness}

A match consists of:
\begin{itemize}
  \item Two teams (you vs.\ opponent), each of size $T$.
  \item An initiative order telling us who acts first, second, etc.
  \item Random events like critical hits, misses, or random artifact effects.
\end{itemize}

We will model some of this randomness in a very simplified way (e.g., each attack
hits with probability $p$, crits with probability $q$, etc.) so that we can:
\begin{itemize}
  \item compute exact probabilities in small toy versions, and
  \item simulate matches using Monte Carlo in Python.
\end{itemize}


\section{Core Mechanics in Math-Friendly Language}

Now we strip away the flavor and look at the \emph{combinatorial skeleton}
of the game.

\subsection*{Teams as Sets or Sequences}

A team configuration can be viewed in two ways:
\begin{enumerate}[label=\arabic*.]
  \item As a \textbf{set} of $T$ heroes chosen from $H$:
        order does not matter, only who is on the team.
  \item As a \textbf{sequence} (permutation) of $T$ heroes:
        we care about the order in which they act.
\end{enumerate}

This means we will use:
\begin{itemize}
  \item \emph{combinations} to count the number of possible teams,
  \item \emph{permutations} to count initiative orders and lineups.
\end{itemize}

\subsection*{Artifact Loadouts}

An artifact loadout is:
\begin{itemize}
  \item a subset of the $A$ artifact types,
  \item with size at most $A_{\max}$,
  \item where order does not matter.
\end{itemize}

So loadouts are also counted with combinations, although we may have to add up
several cases (size 0, 1, 2, \dots)

In more advanced versions of the project, you could allow:
\begin{itemize}
  \item multiple copies of the same artifact (a multiset),
  \item per-hero artifact slots (combinatorics inside combinatorics),
  \item caps or categories (at most one ``legendary'', etc.).
\end{itemize}


\subsection*{Skill Allocations as Stars and Bars}

Skill point allocations are a perfect place to use the \emph{stars and bars} technique.

We imagine:
\begin{itemize}
  \item $S$ identical ``stars'' (skill points),
  \item $k$ labeled ``buckets'' (skill lines),
  \item and we count how many ways to drop the $S$ stars into $k$ buckets.
\end{itemize}

This is exactly the classic stars-and-bars setup you see in counting problems:
\[
  \text{number of allocations} = \binom{S + k - 1}{k - 1},
\]
under the assumption that each skill line can receive zero or more points.

We can also add constraints:
\begin{itemize}
  \item minimum points in some line (e.g., at least 1 point in defense),
  \item caps on lines (e.g., at most 3 points in any single line).
\end{itemize}


\subsection*{From Counting to Probability}

Once we know how many configurations or outcomes there are, we can ask questions like:
\begin{itemize}
  \item What is the probability that a randomly generated team has at least
        one healer?
  \item What is the probability that a randomly chosen artifact loadout includes
        a particular legendary item?
  \item Under a simple model of combat, what is the probability that a fight ends
        in 3 turns or fewer?
\end{itemize}

For small cases, we will compute probabilities exactly using:
\[
  P(E) = \frac{\lvert E \rvert}{\lvert \Omega \rvert},
\]
where $\Omega$ is the sample space of all equally likely outcomes and $E$ is
the event we care about.

For larger cases, we will use Python to approximate $P(E)$ by simulation:
run many matches, count how often $E$ happens, and look at the relative frequency.


\section{Required Math Questions for This Universe}

To keep the project focused, Tiny Tactics Arena is designed to support at least
the following five mathematical questions.

\subsection*{1. Permutation Question}

\textbf{Example prompt:}  
Given a team of $T$ distinct heroes, in how many ways can we order them in an
initiative track from first to $T$th?

Possible extensions:
\begin{itemize}
  \item Some heroes must act before others (partial order).
  \item One hero always acts last because they are slow but dramatic.
\end{itemize}

\subsection*{2. Combination Question}

\textbf{Example prompt:}  
Given $A$ distinct artifacts and a cap of $A_{\max}$ on how many a team can bring,
in how many ways can a team choose its artifact loadout?

Variants:
\begin{itemize}
  \item Exactly $A_{\max}$ artifacts, no fewer.
  \item At most one artifact of each type vs.\ allowing duplicates.
  \item Artifacts restricted by hero role (certain artifacts only usable by tanks, etc.).
\end{itemize}

\subsection*{3. Stars-and-Bars Question}

\textbf{Example prompt:}  
A hero has $k$ skill lines and $S$ total skill points. In how many ways can we
allocate those $S$ points across the $k$ lines?

Variants:
\begin{itemize}
  \item Each line must receive at least one point.
  \item No line can receive more than some cap $C$.
\end{itemize}

\subsection*{4. Probability Question}

\textbf{Example prompt:}  
Assume a very simple combat model: each attack has probability $p$ of hitting and,
on a hit, probability $q$ of being a critical hit. For a fixed sequence of $n$
attacks in a match, what is the probability that we see:
\begin{itemize}
  \item exactly $k$ hits,
  \item at least one critical hit,
  \item or some other event your story cares about?
\end{itemize}

You might use binomial coefficients here: the probability of exactly $k$ hits in $n$
independent trials is:
\[
  \binom{n}{k} p^k (1-p)^{n-k}.
\]

\subsection*{5. AI / Machine Learning Question}

\textbf{Example prompt:}  
We generate a large dataset of simulated matches under different team and artifact
configurations. Can we:
\begin{itemize}
  \item detect imbalanced heroes or artifacts,
  \item or train a simple classifier to predict which team is favored to win?
\end{itemize}

In the full project, this could involve:
\begin{itemize}
  \item encoding matches as rows in a CSV file (\texttt{data/matches.csv}),
  \item using Python (and possibly \texttt{scikit-learn}) to fit a simple model,
  \item interpreting the learned parameters as ``balance hints'' for the game.
\end{itemize}


\section{Planned Code Artifacts and File Outputs}

To support these questions, we will create a small collection of Python scripts.

Each script will:
\begin{itemize}
  \item live in the \texttt{scripts/} directory,
  \item take simple command-line arguments (e.g., \texttt{-{}-heroes 6 -{}-team-size 3}),
  \item print a short summary line to standard output,
  \item and write the interesting results to files in \texttt{data/}, \texttt{figures/},
        or \texttt{files/}.
\end{itemize}

Here is the planned lineup.

\subsection*{Script 1: \texttt{turn\_orders.py}}

Purpose:
\begin{itemize}
  \item Compute and/or enumerate possible initiative orders for a team.
  \item Optionally write all orders (for small $T$) to a CSV file
        (\texttt{data/turn\_orders\_T3.csv}, etc.).
\end{itemize}

Console output example:
\begin{quote}
  \texttt{Team size T=3: 6 possible initiative orders. Data saved to data/turn\_orders\_T3.csv}
\end{quote}

\subsection*{Script 2: \texttt{artifact\_loadouts.py}}

Purpose:
\begin{itemize}
  \item Count and (for small cases) list all artifact loadouts.
  \item Save loadouts to \texttt{data/artifact\_loadouts\_A5\_Amax2.csv}.
\end{itemize}

Console output example:
\begin{quote}
  \texttt{A=5 artifacts, Amax=2: 16 possible loadouts. Results in data/artifact\_loadouts\_A5\_Amax2.csv}
\end{quote}

\subsection*{Script 3: \texttt{skill\_allocations.py}}

Purpose:
\begin{itemize}
  \item Compute the number of ways to allocate $S$ skill points into $k$ lines.
  \item Optionally enumerate allocations for small $S$ and $k$ and write them to CSV.
\end{itemize}

Console output example:
\begin{quote}
  \texttt{k=3 lines, S=4 points: 15 allocations. Sample saved to data/skill\_allocations\_k3\_S4.csv}
\end{quote}

\subsection*{Script 4: \texttt{match\_simulator.py}}

Purpose:
\begin{itemize}
  \item Simulate many Tiny Tactics Arena matches under a simplified combat model.
  \item Record key statistics (winner, turns taken, damage dealt, etc.).
  \item Write all simulated matches to \texttt{data/matches.csv}.
\end{itemize}

Console output example:
\begin{quote}
  \texttt{Simulated 10,000 matches. Win rate: Team A 57.2\%, Team B 42.8\%. Data in data/matches.csv}
\end{quote}

\subsection*{Script 5: \texttt{ai\_balance\_checker.py}}

Purpose:
\begin{itemize}
  \item Read \texttt{data/matches.csv}.
  \item Train a simple model (even just basic statistics at first) to detect:
        \begin{itemize}
          \item which heroes or artifacts appear most often on winning teams,
          \item whether some configuration is overwhelmingly strong.
        \end{itemize}
  \item Write summary results to \texttt{files/balance\_report.txt}
        and possibly plots to \texttt{figures/}.
\end{itemize}

Console output example:
\begin{quote}
  \texttt{Balance report written to files/balance\_report.txt; hero win rates plot saved to figures/hero\_winrates.png}
\end{quote}


\section{Project Roadmap and Expectations}

This chapter is the ``design doc'' for your universe.

In later chapters, we will:
\begin{itemize}
  \item turn each core question into:
        \begin{itemize}
          \item a clean mathematical problem,
          \item a worked-out solution with formulas and explanations,
          \item and a small Python script that generates matching results;
        \end{itemize}
  \item connect the pieces together into a coherent story about the game;
  \item optionally, push into AI/ML territory by analyzing the simulated data.
\end{itemize}

Think of it this way:
\begin{itemize}
  \item The \textbf{math} tells you how big and complex your universe is.
  \item The \textbf{code} lets you actually wander around that universe, sample it,
        and collect data.
  \item The \textbf{story} is what keeps humans (including you) caring about
        the answers.
\end{itemize}


\section*{Checklist for Your Brain}

As you read this and start planning your own version of the project, keep an eye on:
\begin{itemize}
  \item Which parts of your universe are naturally modeled as:
        \begin{itemize}
          \item permutations (order matters),
          \item combinations (order does not),
          \item stars-and-bars (distributing points/resources).
        \end{itemize}
  \item Which questions are purely counting, and which turn into probabilities.
  \item What data you might want to generate and analyze from simulations.
  \item How you could tell a story around all of this
        (``hero balance council'', ``artifact nerf emergency'', etc.).
\end{itemize}

If you can explain your game to a friend in under two minutes, and they can point
to at least three places where ``we should really count how many possibilities there are'',
you are in the right place.

Welcome to Tiny Tactics Arena. Time to start counting.

