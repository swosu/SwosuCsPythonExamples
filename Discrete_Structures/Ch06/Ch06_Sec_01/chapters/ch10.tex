\chapter{Section 6.1.6 — Tree Diagrams}

\section*{Why this section matters}
When a counting problem starts to branch—when every choice opens new sub-choices—our mental arithmetic hits its limits.  
Tree diagrams visualize those branching worlds.  
Each path from the root to a leaf represents one distinct outcome, and the total number of leaves tells us how many possibilities exist.  
In computer science, these trees reappear as recursion traces, decision trees, and game trees—every “if → else” path you’ve ever coded.

\section{Tree Diagrams}

\begin{definition}[Tree Diagram]
A \textbf{tree diagram} is a branching structure that represents sequential choices.
Each vertex (node) shows a decision point, and each edge (branch) corresponds to an outcome of that decision.
Every path from the root to a leaf represents one complete sequence of choices.
\end{definition}

\noindent
Tree diagrams make the invisible visible:
\begin{itemize}
  \item Each branch multiplies choices (Product Rule).
  \item Each split visualizes conditional logic (“if/else” in code).
  \item The number of leaves equals the total number of possible outcomes.
\end{itemize}

\section{Worked Examples}

\begin{example}[Bit strings without consecutive 1s (Rosen Example 22)]
How many bit strings of length 4 contain no two consecutive 1s?
\end{example}

\begin{solution}
Draw a tree where each level adds one bit (0 or 1).  
When a 1 is chosen, the next level cannot start with another 1.  
This yields the valid strings:
\[
0000,\;0001,\;0010,\;0100,\;0101,\;1000,\;1001,\;1010.
\]
Hence, 8 valid bit strings in total.
\end{solution}

\begin{example}[Playoff possibilities (Rosen Example 23)]
A playoff between two teams consists of at most five games.
The first team to win three games wins the playoff.
How many different ways can the playoff unfold?
\end{example}

\begin{solution}
Each game outcome is either W (win) or L (loss).  
The series stops as soon as one team reaches 3 wins.  
Enumerating all branches that terminate at “3 wins” yields 20 distinct series paths.  
Thus, there are 20 ways the playoff can occur.
\end{solution}

\begin{example}[T-shirt varieties (Rosen Example 24)]
Suppose a souvenir shop sells “I ♥ New Jersey” T-shirts in five sizes (S, M, L, XL, XXL) and three colors (white, red, black).  
How many distinct shirts must be stocked?
\end{example}

\begin{solution}
Each shirt is determined by one choice of size and one choice of color.
The tree’s first branch: five sizes.  
Each size branch splits into three color options.  
By the Product Rule:
\[
5\times3 = 15
\]
distinct shirts must be stocked.
\end{solution}

\section{Try It — Easier (solution on the next page)}
A restaurant offers 3 appetizers, 4 main courses, and 2 desserts.  
How many full three-course meals can be made if you choose one of each?

\newpage
\begin{solution}
Each appetizer branch splits into 4 main courses, each of which splits into 2 desserts:
\[
3\times4\times2 = 24.
\]
Hence, 24 possible full meals.
\end{solution}

\section{Challenge — Harder (solution on the next page)}
A security code is three characters long.
Each character is a letter (A–Z) or a digit (0–9).  
However, a code may not contain two consecutive digits.
How many valid codes exist?

\newpage
\begin{solution}
We use a branching tree or recursive reasoning.

Let $L_n$ = codes of length $n$ ending with a letter,  
and $D_n$ = codes of length $n$ ending with a digit.  

Recurrence:
\[
L_n = 26(L_{n-1}+D_{n-1}), \qquad D_n = 10L_{n-1}.
\]
Base case: $L_1=26$, $D_1=10$.

Compute:
\[
\begin{aligned}
L_2 &= 26(26+10)=936, & D_2 &= 10(26)=260,\\
L_3 &= 26(936+260)=31{,}096, & D_3 &= 10(936)=9{,}360.
\end{aligned}
\]
Total = $L_3+D_3=40{,}456$ valid codes.
\end{solution}

\section{Python Demonstration}

\begin{lstlisting}[language=Python,caption={Tree diagram counting via recursion}]
def codes(n):
    """Count length-n codes with no consecutive digits."""
    if n == 1:
        return 26, 10  # (end-with-letter, end-with-digit)
    L_prev, D_prev = codes(n-1)
    L = 26 * (L_prev + D_prev)
    D = 10 * L_prev
    return L, D

L3, D3 = codes(3)
print("Valid 3-char codes:", L3 + D3)
\end{lstlisting}

\section*{Reflection}
Tree diagrams transform abstract multiplication and subtraction rules into visual reasoning.
They help debug logic—each branch exposes a case that might otherwise hide in algebra.
For programmers, they mirror recursive call trees and state machines.
When you can see the structure, you can count the structure.
