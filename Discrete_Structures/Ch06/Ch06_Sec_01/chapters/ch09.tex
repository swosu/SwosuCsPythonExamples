\chapter{Section 6.1.5 — The Division Rule}

\section*{Why This Section Matters}
Sometimes we intentionally overcount by labeling or rotating, and then need to \emph{divide out} the symmetry.  
The \textbf{Division Rule} tells us when dividing by a constant number of equivalent descriptions yields the correct number of unique outcomes.

\section{The Division Rule}
\begin{definition}[Division Rule]
If each distinct outcome of a task is represented by exactly $d$ different descriptions 
(that is, there is a $d$-to-$1$ mapping from descriptions to outcomes), 
and there are $n$ total descriptions, then the number of distinct outcomes is:
\[
\frac{n}{d}.
\]
\end{definition}

\noindent
\textbf{Set/function viewpoint.}  
If $A$ is the set of descriptions and $B$ the set of outcomes, and 
$f\!:\!A\to B$ is onto with each $b\in B$ having exactly $d$ preimages in $A$, then $|B| = |A|/d$.

\medskip
\noindent
\textbf{Sanity checks and gotchas.}
\begin{itemize}
  \item The rule only works when \emph{every} outcome has the \emph{same} number $d$ of representations.
  \item “Equivalent” must form an \emph{equivalence relation} (reflexive, symmetric, transitive) so that $A$ splits into equal-sized buckets.
  \item If different outcomes have different numbers of representations, you can’t use a single division—this is where tools like Burnside’s Lemma or the orbit–stabilizer theorem appear.
\end{itemize}

---

\section{Worked Examples}

\begin{example}[Counting cows from legs (Rosen-flavored)]
An automated system counts exactly $572$ cow legs in a pasture.  
Assuming only cows are present and each cow has four legs, how many cows are in the pasture?
\end{example}

\begin{solution}
Descriptions $A$: individual \emph{legs} ($n = 572$).  
Outcomes $B$: individual \emph{cows}.  
Each cow contributes $d = 4$ leg-descriptions, so
\[
|B| = \frac{572}{4} = 143.
\]
\end{solution}

---

\begin{example}[Circular seatings up to rotation]
How many distinct seatings of four distinct people around a circular table are there, 
if seatings that differ only by a rotation are considered the same?
\end{example}

\begin{solution}
Linear seatings: $4!$ descriptions.  
Rotating the circle doesn’t change the pattern, and there are $d = 4$ rotations.  
Thus
\[
\frac{4!}{4} = 3! = 6.
\]
\end{solution}

---

\begin{example}[Unordered pairs from ordered pairs]
How many undirected edges (unordered pairs) exist on $n$ labeled vertices?
\end{example}

\begin{solution}
Ordered pairs of distinct vertices: $n(n - 1)$.  
Each undirected edge $\{u, v\}$ corresponds to exactly $d = 2$ ordered descriptions $(u, v)$ and $(v, u)$, so
\[
\binom{n}{2} = \frac{n(n - 1)}{2}.
\]
\end{solution}

---

\begin{example}[Anagrams with repeated letters]
How many distinct anagrams does the word \texttt{BALLOON} have?
\end{example}

\begin{solution}
Treat the seven positions as labeled: $7!$ descriptions.  
Swapping the two indistinguishable $L$’s and the two indistinguishable $O$’s doesn’t change the outcome.  
Thus each unique arrangement has $d = 2!\times2!$ equivalent descriptions:
\[
\frac{7!}{2!\,2!} = 1260.
\]
\end{solution}

---

\section{Try It — Easier (solution on next page)}

A warehouse robot counts $1{,}140$ wheels on identical cargo carts.  
Each cart has exactly six wheels.  
Assuming no other wheeled objects are present, how many carts are there?

\clearpage
\begin{solution}
Each cart contributes $d = 6$ wheel-descriptions.  
By the Division Rule:
\[
\frac{1{,}140}{6} = 190 \text{ carts.}
\]
\end{solution}

---

\section{Challenge — Harder (solution on next page)}

Seven distinct programmers sit around a round table.  
Two seatings are the same if one can be rotated into the other (neighbor relationships match).  
How many distinct seatings exist? Then generalize to $n$ people.

\clearpage
\begin{solution}
Linear arrangements: $7!$.  
Each circular seating has exactly $d = 7$ rotations, so
\[
\frac{7!}{7} = 6! = 720.
\]
In general, for $n$ people:
\[
\frac{n!}{n} = (n - 1)!.
\]
\emph{Bonus:} If reflections also count as identical (like a necklace flip), divide by 2 again for $n > 2$.
\end{solution}

---

\section{Python Demonstrations}

\begin{lstlisting}[language=Python, caption={Division Rule demos in Python}]
from math import factorial

def circular_seatings(n):
    """Count seatings up to rotation: n!/n = (n-1)!"""
    return factorial(n) // n

def unordered_pairs(n):
    """Count unordered pairs {u, v}: n(n-1)/2"""
    return n * (n - 1) // 2

print("Circular seatings for 4:", circular_seatings(4))   # 6
print("Circular seatings for 7:", circular_seatings(7))   # 720
for n in [5, 10, 100]:
    print(f"K_{n} has", unordered_pairs(n), "edges")
\end{lstlisting}

---

\section*{Checklist: When Can I Divide?}
\begin{itemize}
  \item I first counted \emph{descriptions}.
  \item I identified an equivalence relation grouping them into outcomes.
  \item Each outcome has the same number $d$ of descriptions.
  \item Therefore, \(\text{outcomes} = \dfrac{\text{descriptions}}{d}\).
\end{itemize}

---

\section*{Reflection}
The Product Rule multiplies possibilities.  
The Subtraction Rule removes overlap.  
The Division Rule removes symmetry.  
Together, they form a trinity of clarity for counting—and they echo everywhere in computer science:
hash buckets, graph edges, canonical forms, and seating symmetries alike.

