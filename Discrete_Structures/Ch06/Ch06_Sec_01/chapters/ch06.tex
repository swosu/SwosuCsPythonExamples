\chapter{Section 6.3 — Permutations and Combinations}

\section{Counting Arrangements and Selections}

Two questions shape almost every counting problem:

\begin{enumerate}
  \item Does \emph{order} matter?
  \item Can we \emph{reuse} elements?
\end{enumerate}

If order matters, we count \textbf{permutations}.  
If order does not matter, we count \textbf{combinations}.  
Let’s explore both ideas carefully.

---

\begin{example}[Permutations — When Order Matters]
You have three distinct books and want to know how many ways they can be arranged on a shelf.

\textbf{Reasoning.}
\begin{itemize}
  \item 3 choices for the first position,
  \item 2 choices for the second,
  \item 1 choice for the last.
\end{itemize}

By the product rule:
\[
3 \times 2 \times 1 = 6.
\]

We call this $3!$ (“three factorial”).  
In general, $n! = n \times (n - 1) \times \cdots \times 1$.
\end{example}

\begin{solution}
If the books are labeled A, B, C, then the six permutations are:
\[
\text{ABC, ACB, BAC, BCA, CAB, CBA.}
\]
Factorials grow explosively—$10! = 3{,}628{,}800$—so even small sets lead to huge numbers of arrangements.
\end{solution}

---

\begin{example}[Permutations of Subsets]
How many ways can you arrange four of ten distinct students in a line?

\textbf{Reasoning.}
You are selecting four \emph{different} people, and order matters:
\[
P(10,4) = \frac{10!}{(10-4)!} = \frac{10!}{6!} = 10 \times 9 \times 8 \times 7 = 5040.
\]
\end{example}

\begin{solution}
$P(n,r)$ counts the number of ways to arrange $r$ objects from $n$.  
Python provides \texttt{math.perm(n, r)} to compute this directly.
\end{solution}

---

\section{Combinations — When Order Does Not Matter}

When order no longer matters, we divide out repeated arrangements that represent the same group.

\begin{example}[Combinations Formula]
From ten students, how many unique groups of four can be formed?

\textbf{Reasoning.}
\[
C(10,4) = \frac{10!}{4! \times (10-4)!} = \frac{10!}{4! \, 6!} = 210.
\]
\end{example}

\begin{solution}
Use \texttt{math.comb(10, 4)} in Python to verify.  
Combinations appear in team selection, probability, and sampling.
\end{solution}

---

\subsection*{Practice Problems — Permutations and Combinations}

\textbf{1. Easier:}  
How many ways can five different books be arranged on a shelf?

\newpage
\textbf{Solution:}  
\[
5! = 5 \times 4 \times 3 \times 2 \times 1 = 120.
\]

---

\textbf{2. Moderate:}  
From a group of eight students, how many teams of three can be formed?

\newpage
\textbf{Solution:}  
\[
C(8,3) = \frac{8!}{3! \, 5!} = 56.
\]

---

\textbf{3. Stretch:}  
How many four-digit PINs can be created using digits 0–9 if:
\begin{enumerate}
  \item digits cannot repeat, and
  \item digits may repeat?
\end{enumerate}

\newpage
\textbf{Solution:}  
\[
\text{(a) No repeats: } P(10,4) = \frac{10!}{6!} = 5040,
\qquad
\text{(b) Repeats allowed: } 10^4 = 10{,}000.
\]
Repetition multiplies the space of possibilities dramatically.
\newpage

---

\section{Python Demonstrations}

\begin{lstlisting}[language=Python, caption={Exploring permutations and combinations in Python}]
"""
COMSC 2043 - Section 6.3: Permutations & Combinations
Author: Jeremy Evert
"""

import math

# Factorials
print("5! =", math.factorial(5))

# Permutations (order matters)
print("P(10,4) =", math.perm(10, 4))

# Combinations (order doesn't matter)
print("C(10,4) =", math.comb(10, 4))

# Comparing repetition
no_repeats = math.perm(10, 4)
with_repeats = 10**4
print(f"No repeats: {no_repeats:,}")
print(f"With repeats: {with_repeats:,}")
\end{lstlisting}

---

\section*{Reflection}

Permutations and combinations form the twin pillars of counting theory.  
Adding order multiplies possibilities; ignoring it divides them out.  
These ideas echo through every algorithm, from password generation to genetic code.  

\[
\text{Permutations: } P(n,r)=\frac{n!}{(n-r)!}, 
\quad
\text{Combinations: } C(n,r)=\frac{n!}{r!(n-r)!}.
\]

Understanding the tension between “order” and “group” transforms abstract formulas into tools for real-world reasoning.

