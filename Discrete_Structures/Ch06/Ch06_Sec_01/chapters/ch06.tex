\chapter{Section 6.3 — Permutations and Combinations}

\section{Counting Arrangements and Selections}

When counting, we often face two big questions:

\begin{enumerate}
  \item Does \emph{order} matter?
  \item Can we \emph{reuse} elements?
\end{enumerate}

If order matters, we are dealing with \textbf{permutations}.  
If order does not matter, we are counting \textbf{combinations}.  
Let’s unpack each idea carefully.

---

\begin{example}[Permutations — When Order Matters]
Suppose you have three different books and you want to know how many ways
they can be arranged on a shelf.

\textbf{Reasoning.}
\begin{itemize}
  \item 3 choices for the first position.
  \item 2 choices for the second position.
  \item 1 choice for the last position.
\end{itemize}

By the product rule:
\[
3 \times 2 \times 1 = 6.
\]

We call this $3!$ (“three factorial”).  
In general, $n! = n \times (n - 1) \times \cdots \times 1$.

\end{example}

\begin{solution}
If you label the books A, B, and C, the six permutations are:
\[
\text{ABC, ACB, BAC, BCA, CAB, CBA.}
\]
Factorials grow explosively—$10! = 3{,}628{,}800$—so a few extra items
make a huge difference!
\end{solution}

---

\begin{example}[Permutations of Subsets]
How many ways can you arrange 4 of 10 distinct students in a line?

\textbf{Reasoning.}
You are selecting 4 \emph{different} people, and the order matters.
\[
P(10, 4) = \frac{10!}{(10 - 4)!} = \frac{10!}{6!} = 10 \times 9 \times 8 \times 7 = 5040.
\]
\end{example}

\begin{solution}
$P(n, r)$ counts the number of ways to arrange $r$ objects from a set of $n$.
Python’s \texttt{math.perm(n, r)} computes this directly.
\end{solution}

---

\section{Combinations — When Order Does Not Matter}

When order no longer matters, we divide out the repeated arrangements
that represent the same group.

\begin{example}[Combinations Formula]
From 10 students, how many groups of 4 can you form?

\textbf{Reasoning.}
\[
C(10, 4) = \frac{10!}{4! \times (10 - 4)!} = \frac{10!}{4! \, 6!} = 210.
\]
\end{example}

\begin{solution}
In Python, use \texttt{math.comb(10, 4)} to get the same answer.
Combinations appear in probability, lotteries, and team selections.
\end{solution}

---

\subsection*{Practice Problems — Permutations and Combinations}

\textbf{1. Easier:}  
How many ways can 5 different books be arranged on a shelf?

\newpage
\textbf{Solution:}  
\[
5! = 5 \times 4 \times 3 \times 2 \times 1 = 120.
\]

---

\textbf{2. Moderate:}  
From a group of 8 students, how many teams of 3 can be formed?

\newpage
\textbf{Solution:}  
\[
C(8, 3) = \frac{8!}{3! \, 5!} = 56.
\]

---

\textbf{3. Stretch:}  
How many 4-digit PINs can be formed using digits 0–9 if:
\begin{enumerate}
  \item Digits cannot repeat.
  \item Digits may repeat.
\end{enumerate}

\newpage
\textbf{Solution:}  
\[
\text{(a) No repeats: } P(10, 4) = \frac{10!}{6!} = 5040.
\]
\[
\text{(b) Repeats allowed: } 10^4 = 10{,}000.
\]
This comparison highlights how reuse (repetition) increases the count dramatically.

---

\section{Python Demonstrations}

\begin{lstlisting}[language=Python, caption={Exploring permutations and combinations in Python}]
"""
COMSC 2043 - Section 6.3: Permutations & Combinations
"""

import math

# 1. Factorials
print("5! =", math.factorial(5))

# 2. Permutations (order matters)
print("Permutations of 10 choose 4:", math.perm(10, 4))

# 3. Combinations (order does not matter)
print("Combinations of 10 choose 4:", math.comb(10, 4))

# 4. Comparing repetition
no_repeats = math.perm(10, 4)
with_repeats = 10**4
print(f"No repeats: {no_repeats}, With repeats: {with_repeats}")
\end{lstlisting}

---

\section*{Reflection}

Permutations and combinations represent the heart of counting theory.
They explain why adding order multiplies possibilities—and why ignoring it divides them out.  
When students grasp this duality, they start to see connections
between probability, coding loops, and even DNA patterns.

Remember:  
\[
\text{Permutations: } P(n, r) = \frac{n!}{(n - r)!}, \qquad
\text{Combinations: } C(n, r) = \frac{n!}{r!(n - r)!}.
\]
