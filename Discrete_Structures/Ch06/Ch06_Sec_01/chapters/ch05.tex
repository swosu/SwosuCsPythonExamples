\chapter{Section 6.2 — DNA, Genomes, and the Sum Rule}

\section{DNA and Combinatorics}

DNA (deoxyribonucleic acid) and RNA (ribonucleic acid) encode the very instructions of life.  
Each strand of DNA consists of \emph{nucleotides}, with four possible bases:
\[
\text{A (adenine)}, \quad \text{C (cytosine)}, \quad \text{G (guanine)}, \quad \text{T (thymine)}.
\]
In RNA, thymine (T) is replaced by uracil (U).  
These bases form long chains whose order determines genes and, ultimately, the proteins that define an organism.

\medskip
From a combinatorial perspective, each base position can be filled in one of four ways.  
Hence, a DNA strand with $n$ bases corresponds to $4^n$ possible sequences — the \textbf{product rule} in biological disguise.

---

\begin{example}[Encoding Amino Acids]
Proteins are built from amino acids. Humans use 22 essential amino acids, and each amino acid is encoded by a \emph{codon} — a short sequence of bases.

\textbf{Reasoning:}
\begin{itemize}
  \item One base: $4^1 = 4$ possibilities (not enough).
  \item Two bases: $4^2 = 16 < 22$ (still too few).
  \item Three bases: $4^3 = 64$ (more than enough).
\end{itemize}

Therefore, a codon must consist of three bases, providing $64$ distinct triplets to encode 22 amino acids.  
This redundancy helps prevent catastrophic errors from single-base mutations.
\end{example}

\begin{solution}
Nature, like a clever coder, includes redundancy and error recovery in its data encoding.  
The product rule explains why three-base codons are the smallest possible unit that can represent all amino acids.
\end{solution}

---

\begin{example}[Counting DNA Sequences]
DNA length varies dramatically among organisms:
\begin{itemize}
  \item Simple organisms (e.g., bacteria): $10^5$–$10^7$ bases.
  \item Complex organisms (e.g., mammals): $10^8$–$10^{10}$ bases.
\end{itemize}

By the product rule, a DNA molecule of length $n$ has $4^n$ possible base sequences:
\[
4^{10^5} \quad \text{for bacteria}, \qquad 4^{10^8} \quad \text{for mammals.}
\]
These quantities are astronomically large, illustrating why genetic diversity is effectively limitless.
\end{example}

\begin{solution}
Even though DNA uses just four symbols, its combinatorial potential is enormous.  
This exponential explosion of possible sequences ensures endless biological variety — one reason every living thing is unique.
\end{solution}

---

\subsection*{Practice Problems — DNA and Counting}

\textbf{1. Easier:}  
How many distinct RNA sequences of length four are possible?

\newpage
\textbf{Solution:}  
Each position has four choices (A, C, G, U):
\[
4^4 = 256 \text{ possible sequences.}
\]

---

\textbf{2. Moderate:}  
A gene contains 12 bases. How many unique base sequences can it have?

\newpage
\textbf{Solution:}  
\[
4^{12} = 16{,}777{,}216.
\]
Even a short gene can produce over sixteen million variations.

---

\textbf{3. Stretch:}  
A virus has a genome with 30,000 bases. Estimate $4^{30{,}000}$ in powers of ten.

\newpage
\textbf{Solution:}  
\[
4^{30{,}000} = (2^2)^{30{,}000} = 2^{60{,}000} \approx 10^{18{,}000}.
\]
That’s a one followed by eighteen thousand zeros — a vast genetic universe.


---

\section{Introducing the Sum Rule}

So far, our counting used multiplication — combining independent choices.  
But what if we have \emph{mutually exclusive} options, and we must choose exactly one?  
That’s where the \textbf{sum rule} comes in.

\begin{definition}[Sum Rule]
If a task can be done in $n_1$ ways or in $n_2$ ways, where the sets of outcomes are disjoint (no overlap),  
then the total number of ways is:
\[
n_1 + n_2.
\]
\end{definition}

---

\begin{example}[Faculty or Student Representative]
Suppose either a member of the mathematics faculty or a mathematics major is chosen to serve on a university committee.

\textbf{Given:}
\[
37 \text{ faculty members,} \qquad 83 \text{ students.}
\]

Since no one is both faculty and student, the total number of possible representatives is:
\[
37 + 83 = 120.
\]
\end{example}

\begin{solution}
The sum rule models disjoint alternatives — you choose one group or the other, not both.  
In programming, this parallels an \texttt{if/else} decision branch.
\end{solution}

---

\begin{example}[Choosing from Multiple Project Lists]
A student can pick one project from any of three lists containing 23, 15, and 19 options, respectively.  
No project appears on more than one list.

\textbf{Solution:}
\[
23 + 15 + 19 = 57
\]
possible projects in total.
\end{example}

\begin{solution}
Each list represents a distinct pool of choices — separate “doors” leading to different sets of outcomes.  
Adding their sizes counts all unique possibilities.
\end{solution}

---

\subsection*{Practice Problems — Applying the Sum Rule}

\textbf{1. Easier:}  
You can adopt either a cat (6 breeds) or a dog (9 breeds) from the shelter.  
How many total options are there?

\newpage
\textbf{Solution:}  
\[
6 + 9 = 15.
\]

---

\textbf{2. Moderate:}  
A restaurant offers 12 vegetarian entrées and 8 meat dishes.  
If you can choose only one, how many possible meals are there?

\newpage
\textbf{Solution:}  
\[
12 + 8 = 20.
\]

---

\textbf{3. Stretch:}  
A student can take one course chosen from:
\begin{itemize}
  \item 5 online classes,
  \item 3 hybrid classes, or
  \item 4 in-person classes.
\end{itemize}
No course appears in more than one category.  
How many total courses are available?

\newpage
\textbf{Solution:}  
\[
5 + 3 + 4 = 12.
\]
This illustrates the extended sum rule:
\[
n_1 + n_2 + \cdots + n_m.
\]

---

\section{Python Demonstrations}

\begin{lstlisting}[language=Python, caption={DNA and Sum Rule Demonstrations}]
"""
COMSC 2043 - DNA and Sum Rule Examples
Author: Jeremy Evert
"""

from math import pow

# DNA combinatorics
def dna_sequences(length):
    """Returns the number of possible DNA (or RNA) sequences of given length."""
    return int(pow(4, length))

# Sum rule demonstrations
faculty, students = 37, 83
project_lists = [23, 15, 19]

print("=== DNA and Sum Rule Examples ===\n")
print(f"DNA sequences of length 4: {dna_sequences(4)}")
print(f"DNA sequences of length 12: {dna_sequences(12):,}")
print(f"Committee choices (faculty or students): {faculty + students}")
print(f"Total projects across lists: {sum(project_lists)}")
print("\nDone!")
\end{lstlisting}

---

\section*{Reflection}

The \textbf{product rule} multiplies possibilities; the \textbf{sum rule} adds alternatives.  
Together they are the foundation of all combinatorics — and of all computation.  
Loops represent multiplication (repetition of actions), while conditional branches represent addition (choosing between paths).  
From DNA sequences to decision trees, counting principles are the quiet mathematics behind every system of order — living or digital.
