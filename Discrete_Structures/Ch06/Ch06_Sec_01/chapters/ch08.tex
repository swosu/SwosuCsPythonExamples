\chapter{Section 6.1.4 — The Subtraction Rule (Avoiding Double Counting)}

\section{Introduction: When Counting Goes Wrong}

By now, we’ve seen how to count outcomes using the \textbf{Product Rule} (for “and”) and the \textbf{Sum Rule} (for “or”).  
But what if we count the same thing twice?  

That’s where the \textbf{Subtraction Rule} comes in. It tells us how to avoid double-counting when two sets overlap.

\begin{definition}[Subtraction Rule]
If a task can be done in $n_1$ ways or in $n_2$ ways, but some outcomes are counted twice because they can be done both ways (there are $n_{1,2}$ overlaps), then the correct total is:
\[
n_1 + n_2 - n_{1,2}.
\]
\end{definition}

\noindent
This simple idea is foundational in combinatorics and probability — and a secret weapon against overcounting errors.

---

\section{Example 1: Bit Strings with Conditions (Rosen Example 18)}

\begin{example}[Counting Bit Strings of Length 8]
How many bit strings of length 8 start with a 1 \emph{or} end with the two bits 00?

\textbf{Solution Outline:}
\begin{itemize}
  \item Let $A$ = set of 8-bit strings that start with a 1.
  \item Let $B$ = set of 8-bit strings that end with 00.
  \item We want $|A \cup B| = |A| + |B| - |A \cap B|$.
\end{itemize}

\textbf{Step 1: Count $|A|$.}  
If the first bit is fixed as 1, there are $7$ free bits left:  
\[
|A| = 2^7 = 128.
\]

\textbf{Step 2: Count $|B|$.}  
If the last two bits are fixed as 00, there are $6$ free bits:  
\[
|B| = 2^6 = 64.
\]

\textbf{Step 3: Count the overlap $|A \cap B|$.}  
If a string both starts with 1 \emph{and} ends with 00, that leaves $5$ free bits in the middle:  
\[
|A \cap B| = 2^5 = 32.
\]

\textbf{Step 4: Apply the Subtraction Rule.}  
\[
|A \cup B| = 128 + 64 - 32 = 160.
\]
\end{example}

\begin{solution}
So, there are $\boxed{160}$ bit strings of length 8 that start with 1 or end with 00.

\medskip
\noindent
\textbf{Key Idea:}  
The subtraction corrects for overlap — things counted twice. Without it, you’d have $128 + 64 = 192$, which overcounts by the 32 strings that fit both patterns.
\end{solution}

---

\section*{Practice Problems — Subtraction Rule in Action}

\textbf{1. Easier Problem:}  
How many 5-bit strings start with 0 or end with 1?

\newpage
\textbf{Solution:}
\[
|A| = 2^4 = 16, \quad |B| = 2^4 = 16, \quad |A \cap B| = 2^3 = 8.
\]
\[
|A \cup B| = 16 + 16 - 8 = 24.
\]
\textbf{Answer:} 24 such bit strings exist.

---

\textbf{2. Stretch Problem:}  
How many 10-bit strings start with “11” or end with “000”?

\newpage
\textbf{Solution:}  
\begin{itemize}
  \item $|A|$: Starts with 11 $\Rightarrow$ $2^8 = 256$.
  \item $|B|$: Ends with 000 $\Rightarrow$ $2^7 = 128$.
  \item $|A \cap B|$: Starts with 11 and ends with 000 $\Rightarrow$ $2^5 = 32$.
\end{itemize}

\[
|A \cup B| = 256 + 128 - 32 = 352.
\]
\textbf{Answer:} 352 such bit strings exist.

---

\section{Python Demonstration}

\begin{lstlisting}[language=Python, caption={Demonstrating the Subtraction Rule with Bit Strings}]
"""
COMSC 2043 - Subtraction Rule Examples
Author: Jeremy Evert
"""

def count_subtraction_rule(n1, n2, overlap):
    """Apply the subtraction rule."""
    return n1 + n2 - overlap

# Example 1: 8-bit strings starting with 1 or ending with 00
A = 2**7   # starts with 1
B = 2**6   # ends with 00
overlap = 2**5  # both conditions
print("8-bit strings:", count_subtraction_rule(A, B, overlap))

# Practice check: 10-bit version
A, B, overlap = 2**8, 2**7, 2**5
print("10-bit strings:", count_subtraction_rule(A, B, overlap))
\end{lstlisting}

---

\section*{Reflection}

The subtraction rule protects us from our own enthusiasm — we tend to count everything, including repeats!  
Mathematically, it’s the bridge from simple counting to true combinatorial reasoning.  

It also underlies many powerful ideas in computer science:  
\begin{itemize}
  \item detecting overlaps in sets (databases and queries),
  \item correcting for double-counted results in logic and probability,
  \item and later — the \textbf{Inclusion–Exclusion Principle}, which generalizes this very rule.
\end{itemize}

The moral?  
When two paths overlap, you must subtract the crossroads.
