\chapter{Section 6.1.4 — The Subtraction Rule (Avoiding Double Counting)}

\section{Why This Section Matters}
By now, we’ve counted outcomes using multiplication (the \textbf{Product Rule}) and addition (the \textbf{Sum Rule}).  
But what happens when categories overlap—when some items fit both descriptions?  

The \textbf{Subtraction Rule} fixes that problem.  
It corrects for double-counting and lays the foundation for the more powerful Inclusion–Exclusion Principle later.

---

\section{Definition: The Subtraction Rule}

\begin{definition}[Subtraction Rule]
If one task can be done in $n_1$ ways or another in $n_2$ ways, but some outcomes are counted twice because they belong to both sets (there are $n_{1,2}$ overlaps),  
then the true total is:
\[
n_1 + n_2 - n_{1,2}.
\]
\end{definition}

\noindent
Subtracting the intersection ensures every unique outcome is counted once and only once.

---

\section{Example 1 — Bit Strings with Overlap (Rosen Example 18)}

\begin{example}[8-bit Strings Starting with 1 or Ending with 00]
How many bit strings of length 8 start with 1 \emph{or} end with 00?

\textbf{Setup:}
\[
A = \text{strings starting with 1}, \qquad B = \text{strings ending with 00}.
\]
We want:
\[
|A \cup B| = |A| + |B| - |A \cap B|.
\]

\textbf{Count each piece.}
\[
|A| = 2^7 = 128, \quad |B| = 2^6 = 64, \quad |A \cap B| = 2^5 = 32.
\]
\[
|A \cup B| = 128 + 64 - 32 = 160.
\]
\end{example}

\begin{solution}
There are $\boxed{160}$ valid 8-bit strings.  
The subtraction removes the 32 strings that satisfy both conditions.  
Without it, $128 + 64 = 192$ would double-count those overlaps.
\end{solution}

---

\section{Example 2 — Library Database Overlap}
\begin{example}[Students Enrolled in Math or Computer Science]
At SWOSU, 130 students take a math course, 85 take a computer science course, and 42 take both.  
How many students take at least one of the two?

\textbf{Apply the subtraction rule:}
\[
130 + 85 - 42 = 173.
\]
\end{example}

\begin{solution}
$\boxed{173}$ unique students are represented.  
This same logic powers database joins and query optimization: we must subtract the intersection to avoid duplication.
\end{solution}

---

\section*{Practice Problems — Subtraction in Action}

\textbf{1. Easier:}  
How many 5-bit strings start with 0 or end with 1?

\newpage
\textbf{Solution:}
\[
|A| = 2^4 = 16, \quad |B| = 2^4 = 16, \quad |A \cap B| = 2^3 = 8.
\]
\[
|A \cup B| = 16 + 16 - 8 = 24.
\]
\textbf{Answer:} 24 such bit strings.

---

\textbf{2. Moderate:}  
How many 10-bit strings start with “11” or end with “000”?

\newpage
\textbf{Solution:}
\[
|A| = 2^8 = 256, \quad |B| = 2^7 = 128, \quad |A \cap B| = 2^5 = 32.
\]
\[
|A \cup B| = 256 + 128 - 32 = 352.
\]
\textbf{Answer:} 352 strings meet the condition.

---

\textbf{3. Stretch:}  
A database lists 420 students in biology, 380 in chemistry, and 160 in both.  
How many distinct students are in at least one science course?

\newpage
\textbf{Solution:}
\[
420 + 380 - 160 = 640.
\]
\textbf{Answer:} 640 unique students.

---

\section{Python Demonstration}

\begin{lstlisting}[language=Python, caption={Demonstrating the Subtraction Rule in Python}]
"""
COMSC 2043 - Subtraction Rule Demonstrations
Author: Jeremy Evert
"""

def subtraction_rule(n1, n2, overlap):
    """Return n1 + n2 - overlap."""
    return n1 + n2 - overlap

# Example 1: 8-bit strings
A, B, overlap = 2**7, 2**6, 2**5
print("8-bit strings:", subtraction_rule(A, B, overlap))

# Example 2: Students in Math or CS
math, cs, both = 130, 85, 42
print("Unique students:", subtraction_rule(math, cs, both))
\end{lstlisting}

---

\section*{Reflection}
The subtraction rule guards against double-vision in counting.  
Whenever two sets overlap, the shared region must be removed once to restore accuracy.  

From bit strings to course enrollments, the rule ensures each unique element is tallied only once—  
a quiet act of mathematical honesty that underpins databases, logic, and probability alike.

