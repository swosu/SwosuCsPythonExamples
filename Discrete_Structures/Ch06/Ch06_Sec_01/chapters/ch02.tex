\chapter{Section 6.1 — The Basics of Counting}

\section{Introduction}

Suppose that a password on a computer system consists of six, seven, or eight characters. 
Each of these characters must be a digit or a letter of the alphabet, and each password must contain at least one digit. 
How many such passwords are there?

Questions like this form the beating heart of discrete mathematics:  
\emph{How many ways can something happen?}  
Counting allows us to measure complexity, probability, and possibility.  
In computer science, every algorithm that loops, branches, or searches is secretly counting something.

\section{Basic Counting Principles}

Two fundamental rules guide our reasoning:

\begin{description}
  \item[\textbf{The Sum Rule}]  
  If a task can be done in $n_1$ ways \emph{or} $n_2$ ways (but not both), then it can be done in $n_1 + n_2$ ways.

  \item[\textbf{The Product Rule}]  
  If a task can be broken into two independent subtasks—first done in $n_1$ ways, then in $n_2$ ways—then the whole procedure can be done in $n_1 \times n_2$ ways.
\end{description}

\section{Worked Examples}

\begin{example}[Assigning Offices]
A new company with two employees, Sanchez and Patel, rents a floor with 12 offices.  
How many ways are there to assign different offices to these two employees?

\textbf{Solution.}  
12 choices for Sanchez, then 11 remaining for Patel.  
By the product rule: $12 \times 11 = 132$ possibilities.
\end{example}

\begin{example}[Labeling Auditorium Chairs]
Each seat is labeled with one uppercase English letter followed by a positive integer not exceeding 100.  
How many unique chair labels exist?

\textbf{Solution.}  
$26$ possible letters $\times$ $100$ integers $= 2600$ labels.
\end{example}

\begin{example}[Counting Ports in a Data Center]
There are 32 computers, each with 24 ports.  
How many total ports exist?

\textbf{Solution.}  
$32 \times 24 = 768$ ports.
\end{example}

\begin{example}[Challenging A: License Plate Combinations]
A region issues license plates with 3 uppercase letters followed by 3 digits (e.g., ABC123).  
How many distinct plates are possible if repetition is allowed?

\textbf{Solution.}  
$26^3 \times 10^3 = 175{,}760{,}000$ possible plates.

If letters and digits cannot repeat, the count becomes  
$26 \times 25 \times 24 \times 10 \times 9 \times 8 = 112{,}320{,}000$.
\end{example}

\begin{example}[Challenging B: Passwords with a Digit Requirement]
A password must be 6, 7, or 8 characters long, each either a letter or digit, 
and must contain at least one digit.  
How many possible passwords exist?

\textbf{Solution.}  
Let $A$ be all possible strings of the given length (letters + digits),  
and $B$ be those with only letters.  

Then valid passwords $= A - B$.

\begin{align*}
|A_6| &= 36^6, & |B_6| &= 26^6 \\
|A_7| &= 36^7, & |B_7| &= 26^7 \\
|A_8| &= 36^8, & |B_8| &= 26^8 \\
\text{Total} &= (36^6 - 26^6) + (36^7 - 26^7) + (36^8 - 26^8)
\end{align*}
\end{example}

\section{Python Demonstrations}

Below is Python code that mirrors the reasoning in these examples.  
Each section prints both the symbolic reasoning and the computed value.  
Students are encouraged to modify the numbers and observe how the results change.

\begin{lstlisting}[language=Python, caption={Demonstrating the Product and Sum Rules in Python}]
"""
COMSC 2043 - Counting Demonstrations
Author: Jeremy Evert
This script demonstrates the basic principles of counting using Python.
"""

from math import comb, perm

def example_offices():
    n1, n2 = 12, 11
    total = n1 * n2
    print(f"Example 1: Assigning offices\n12 * 11 = {total}\n")

def example_chairs():
    total = 26 * 100
    print(f"Example 2: Chair labeling\n26 * 100 = {total}\n")

def example_ports():
    total = 32 * 24
    print(f"Example 3: Data center ports\n32 * 24 = {total}\n")

def example_license_plates():
    allow_repeat = 26**3 * 10**3
    no_repeat = (26*25*24) * (10*9*8)
    print("Example 4: License plates")
    print(f"  With repetition: {allow_repeat:,}")
    print(f"  Without repetition: {no_repeat:,}\n")

def example_passwords():
    # Helper function: A^n - B^n
    def count_valid(length):
        return 36**length - 26**length

    total = sum(count_valid(n) for n in (6, 7, 8))
    print("Example 5: Passwords with at least one digit")
    print(f"  Total valid passwords: {total:,}\n")

def main():
    print("=== Counting Demonstrations ===\n")
    example_offices()
    example_chairs()
    example_ports()
    example_license_plates()
    example_passwords()
    print("Done!")

if __name__ == "__main__":
    main()
\end{lstlisting}

\section*{Discussion and Reflection}

Each of these problems could be solved by intuition, but writing code forces precision.  
Python doesn’t “believe” in the product rule—it performs it.  
This helps students see that the abstract logic of counting translates directly into computation.

Encourage students to:
\begin{itemize}
  \item Change numbers and verify the rules still hold.
  \item Add print statements to trace intermediate steps.
  \item Reflect on where independence between tasks exists—and where it doesn’t.
\end{itemize}

\section*{Next Steps}

In the next section we will apply these rules to cases where choices overlap or restrict one another—  
leading to the powerful and deceptively simple \textbf{Pigeonhole Principle}.
