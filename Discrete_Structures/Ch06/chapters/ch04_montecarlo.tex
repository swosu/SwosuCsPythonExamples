\chapter{Roll the Dice, Check the Math}

\section{Story Hook: Can We Trust Our Formulas?}
% - Scenario:
%   * We’ve counted some combinatorial objects and claimed probabilities.
%   * Now we want to see if reality (simulated) matches the theory.
% - Simple example:
%   * Probability of exactly 3 heads in 5 flips.
%   * Probability of rolling at least one “6” in 4 dice.

\section{From Counting to Probability}
% - Recall:
%   * Probability = (# favorable outcomes) / (# total outcomes),
%     when all outcomes are equally likely.
% - Connect shapes:
%   * binomial coefficients (coin flips),
%   * combinations (card hands),
%   * permutations (order-sensitive events),
%   * stars-and-bars (distributions of outcomes).

\section{Python Lab: Monte Carlo vs Exact}
% - Script sketch (monte_dice.py or monte_coins.py):
%   * Ask students to:
%       1. Compute an exact probability using counting + algebra.
%       2. Simulate the experiment N times using random or numpy.
%       3. Track how the estimated probability changes as N grows.
%   * Examples:
%       - Dice patterns (like “exactly two sixes”, “at least one six”).
%       - Coin flips, specified number of heads.
%       - Random usernames matching a simple pattern.

\section{Interpreting the Results}
% - Discussion prompts:
%   * Why do simulated probabilities bounce around for small N?
%   * How do we know when N is “large enough” for a decent approximation?
%   * When is simulation easier than exact counting, and when is it the reverse?

\section*{Podcast: Episode 4 -- The Dice Don’t Lie (Much)}
% - Recap script notes:
%   * Characters argue whether math or “luck” is right.
%   * They run a bunch of simulated experiments.
%   * They discover the law of large numbers in a friendly way.
%   * Tease the final universe-design capstone.

