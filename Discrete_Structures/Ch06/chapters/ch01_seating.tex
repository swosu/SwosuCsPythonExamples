\chapter{Seating the Party at the Round Table}

\section{Story Hook: Dinner Disaster}

You are in charge of a big group dinner.

There are six friends and six chairs in a row. Everyone has Opinions:
\begin{itemize}
  \item Amina \emph{must} sit on an end.
  \item Jake refuses to sit next to Zahra.
  \item Lin and Zahra are best friends and want to sit together.
\end{itemize}

The host asks you a simple question:

\begin{quote}
  ``How many different seating charts are possible?''
\end{quote}

At first, this seems like a small thing. But as the number of people grows, the
number of possible seatings explodes. By the end of this chapter you will know:
\begin{itemize}
  \item how to count all possible seatings when everyone is distinct;
  \item how to handle simple constraints (must sit together / apart / on an end);
  \item how to use Python to \emph{check} your counting on small examples.
\end{itemize}


\section{Warm-Up: Listing All Seatings for Three Friends}

Let us start very small. Suppose we only have three people:
Amina (A), Lin (L), and Jake (J), and three chairs in a row.

\subsection*{Try it yourself}

\begin{enumerate}
  \item Draw three boxes to represent the chairs.
  \item List every possible way to place A, L, and J into those chairs.
\end{enumerate}

If you do this carefully, you should find these six arrangements:
\[
  \text{ALJ},\ \text{AJL},\ \text{LAJ},\ \text{LJA},\ \text{JAL},\ \text{JLA}.
\]

There are $6$ different seatings. This is already a lot for just three people!
Soon we will see how this connects to the factorial function and why three
people give six seatings, four people give twenty-four seatings, and so on.


\section{Core Ideas: Factorials and Linear Permutations}

\subsection{The factorial function}

The \emph{factorial} of a positive integer $n$ is written $n!$ and defined as
\[
  n! = n \cdot (n-1) \cdot (n-2) \cdots 3 \cdot 2 \cdot 1,
\]
with the special convention that $0! = 1$.

Some small values:
\[
  1! = 1,\quad
  2! = 2,\quad
  3! = 6,\quad
  4! = 24,\quad
  5! = 120.
\]

\subsection{Factorials as seatings}

We can interpret $n!$ as the number of ways to seat $n$ distinct people in $n$
distinct chairs in a row.

Why?

\begin{itemize}
  \item For the first chair, we can choose any of the $n$ people.
  \item For the second chair, we have $n-1$ people left.
  \item For the third chair, we have $n-2$ people left.
  \item \dots
  \item For the last chair, we have $1$ person left.
\end{itemize}

By the \emph{product principle},
the total number of seatings is
\[
  n \cdot (n-1) \cdot (n-2) \cdots 2 \cdot 1 = n!.
\]

This explains why:
\begin{itemize}
  \item $3! = 6$ seatings for A, L, J (which we listed),
  \item $4! = 24$ seatings for four distinct people,
  \item $6! = 720$ seatings for six distinct people.
\end{itemize}

Already at six people, $720$ is a lot of possible seating charts. No human
host is going to check all of those by hand.


\subsection{Permutations of a set}

If we have a set $S$ of $n$ distinct elements, any ordering of the elements of
$S$ in a row is called a \emph{permutation} of $S$. The discussion above says:

\begin{quote}
  The number of permutations of a set of size $n$ is $n!$.
\end{quote}

This is one of the most basic counting facts in discrete mathematics: whenever
you are arranging $n$ distinct things in a line and every order is allowed,
the answer is $n!$.


\section{Restricted Seatings: Adding Drama}

Real dinners have rules. Let us see how a few common restrictions change the
counting. Throughout this section, think of six seats in a row and six friends:
Amina (A), Lin (L), Zahra (Z), Jake (J), plus two more friends, M and N.

\subsection{Example 1: One person in a fixed seat}

Suppose Amina \emph{must} sit in the first chair.

\begin{itemize}
  \item We no longer have a choice for the first chair: it is always A.
  \item For the remaining five chairs, we can seat the other five friends in any order.
\end{itemize}

So the total number of seatings is simply
\[
  5!
\]
because we are only permuting the remaining five people.

In general, if one person has a fixed location and the others are free, the
number of seatings is $(n-1)!$.

\subsection{Example 2: Two people sit together as a block}

Now suppose Lin and Zahra insist on sitting next to each other.

A standard trick is to treat the pair (LZ) as a single ``block.''

\begin{itemize}
  \item First, create a new list of ``objects'': the block (LZ) plus the
        other four people A, J, M, N. That gives 5 objects to arrange.
  \item The number of ways to arrange these 5 objects is $5!$.
  \item Inside the block, Lin and Zahra can sit in two orders: LZ or ZL.
\end{itemize}

By the product principle:
\[
  \text{number of seatings with L,Z together} = 5! \cdot 2.
\]

This idea appears over and over: when some group must stay together, we often
treat that group as a single unit, count arrangements of the units, then
multiply by the internal arrangements of each unit.

\subsection{Example 3: Two people must not sit together}

Now suppose Jake and Zahra \emph{refuse} to sit next to each other. How many
seatings avoid J and Z being adjacent?

A common strategy is:
\begin{enumerate}
  \item Count all possible seatings with no restriction: $6!$.
  \item Count how many seatings have J and Z together (as a block).
  \item Subtract: (all seatings) $-$ (bad seatings).
\end{enumerate}

We already know from the previous example that the number of seatings with J
and Z together is $5! \cdot 2$ (treat JZ as a block, and swap inside the
block). Therefore:
\[
  \text{seatings with J and Z not adjacent} = 6! - 5! \cdot 2.
\]

This pattern---\emph{count everything, then subtract the bad cases}---is a very
powerful idea in counting. We will see it again later in the unit with more
complicated sets of ``bad'' outcomes.


\section{Python Lab: Brute-Forcing the Seating}

Mathematics gives us formulas, but Python lets us \emph{test} those formulas
on small examples and explore different constraints without doing all the
bookkeeping by hand.

In this lab you will write or modify a script (for example,
\texttt{round\_table.py}) with the following goals.

\subsection*{Step 1: Generate all seatings}

\begin{enumerate}
  \item Create a list of names, such as
        \texttt{["Amina", "Lin", "Zahra", "Jake"]}.
  \item Use \texttt{itertools.permutations} to generate all possible orderings
        of the list.
  \item Confirm that the number of permutations matches $n!$ for your $n$.
\end{enumerate}

\noindent
A short code sketch might look like this:

\begin{lstlisting}
import itertools
import math

friends = ["A", "L", "Z", "J"]
perms = list(itertools.permutations(friends))

print("Number of permutations:", len(perms))
print("Should be:", math.factorial(len(friends)))
\end{lstlisting}

\subsection*{Step 2: Count seatings with a fixed person at one end}

Modify your code so that it counts only those permutations where Amina sits in
the first chair.

\begin{itemize}
  \item Compare the Python count to the formula $(n-1)!$.
  \item Try both ends (first or last) and see what happens.
\end{itemize}

\subsection*{Step 3: Count seatings where two friends sit together}

Choose two friends (for example, Lin and Zahra) and:

\begin{itemize}
  \item count how many permutations have them sitting in adjacent seats;
  \item compare to the ``block'' formula $5! \cdot 2$ (for $n=6$);
  \item experiment with different values of $n$ and positions.
\end{itemize}

\subsection*{Step 4: Count seatings where two friends do \emph{not} sit together}

Finally, let Python estimate how many permutations avoid J and Z being
neighbors.

\begin{itemize}
  \item Compute the number directly by checking every permutation.
  \item Compare with the subtraction formula $6! - 5! \cdot 2$.
  \item Try other pairs of friends and see if the pattern holds.
\end{itemize}

The goal is not to write huge programs, but to use short scripts as mirrors
that reflect the combinatorial structure we discovered on paper.


\section{Practice and Extensions}

Here are some practice problems you might see after reading this chapter.

\subsection*{Basic practice}

\begin{enumerate}
  \item Explain in words what $5!$ means in the context of seating five
        distinct friends in five chairs.
  \item Compute $6!$ and describe a real-life scenario where $6!$ is the
        correct answer.
\end{enumerate}

\subsection*{Medium spice}

\begin{enumerate}
  \item Six friends are to be seated in a row. Two of them insist on sitting
        together. How many seatings are possible?
  \item Six friends are to be seated in a row. Two of them refuse to sit next
        to each other. How many seatings are possible?
\end{enumerate}

\subsection*{Extra spicy (optional)}

\begin{enumerate}
  \item Circular table variation: six friends sit around a round table. Two
        seatings that can be rotated into each other are considered the same.
        How many distinct seatings are possible?
  \item Modify your Python script to treat rotations as identical and test
        your answer on small values of $n$.
\end{enumerate}


\section*{Podcast: Episode 1 --- The Seating Chart}

At the end of this chapter, there is a short podcast episode. A possible
outline for the episode:

\begin{itemize}
  \item The characters are trying to organize a group dinner.
  \item They first try to list out seatings for a small group and quickly get
        overwhelmed as the group grows.
  \item Someone introduces the idea of $n!$ and the ``product of choices''
        story: first seat, second seat, and so on.
  \item They play with the idea of treating two friends as a ``block'' and
        of subtracting the bad arrangements.
  \item They mention that Python helped them check their answers by
        generating all seatings for small examples.
  \item They end with a teaser:
        \begin{quote}
          ``If we can count seatings, can we also count usernames?  
          Next time: gamer tags, license plates, and the size of the internet.''
        \end{quote}
\end{itemize}

