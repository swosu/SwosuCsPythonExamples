\chapter{Tiny Tactics Arena: A Fully Counted Universe}

\section{Story Hook: Balancing a Mini-Game}

You have been hired as the \emph{official mathematician} for a tiny
tactics game called \emph{Tiny Tactics Arena}.

The game designers are worried:
\begin{itemize}
  \item ``Do we have enough variety, or will players see the same teams over and over?''
  \item ``Are some builds secretly way more common than others?''
  \item ``If we tweak a rule, how much does it shrink or grow the universe of possibilities?''
\end{itemize}

Your mission in this chapter is to \emph{fully count} a small,
self-contained game universe:
\begin{itemize}
  \item who the heroes are,
  \item how they act (turn order),
  \item what artifacts they bring,
  \item how they distribute skill points,
  \item and a few simple probabilities.
\end{itemize}

Along the way, you will use:
\begin{itemize}
  \item permutations (order matters),
  \item combinations (order does not),
  \item stars-and-bars (distributing identical points),
  \item and Python to sanity-check your math.
\end{itemize}


\section{The Rules of Tiny Tactics Arena}

We begin with a deliberately small rule set.

\subsection*{Basic ingredients}

\begin{itemize}
  \item There are exactly 3 hero classes:
        \[
          \text{Warrior (W)},\quad
          \text{Mage (M)},\quad
          \text{Rogue (R)}.
        \]
  \item Each team brings \emph{one hero of each class}.
        So every team has three heroes:
        W, M, and R.
  \item There are 4 artifacts available:
        \[
          \text{Sword (S)},\quad
          \text{Staff (T)},\quad
          \text{Dagger (D)},\quad
          \text{Shield (H)}.
        \]
        Artifacts are all distinct.
  \item At the start of a match, the team chooses \emph{exactly 2}
        different artifacts to bring.
        For now, we do not care who holds which artifact.
  \item Each hero has 5 skill points to distribute among
        three stats:
        \(\mathrm{STR}\) (Strength),
        \(\mathrm{DEX}\) (Dexterity),
        and \(\mathrm{INT}\) (Intellect).
        Skill points are indistinguishable: they are just little pips.
\end{itemize}

\subsection*{Game state we will count}

To keep things focused, define a \emph{configuration} to consist of:

\begin{enumerate}
  \item An \emph{initiative order} of the three heroes
        (who goes first, second, third).
  \item A \emph{set of 2 artifacts} chosen from the 4 available.
  \item A \emph{skill build} for each hero
        (how each hero distributes 5 points among STR, DEX, INT).
\end{enumerate}

In this chapter we will:

\begin{itemize}
  \item Count each piece separately,
  \item Assemble them into a total count for all configurations,
  \item Write Python code to confirm that our counting matches
        reality for the tiny universe.
\end{itemize}


\section{Permutations: Initiative Order}

The three heroes W, M, and R will act in some order each round.

\subsection*{Counting the orders}

An \emph{initiative order} is simply a permutation of the set
\(\{W, M, R\}\).

\begin{itemize}
  \item There are 3 choices for who goes first.
  \item Then 2 choices remain for who goes second.
  \item Then 1 choice remains for who goes third.
\end{itemize}

By the product rule:
\[
  3 \cdot 2 \cdot 1 = 3! = 6
\]
possible initiative orders.

Example orders:
\((W, M, R)\),
\((M, W, R)\),
\((R, M, W)\), and so on.

Here, \emph{order matters}:
\((W, M, R)\) is not the same as \((M, W, R)\), because different heroes
act first.


\section{Combinations: Choosing Artifacts}

The team must choose exactly 2 distinct artifacts from the 4 available:
\(\{S, T, D, H\}\).

\subsection*{Order does not matter}

If we only care which two artifacts are brought, and we do not care
which one we list first (or who holds them), this is a combinations
problem.

We are choosing 2 objects out of 4, without repetition, and without order:

\[
  \binom{4}{2}
  = \frac{4!}{2! \cdot 2!}
  = 6.
\]

We can even list them all:

\[
  \{S,T\},\ \{S,D\},\ \{S,H\},\ \{T,D\},\ \{T,H\},\ \{D,H\}.
\]

Each of these sets is one possible artifact loadout for the squad.


\section{Stars and Bars: Skill Trees for a Single Hero}

Now consider a single hero, for example the Mage.

\subsection*{Distributing skill points}

The Mage has 5 identical skill points to distribute among three stats:
\(\mathrm{STR}\), \(\mathrm{DEX}\), \(\mathrm{INT}\).

Let
\[
  x_{\mathrm{STR}},\quad
  x_{\mathrm{DEX}},\quad
  x_{\mathrm{INT}}
\]
be the number of points in each stat.

We are looking at integer solutions to
\[
  x_{\mathrm{STR}} + x_{\mathrm{DEX}} + x_{\mathrm{INT}} = 5,
\]
with
\[
  x_{\mathrm{STR}}, x_{\mathrm{DEX}}, x_{\mathrm{INT}} \ge 0.
\]

This is a classic \emph{stars-and-bars} situation: we place
5 stars into 3 bins.

The number of solutions is
\[
  \binom{5 + 3 - 1}{3 - 1}
  = \binom{7}{2}
  = 21.
\]

Each solution corresponds to one skill build.  
For example:
\begin{itemize}
  \item \((5,0,0)\) means full Strength,
  \item \((3,2,0)\) means 3 STR, 2 DEX, 0 INT,
  \item \((1,1,3)\) means 1 STR, 1 DEX, 3 INT.
\end{itemize}

\subsection*{All three heroes}

In our universe, \emph{each} hero independently distributes 5 points
among the 3 stats.

So, if each hero has 21 possible skill builds, and order of heroes is
fixed as (W, M, R) \emph{for the purpose of counting builds}, then:

\[
  \text{number of skill configurations for all 3 heroes}
  = 21 \cdot 21 \cdot 21
  = 21^3.
\]

We will leave the number as \(21^3\) for now,
and multiply everything together in the next section.


\section{Putting It All Together}

We are ready to count full configurations:
\begin{itemize}
  \item Initiative order of the three heroes,
  \item Set of 2 artifacts,
  \item Skill builds for all three heroes.
\end{itemize}

\subsection*{Step-by-step product}

From previous sections:

\begin{itemize}
  \item Initiative orders: \(3! = 6\).
  \item Artifact sets: \(\binom{4}{2} = 6\).
  \item Hero skill configurations:
        each hero has 21 possibilities, so \(21^3\) for the trio.
\end{itemize}

Assuming all of these choices are made independently, the product rule
tells us the total number of configurations is:

\[
  \underbrace{3!}_{\text{initiative}} \times
  \underbrace{\binom{4}{2}}_{\text{artifacts}} \times
  \underbrace{21^3}_{\text{skills}}
  = 6 \times 6 \times 21^3.
\]

If you really want the raw integer:

\[
  21^2 = 441, \quad
  21^3 = 21 \cdot 441 = 9261,
\]
so
\[
  6 \times 6 \times 9261
  = 36 \times 9261.
\]

One way:
\[
  9261 \cdot 30 = 277{,}830,\quad
  9261 \cdot 6 = 55{,}566,
\]
so
\[
  277{,}830 + 55{,}566 = 333{,}396.
\]

Thus our tiny universe already has
\[
  333{,}396
\]
different configurations!

Even with only three heroes, four artifacts, and 5 skill points each,
we are way past the point where ``just listing them all'' is pleasant.


\section{Python Lab: Brute-Forcing the Arena}

Now we will write a short Python script to:

\begin{enumerate}
  \item Generate all initiative orders,
  \item Generate all artifact sets,
  \item Generate all skill builds for a single hero,
  \item Combine them and count how many full configurations we get,
  \item Check that the count matches our formula
        \(3! \cdot \binom{4}{2} \cdot 21^3\).
\end{enumerate}

\subsection*{Generating skill builds}

We can generate the \((x_{\mathrm{STR}}, x_{\mathrm{DEX}}, x_{\mathrm{INT}})\)
triples for a single hero using simple loops.

\begin{lstlisting}[language=Python, caption={Generating all skill builds for one hero}]
def skill_builds(points=5):
    """Return all (STR, DEX, INT) triples that sum to `points`."""
    builds = []
    for str_pts in range(points + 1):
        for dex_pts in range(points + 1 - str_pts):
            int_pts = points - str_pts - dex_pts
            builds.append((str_pts, dex_pts, int_pts))
    return builds
\end{lstlisting}

You can quickly check:

\begin{lstlisting}[language=Python]
>>> len(skill_builds(5))
21
\end{lstlisting}

which matches our stars-and-bars count.


\subsection*{Putting the whole universe together}

Here is a complete script for Tiny Tactics Arena.

\begin{lstlisting}[language=Python, caption={Brute-force Tiny Tactics Arena}]
import itertools
from math import comb, factorial

CLASSES = ["W", "M", "R"]
ARTIFACTS = ["S", "T", "D", "H"]

def skill_builds(points=5):
    builds = []
    for str_pts in range(points + 1):
        for dex_pts in range(points + 1 - str_pts):
            int_pts = points - str_pts - dex_pts
            builds.append((str_pts, dex_pts, int_pts))
    return builds

def main():
    # 1. Initiative orders (permutations of W, M, R)
    initiatives = list(itertools.permutations(CLASSES))
    print("Number of initiative orders:", len(initiatives))

    # 2. Artifact sets: choose 2 out of 4 (combinations)
    artifact_sets = list(itertools.combinations(ARTIFACTS, 2))
    print("Number of artifact sets:", len(artifact_sets))

    # 3. Skill builds for a single hero
    hero_builds = skill_builds(points=5)
    print("Number of builds for one hero:", len(hero_builds))

    # 4. Skill configurations for all three heroes
    triple_builds = list(itertools.product(hero_builds, repeat=3))
    print("Number of builds for three heroes:", len(triple_builds))

    # 5. Combine everything into full configurations
    count = 0
    for init in initiatives:
        for artifacts in artifact_sets:
            for builds in triple_builds:
                # builds is a triple:
                # (build_for_W, build_for_M, build_for_R)
                count += 1

    print("Total configurations (brute force):", count)

    # 6. Compare with the formula:
    formula = factorial(3) * comb(4, 2) * (len(hero_builds) ** 3)
    print("Total from formula:", formula)


    if count == formula:
        print("Counts match! Our combinatorics is consistent.")
    else:
        print("Mismatch -- something is off somewhere.")


if __name__ == "__main__":
    main()
\end{lstlisting}

This script is intentionally straightforward (and not optimized).
On a modern machine, it should happily crunch through the
\(333{,}396\) configurations.


\section{Probabilities in the Arena}

Once we can count configurations, we can also talk about probabilities.

To keep things simple, suppose:
\begin{itemize}
  \item Each team configuration is chosen uniformly at random
        from the \(333{,}396\) possibilities.
\end{itemize}

We can ask questions like:

\begin{itemize}
  \item What is the probability that the Mage acts first?
  \item What is the probability that the team brings a Shield?
  \item What is the probability that the Rogue is ``glass cannon''
        (0 points in STR and 0 points in DEX)?
\end{itemize}

We will do just one example here.

\subsection*{Example: probability the Mage acts first}

How many configurations have the Mage acting first?

\begin{itemize}
  \item Fix Mage first.
        Then the other two heroes (W and R) can be arranged in 2 ways:
        WR or RW.
        So there are 2 initiative orders with Mage first.
  \item The rest of the choices (artifact sets, skill builds) do not
        depend on this detail.
        For each initiative order, there are
        \(\binom{4}{2} \cdot 21^3\) ways to choose artifacts and skills.
\end{itemize}

So the number of configurations with Mage first is
\[
  2 \times \binom{4}{2} \times 21^3.
\]

The probability is
\[
  \frac{2 \times \binom{4}{2} \times 21^3}{
         3! \times \binom{4}{2} \times 21^3}
  = \frac{2}{6}
  = \frac{1}{3}.
\]

This matches our intuition: all three heroes are symmetric in the rules,
so each should be first one-third of the time.


\section{Design Variations and Chaos Modes}

Once the basic universe is built, it is easy (and fun) to mutate it.

Here are some design prompts:

\begin{itemize}
  \item \textbf{Add more heroes:}
        What happens if you add a fourth class (Healer)?
        How does that change the initiative permutations and
        total configuration count?
  \item \textbf{Artifact limits:}
        What if the team can bring 3 artifacts instead of 2?
        Or what if certain classes are not allowed to use some artifacts?
  \item \textbf{Skill caps:}
        You could require that no stat may exceed 3 points.
        How many skill builds are left for a hero?
        (This becomes a stars-and-bars problem with upper bounds.)
  \item \textbf{Random generation:}
        Write Python code that generates random valid configurations,
        and then estimates the probability of some condition
        (for example, ``Mage acts first and has at least 3 INT'').
        Compare the simulated probability to the exact one.
\end{itemize}

Your challenge: take this tiny tactics universe, tweak one rule, and
then \emph{track exactly how the counts change}.
That is exactly what game designers, security engineers, and combinatorics
fans do in the real world.


\section*{Podcast: Episode 6 -- Live from the Arena}

% - Recap script notes:
%   * The cast walks through the Tiny Tactics Arena universe:
%     heroes, artifacts, skill trees.
%   * One character complains: "This is too many possibilities!
%     How is anyone supposed to balance this?"
%   * Another reveals the secret weapon: counting + Python.
%   * They briefly simulate random teams, confirm the counts,
%     and joke about banning the all-INT Mage build.
%   * Final button: "Any time a game, system, or schedule feels
%     impossibly huge, that's your cue that combinatorics is nearby."

