\chapter{Design Your Own Universe}

\section{Project Brief}
% - Students design a small universe of possibilities:
%   * Card game, RPG, character builder, resource distribution system, etc.
% - Requirements:
%   * At least one permutation-based situation (order matters).
%   * At least one combination-based situation (order doesn’t).
%   * At least one stars-and-bars style distribution.
%   * At least one probability that can be computed from counts.

\section{Planning the Universe}
% - Prompts:
%   * Who/what exists in your universe? (characters, items, skills, etc.)
%   * What are the choices the player/system makes?
%   * Which of those choices are best modeled by permutations, combinations,
%     or stars-and-bars?
% - Encourage sketches, diagrams, and tables.

\section{Mathematical Analysis}
% - Students:
%   * State each counting problem clearly in symbols.
%   * Show the relevant formula and brief justification.
%   * Compute a few concrete examples.
% - Optional stretch:
%   * Relate two different formulas that give the same count.
%   * Explain why they must agree.

\section{Python Component}
% - Requirements:
%   * A Python script (or small set of scripts) that:
%       - Computes one or more counts using formulas (math.comb, factorial, etc.).
%       - Verifies at least one result by brute force or simulation.
%       - Prints a short, human-readable summary of results.
% - Nice-to-have:
%   * Parameterized universe (change number of players, items, etc.).
%   * Random generator: create random teams, loadouts, or scenarios.

\section{Presentations and Reflection}
% - What students share:
%   * Story summary of their universe.
%   * Key counting challenges and their solutions.
%   * One surprising thing they discovered when coding or simulating.
% - Reflection prompts:
%   * Where did combinatorics make your life easier?
%   * Where did it get messy? How did you manage that complexity?

\section*{Podcast: Episode 5 -- Universe Builders}
% - Recap script notes:
%   * Cast interviews each other about their universes.
%   * Highlight 2–3 student-style universes and the math inside them.
%   * Close the arc: “Counting is how we tame infinite-feeling possibility spaces.”

