\chapter{Design Your Own Universe}

\section{Project Brief}

By this point in the course, we have seen permutations, combinations,
stars-and-bars distributions, and probabilities built from counting.
We have also touched Python scripts that help us explore and verify
our work. In this capstone you will flip the script:

\medskip

\noindent\textbf{Your mission:} design a small ``universe of possibilities'' that
you actually care about, and then analyze it with the tools from this
chapter.

\medskip

Your universe might be

\begin{itemize}
  \item a card game or deck-building game,
  \item an RPG character builder or loadout system,
  \item a resource distribution system (skill points, credits, crafting mats),
  \item or anything else where there are many possible outcomes and choices.
\end{itemize}

To keep the project focused, your universe must include at least:

\begin{itemize}
  \item \textbf{One permutation-based situation} where order matters
        (for example, initiative order, turn order, or a queue).
  \item \textbf{One combination-based situation} where order does not matter
        (such as choosing a team, a hand of cards, or a set of items).
  \item \textbf{One stars-and-bars style distribution} of identical things into
        distinct buckets (for example, skill points into stats, scoops into
        flavors, or coins into piggy banks).
  \item \textbf{At least one probability} that you can compute exactly from
        counts (for example, the probability of drawing a certain hand, or
        getting a certain type of build).
\end{itemize}

Your final deliverable will be a blend of story and math: a short write-up
explaining your universe, a set of clearly stated counting questions, their
solutions, and a small Python component that checks at least one of your
answers by brute force or simulation.


\section{Planning the Universe}

Before you rush into formulas, spend time making your universe fun and
coherent. Use these questions as planning prompts; sketch answers in words
or pictures before turning anything into symbols.

\subsection*{Who and what lives in your universe?}

Describe the basic ingredients:

\begin{itemize}
  \item \textbf{Characters or agents:} players, heroes, monsters, robots,
        students, NPCs, \dots
  \item \textbf{Items or resources:} cards, weapons, spells, skill points,
        coins, tokens, \dots
  \item \textbf{Actions or choices:} drawing a hand, choosing a team,
        allocating points, rolling dice, selecting a username, \dots
\end{itemize}

Write this part so that someone who has \emph{not} taken the class could
still understand the story.

\subsection*{What are the important decisions?}

Think about where combinatorics naturally appears. For each core decision
in your universe, ask:

\begin{itemize}
  \item What is being chosen or arranged?
  \item Are the objects \emph{distinct} (like named players or unique cards)
        or \emph{identical} (like generic coins or hit points)?
  \item Does the \emph{order} of the choice matter?
  \item Are there any constraints? (At least one wizard, no more than
        two legendaries, exactly three scoops, \dots)
\end{itemize}

Jot each decision down in a little table such as:

\medskip
\noindent
\begin{tabular}{|l|l|l|l|}
\hline
Decision & Distinct/Identical? & Order? & Constraints? \\
\hline
\hline
Team of heroes & distinct & order doesn't matter & at least one healer \\
\hline
Turn order & distinct & order matters & all heroes used once \\
\hline
Mana points & identical & N/A & total of 10 points \\
\hline
\end{tabular}
\medskip

\subsection*{Matching decisions to counting tools}

Now connect each decision to a counting method:

\begin{itemize}
  \item \textbf{Permutations} for order-sensitive arrangements of distinct objects.
  \item \textbf{Combinations} for order-insensitive selections of distinct objects.
  \item \textbf{Stars and bars} for distributing identical objects into labeled boxes.
  \item \textbf{Product rule} and \textbf{sum rule} to glue simpler counts together.
\end{itemize}

You do not need to cram \emph{every} topic into \emph{every} decision. It is
better to have a few well-chosen situations that clearly fit one or two
methods than a giant tangle of formulas that no one (including you) enjoys.


\section{Mathematical Analysis}

Once your universe is sketched, it is time to zoom in on the math. Choose a
small number of core questions (three to five is typical) and analyze them in
detail. For each question:

\subsection*{1. State the problem clearly}

Write a short, precise statement that connects story language to math.
For example:

\begin{quote}
  ``In my card game there are 12 different spell cards and 8 different
  item cards. At the start of the game, each player gets a hand of
  5 cards drawn without replacement from the 20-card deck. How many
  different 5-card hands are possible?''
\end{quote}

Then translate this into pure math language, for example:

\begin{quote}
  ``We are choosing a 5-element subset from a 20-element set, so we
  want \(\binom{20}{5}\).''
\end{quote}

\subsection*{2. Choose and justify a method}

For each problem, name the method you are using and justify briefly:

\begin{itemize}
  \item ``We use \emph{permutations} because the order of the players in
        the initiative track matters.''
  \item ``We use \emph{combinations} because only which heroes you pick
        matters, not the order we list them.''
  \item ``We use \emph{stars and bars} because we are distributing identical
        points into labeled stats.''
\end{itemize}

This does not need to be a full formal proof, but it should show that you
know why your formula is appropriate.

\subsection*{3. Compute concrete answers}

Give actual numbers, not just formulas. If the numbers are enormous, it is
fine to use Python or a calculator, but:

\begin{itemize}
  \item show the combinatorial expression (like \(\binom{20}{5}\) or
        \(6! / 2!\)),
  \item then give the numerical value (like \(15{,}504\)).
\end{itemize}

If two different methods give the same count, say so, and briefly explain
why they must agree.

\subsection*{4. Build at least one probability}

Pick at least one event in your universe and compute its probability
using the formula
\[
  P(E) = \frac{\lvert E \rvert}{\lvert \Omega \rvert},
\]
where \(\Omega\) is the set of all equally likely outcomes and \(E\) is the
event you care about.

Examples:

\begin{itemize}
  \item Probability a random hand has exactly two healers,
  \item Probability a randomly generated character has total strength
        at least 10,
  \item Probability a randomly formed team includes a specific character.
\end{itemize}

Whenever possible, say in words what your answer means:
``So there is roughly an 18\% chance that a random hand has at least one
legendary card.''


\section{Python Component}

The Python part of the project lets you check that your formulas match
reality (or at least simulated reality). It does not need to be long or fancy,
but it does need to be \emph{connected} to your universe.

\subsection*{Minimum requirements}

Your Python code should:

\begin{itemize}
  \item \textbf{Compute at least one count or probability using formulas},
        making use of tools like \verb|math.factorial| and \verb|math.comb|
        (or your own helper functions).
  \item \textbf{Verify at least one result} by:
        \begin{itemize}
          \item brute-force listing of all possibilities (for very small cases), or
          \item Monte Carlo simulation (randomly sample many times and
                estimate a probability).
        \end{itemize}
  \item \textbf{Print a short, human-readable summary} such as
        \begin{quote}
          \small
          \verb|Exact probability: 0.1826| \\
          \verb|Simulated probability after 100000 trials: 0.1819|
        \end{quote}
\end{itemize}

You are encouraged to recycle patterns from earlier labs:

\begin{itemize}
  \item Using \verb|itertools.permutations| and \verb|itertools.combinations|
        for exact counts,
  \item Using loops and the \verb|random| module for Monte Carlo experiments,
  \item Structuring your code into small functions so your main script reads
        like a story of what you are doing.
\end{itemize}

\subsection*{Nice-to-have features}

If you have time and interest, you might:

\begin{itemize}
  \item allow the number of players, cards, or points to be parameters,
  \item add a function that generates a random valid team, loadout, or hand,
  \item print a few example outcomes so readers can see what a typical
        random result looks like.
\end{itemize}

Keep in mind that the goal is not to build a video game---it is to
\emph{show off the math} with a bit of computational support.


\section{Presentations and Reflection}

At the end of the project, you will share your universe with others. The
format may be a short written report, a brief in-class talk, a poster, or some
combination, depending on your instructor.

\subsection*{What to share}

Your presentation should include:

\begin{itemize}
  \item \textbf{A story summary} of your universe in plain language.
  \item \textbf{Your key counting challenges} and how you solved them:
        what was being counted, what method you used, and what you found.
  \item \textbf{At least one probability} and a brief explanation of how
        you got it.
  \item \textbf{One surprising or interesting discovery} from coding or
        simulating. For example:
        \begin{itemize}
          \item the simulated probability took a long time to stabilize,
          \item a built-in intuition (``this almost never happens'') turned
                out to be wrong,
          \item two different-looking formulas gave exactly the same number.
        \end{itemize}
\end{itemize}

\subsection*{Reflection prompts}

To wrap up, write a short reflection (a few paragraphs) addressing questions
like:

\begin{itemize}
  \item Where did combinatorics make your life easier?
  \item Where did it get messy? How did you manage that complexity?
  \item If you were to extend your universe, what new counting or probability
        questions would appear?
  \item How might someone in game design, security, data science, or another
        field use similar counting ideas in the real world?
\end{itemize}

Honest reflections (including ``this part was surprisingly hard'') are more
valuable than pretending that every step was smooth and obvious.


\section*{Podcast: Episode 5 -- Universe Builders}

This final episode of the podcast closes the arc of the course.

\medskip

\noindent\textbf{Script notes (suggested):}

\begin{itemize}
  \item The cast interviews each other about their favorite student-style
        universes: card games, RPG worlds, security systems, classroom
        seating chaos, and more.
  \item Each character describes one counting challenge they faced and
        how they solved it, in informal language: ``At first I thought I
        had to list everything by hand, but then I realized it was just
        a combination problem.''
  \item They play with the idea that behind every ``simple'' game rule
        there might be a huge invisible combinatorial space.
  \item They revisit earlier themes:
        \begin{itemize}
          \item factorial explosions from seating at the round table,
          \item username spaces and noisy strings,
          \item sundaes and resource distributions,
          \item dice, probabilities, and simulations.
        \end{itemize}
  \item The episode closes with the big-picture message:
        counting is how we tame spaces of possibilities that feel
        infinite, and how we design systems---games, passwords,
        experiments, universes---on purpose rather than by accident.
\end{itemize}

\medskip

You have now built and analyzed a universe of your own. The same tools
will follow you into algorithms, data structures, probability, statistics,
and any field where we care about what can happen, how often, and why.

