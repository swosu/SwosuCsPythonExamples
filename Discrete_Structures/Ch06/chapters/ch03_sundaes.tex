\chapter{How Many Sundaes Can We Build?}

\section{Story Hook: Infinite Ice Cream Bar}

It is Friday. The math department has made a terrible mistake: they gave you a
\emph{build-your-own-sundae} bar.

There are $k$ flavors of ice cream lined up: vanilla, chocolate, strawberry,
mint, cookie dough, and whatever experimental flavor Lin convinced the shop to try.

You are given $n$ scoops to distribute among these flavors. The rules:

\begin{itemize}
  \item Scoops of the \emph{same} flavor are indistinguishable.
  \item The only thing that matters is how many scoops of each flavor you take.
\end{itemize}

Someone asks, with a suspicious smile:

\begin{quote}
  ``How many different sundaes can you build?''
\end{quote}

If you try to list them all by hand, your brain melts faster than the ice cream.
But with the right counting idea, you can answer the question cleanly.

This chapter is about \textbf{stars and bars} --- a way to count how many ways
we can distribute identical objects (scoops) into labeled boxes (flavors).


\section{Stars and Bars: Distributing Identical Objects}

\subsection*{From sundaes to equations}

Suppose we have $k$ flavors and $n$ scoops total.

Let:
\[
x_1 = \text{number of scoops of flavor 1}, \quad
x_2 = \text{number of scoops of flavor 2}, \quad \dots,\quad
x_k = \text{number of scoops of flavor $k$}.
\]

Every possible sundae corresponds to a solution of the equation
\[
x_1 + x_2 + \dots + x_k = n,
\]
where each $x_i \ge 0$ is an integer.

So our counting problem becomes:

\begin{quote}
  How many solutions in nonnegative integers $(x_1,\dots,x_k)$ satisfy
  \[
    x_1 + x_2 + \dots + x_k = n?
  \]
\end{quote}

Each solution is one way to build a sundae.

\subsection*{Drawing the picture: stars and bars}

We imagine each scoop as a \emph{star} ($\ast$), and we use \emph{bars} ($|$)
to separate flavors.

For example, suppose $n = 4$ scoops and $k = 3$ flavors. One possible scoop
distribution is:
\[
x_1 = 1,\quad x_2 = 2,\quad x_3 = 1.
\]

In stars-and-bars form, that looks like:
\[
\ast\ \mid\ \ast\ast\ \mid\ \ast
\]
\begin{itemize}
  \item The first block of stars (before the first bar) is flavor 1.
  \item The second block (between bars) is flavor 2.
  \item The last block (after the second bar) is flavor 3.
\end{itemize}

Another distribution, say $x_1 = 0$, $x_2 = 3$, $x_3 = 1$, would look like:
\[
\mid\ \ast\ast\ast\ \mid\ \ast
\]
Here, flavor 1 gets zero stars (no scoops), which is fine.

In general:
\begin{itemize}
  \item We have $n$ stars total (for $n$ scoops).
  \item We need $k-1$ bars to carve the line into $k$ blocks.
\end{itemize}

So every solution corresponds to a line of $n$ stars and $k-1$ bars:
\[
\underbrace{\ast \ast \dots \ast}_{n\ \text{stars}} \quad \text{and}\quad
\underbrace{| | \dots |}_{k-1\ \text{bars}},
\]
arranged in some order.

\subsection*{Counting the arrangements}

We now just have to count how many strings of length $n + (k-1)$ can be made
from $n$ identical stars and $k-1$ identical bars.

One way to think about it:

\begin{itemize}
  \item There are $n + k - 1$ total positions.
  \item We choose which $(k-1)$ of those positions will be bars.
  \item The remaining positions automatically become stars.
\end{itemize}

So the number of different star-bar arrangements is
\[
\binom{n + k - 1}{k - 1}.
\]

Each such arrangement corresponds to exactly one solution
$(x_1,\dots,x_k)$, so we arrive at the classic stars-and-bars formula:

\begin{quote}
  The number of nonnegative integer solutions to
  \[
    x_1 + x_2 + \dots + x_k = n
  \]
  is
  \[
    \binom{n + k - 1}{k - 1}.
  \]
\end{quote}

In sundae language:

\begin{quote}
  The number of ways to build a sundae with $n$ scoops and $k$ flavors
  (allowing some flavors to get zero scoops) is
  \[
    \binom{n + k - 1}{k - 1}.
  \]
\end{quote}


\subsection*{A small example: 4 scoops, 3 flavors}

Take $n = 4$ and $k = 3$.

Our formula says there should be
\[
\binom{4 + 3 - 1}{3 - 1} = \binom{6}{2} = 15
\]
different sundaes.

If we list all integer solutions to $x_1 + x_2 + x_3 = 4$ with
$x_i \ge 0$, we do indeed get 15 possibilities, such as
\[
(4,0,0), (3,1,0), (3,0,1), \dots, (0,0,4).
\]

We will let Python do the actual listing later. For now, the important thing
is to trust that the \emph{picture} (stars and bars) really matches the
\emph{equation} and the \emph{formula}.


\section{Variants: Minimums and Caps}

Real ice cream shops --- and real combinatorics problems --- often come with
extra rules.

\subsection*{Everyone gets at least one scoop}

Suppose you must use all $k$ flavors at least once. In sundae form: every
flavor gets at least one scoop. In math form:
\[
x_i \ge 1 \quad \text{for all } i, \qquad x_1 + \dots + x_k = n.
\]

We can convert this into a nonnegative problem by ``giving everyone one scoop
up front.'' Let
\[
y_i = x_i - 1.
\]
Then $y_i \ge 0$, and
\[
(x_1 + \dots + x_k = n) \quad \Longleftrightarrow \quad
(y_1 + \dots + y_k = n - k).
\]

So the number of solutions is
\[
\binom{(n - k) + k - 1}{k - 1} = \binom{n - 1}{k - 1},
\]
as long as $n \ge k$ (otherwise it’s impossible to give everyone at least one
scoop).

In sundae language:

\begin{quote}
  The number of sundaes with $n$ scoops and $k$ flavors, where every flavor
  appears at least once, is
  \[
    \binom{n - 1}{k - 1}.
  \]
\end{quote}

\subsection*{Caps: at most some number per flavor}

A trickier variation is when each flavor has a maximum number of scoops:

\begin{quote}
  Each flavor can have at most $c$ scoops.
\end{quote}

So we want integer solutions to
\[
x_1 + \dots + x_k = n, \quad 0 \le x_i \le c.
\]

There is no single magic formula as simple as the basic stars-and-bars case,
but there are strategies:

\begin{itemize}
  \item For small $n, k, c$, we can list all possibilities with Python and count.
  \item On paper, we can sometimes:
        \begin{itemize}
          \item use symmetry,
          \item break into cases,
          \item or use ``count everything, subtract the violations'' if only a few
                solutions violate $x_i \le c$.
        \end{itemize}
\end{itemize}

In this chapter, we mostly treat caps as optional challenge problems and lean
on Python to double-check our reasoning for small numbers.


\section{Python Lab: Enumerating Sundaes}

As before, we use Python as a friendly lab assistant. For sundaes, Python is
great for:

\begin{itemize}
  \item generating all small integer solutions to $x_1 + \dots + x_k = n$;
  \item verifying that the total matches $\binom{n + k - 1}{k - 1}$;
  \item experimenting with minimums and caps.
\end{itemize}

\subsection*{Step 1: Enumerate all $(x_1,\dots,x_k)$ with $x_1 + \dots + x_k = n$}

For small $n$ and $k$, we can loop over all possibilities and collect those
that sum to $n$. Conceptually, the code will:

\begin{enumerate}
  \item Fix values for $n$ and $k$.
  \item Loop over all $k$-tuples of nonnegative integers whose sum is $n$.
  \item Store them in a list (each tuple is one sundae).
  \item Count them and compare with $\binom{n + k - 1}{k - 1}$ using
        \texttt{math.comb}.
\end{enumerate}

We can then label the flavors and translate each tuple into a human-readable
description, like:

\begin{quote}
  \texttt{(2,1,1)} $\to$ 2 scoops vanilla, 1 scoop chocolate, 1 scoop strawberry.
\end{quote}

\subsection*{Step 2: Enforcing minimums}

We can modify the code so that each flavor must get at least one scoop
($x_i \ge 1$). In code, this might be as simple as:

\begin{itemize}
  \item generate all nonnegative solutions to $y_1 + \dots + y_k = n - k$,
  \item then set $x_i = y_i + 1$.
\end{itemize}

Our Python count should match $\binom{n-1}{k-1}$.

\subsection*{Step 3: Playing with caps}

For caps, we can use brute-force checking on small examples:

\begin{enumerate}
  \item Generate all $(x_1,\dots,x_k)$ with sum $n$.
  \item Filter for those where $x_i \le c$ for all $i$.
  \item Compare the number to what you get from case-by-case reasoning (if possible).
\end{enumerate}

The main point: the code does not have to be long or fancy. It just needs to
mirror the combinatorial rules.


\section{Practice: Scoops, Spells, and Skill Trees}

Here are some practice directions that fit naturally after this chapter.

\subsection*{Scoops and sundaes}

\begin{enumerate}
  \item You have $5$ scoops and $3$ flavors.
        \begin{enumerate}[label=(\alph*)]
          \item How many different sundaes can you build if some flavors may get zero scoops?
          \item How many different sundaes if every flavor must get at least one scoop?
        \end{enumerate}

  \item A shop offers $4$ toppings for your ice cream, and you may add up to
        $6$ scoops of toppings total (multiple scoops of the same topping are allowed).
        How many different topping combinations are possible, counting only how many
        scoops of each topping you take?
\end{enumerate}

\subsection*{Spells and mana points}

\begin{enumerate}
  \item In a game, a wizard has $8$ mana points to distribute among $3$ spells:
        Fire, Ice, and Lightning. Each point assigned to a spell increases its power.
        \begin{enumerate}[label=(\alph*)]
          \item How many ways can the wizard assign the $8$ points if some spells may get zero?
          \item How many ways if each spell must get at least $1$ point?
        \end{enumerate}

  \item A character has $10$ skill points to distribute among $4$ stats:
        Strength, Dexterity, Intelligence, and Charisma. Each stat can receive
        any nonnegative number of points.
        \begin{enumerate}[label=(\alph*)]
          \item How many builds are possible?
          \item How many builds if every stat must have at least $1$ point?
        \end{enumerate}
\end{enumerate}

\subsection*{Extra spicy (optional)}

\begin{enumerate}
  \item A student council has $7$ identical scholarships to distribute among
        $4$ clubs. Each club may receive any number of scholarships (including $0$).
        \begin{enumerate}[label=(\alph*)]
          \item How many distributions are possible?
          \item How many distributions if no club may receive more than $3$ scholarships?
                (Hint: this is a good place to combine stars-and-bars ideas with
                either casework or a small Python brute-force check.)
        \end{enumerate}
\end{enumerate}


\section*{Podcast: Episode 3 -- Sundae Architect}

At the end of this chapter, you might record a short podcast episode where the
characters wildly over-engineer their sundaes. A possible outline:

\begin{itemize}
  \item Cold open:
        \begin{itemize}
          \item The group is standing in front of an absurd ice cream bar.
          \item Someone claims: ``There are, like, a million sundaes you could build.''
        \end{itemize}

  \item Turn to math:
        \begin{itemize}
          \item One character models the situation with variables
                $x_1,\dots,x_k$ and the equation $x_1 + \dots + x_k = n$.
          \item They draw a stars-and-bars picture and count the arrangements.
          \item They translate the formula $\binom{n + k - 1}{k - 1}$ back into plain language.
        \end{itemize}

  \item Variants:
        \begin{itemize}
          \item Someone insists on using \emph{every} flavor at least once.
          \item Someone else wants a cap like “no more than three scoops of any one flavor.”
          \item They discuss how the counts change, and when it is easier to let Python
                handle the messy details.
        \end{itemize}

  \item Closing:
        \begin{itemize}
          \item The group realizes this same math applies to:
                \begin{itemize}
                  \item distributing skill points in games,
                  \item splitting resources in planning problems,
                  \item and even some probability questions.
                \end{itemize}
          \item Teaser for the next chapter:
                \begin{quote}
                  ``We’ve counted sundaes, stats, and scholarships.
                  Next up: dice, randomness, and whether the universe is actually trolling you.’’
                \end{quote}
        \end{itemize}
\end{itemize}

