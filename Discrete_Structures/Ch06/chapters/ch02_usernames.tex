\chapter{Usernames, License Plates, and Other Noisy Strings}

\section{Story Hook: Launch Day Username Panic}
% - Story:
%   * New online game: we must choose a username rule for millions of players.
%   * Question: “Is our rule going to run out of names?”
% - Example formats:
%   * 4 letters: LLLL
%   * Letter-letter-digit-digit: LLDD
%   * 3-digit PINs for a door lock.

\section{Counting Strings with Repetition}
% - Model strings as positions filled from an alphabet:
%   * Letters, digits, or mixed sets.
% - Product principle:
%   * Each position = independent choice => total = product of options.
% - Examples:
%   * 26^4 usernames with 4 uppercase letters.
%   * 10^3 PINs.
%   * (26 \cdot 10^3) for LDDD.
% - Discuss:
%   * Adding more characters increases search space dramatically.
%   * Simple tie-in to password strength (without going full crypto).

\section{Python Lab: Exploring the Username Space}
% - Script sketch (usernames.py):
%   * Define alphabets: letters, digits, maybe “safe symbols”.
%   * For small toy alphabets:
%       - Use itertools.product to generate and list all strings.
%   * For realistic alphabets:
%       - Just count combinations by formula, maybe sample a few at random.
%   * Optional filters:
%       - At least one digit.
%       - No “banned” characters or substrings.

\section{Practice and Design Questions}
% - Exercises:
%   * Compute number of usernames under different policies.
%   * Compare policies by total count: which is “bigger” and why?
%   * Design a username rule that:
%       - Feels simple to users,
%       - But has at least N possible usernames.
% - Quick writing prompts:
%   * Explain in words how the product rule is hiding in these examples.

\section*{Podcast: Episode 2 -- Name Yourself Wisely}
% - Recap script notes:
%   * Characters try to register their favorite names and hit “name already taken”.
%   * They experiment with stricter/looser rules.
%   * Joke about bad passwords vs good passwords.
%   * Tease sundaes and resource distribution coming up.

