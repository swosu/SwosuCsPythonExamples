\chapter{Welcome to \emph{Counting Worlds}}

\section{A Story About Too Many Possibilities}

Imagine this:

\begin{itemize}
  \item You open a new game.
  \item You have to choose a character: race, class, hairstyle, outfit, special skill.
  \item Five minutes later, you are still on the character creation screen.
\end{itemize}

It feels like there are \emph{infinitely many} options. In reality, there is a
large but finite \emph{universe of possibilities}. This chapter---and this whole
mini-unit---is about learning how to \emph{count} those universes.

We are going to ask questions like:

\begin{itemize}
  \item How many ways can we seat six friends in a row of chairs?
  \item How many usernames can a website support under a given naming rule?
  \item How many sundaes can we build from a limited supply of scoops and flavors?
  \item How often should a particular dice pattern show up if our math is right?
  \item How big is the universe of characters, decks, or skill builds in a game?
\end{itemize}

Instead of guessing, we will use discrete mathematics and a bit of Python to get
real answers. Along the way, we will see that:

\begin{quote}
  \emph{Counting is how we tame spaces that feel infinite.}
\end{quote}


\section{What You Will Be Able to Do}

By the end of this counting unit, you should be able to:

\subsection*{Mathematical skills}

\begin{itemize}
  \item Use the product principle (``AND'' means multiply) and the sum principle
        (``either/or'' means add, when choices are disjoint) in real problems.
  \item Work with \emph{permutations} (where order matters) using factorials.
  \item Work with \emph{combinations} (where order does not matter) using
        binomial coefficients $\binom{n}{k}$.
  \item Use the \emph{stars and bars} technique to count ways of distributing
        identical items (like scoops, points, or coins) into bins.
  \item Connect counting to \emph{probability} in simple finite situations.
\end{itemize}

\subsection*{Computational skills}

\begin{itemize}
  \item Use Python as an ``experimental math lab'':
        \begin{itemize}
          \item generate all outcomes for small cases;
          \item check whether a formula seems to be right;
          \item estimate probabilities via Monte Carlo simulation.
        \end{itemize}
  \item Translate informal stories (about games, food, or chaos) into:
        \begin{itemize}
          \item precise combinatorial models, and
          \item short, readable Python scripts.
        \end{itemize}
\end{itemize}

\subsection*{Design and creativity}

\begin{itemize}
  \item Design your own small ``universe'' (a game, system, or scenario).
  \item Identify where permutations, combinations, and stars-and-bars appear
        inside that universe.
  \item Justify your counting formulas in words, not just symbols.
\end{itemize}


\section{How This Mini-Book Works}

Each chapter follows the same basic rhythm:

\begin{enumerate}[label=\arabic*.]
  \item \textbf{Start with a story.}  
        We begin with something that feels familiar: seating friends, building
        sundaes, creating usernames, rolling dice, or designing a game world.

  \item \textbf{Extract the structure.}  
        We strip away the flavor and look at the underlying combinatorial
        skeleton: choices, constraints, order vs.\ no order, repetition vs.\
        no repetition.

  \item \textbf{Do the math.}  
        We use the tools of discrete mathematics to turn the story into
        symbols and formulas and compute exact counts.

  \item \textbf{Check or explore with Python.}  
        For small versions of the problem, we let a Python script:
        \begin{itemize}
          \item generate the whole universe explicitly,
          \item count how many outcomes have a certain property,
          \item approximate probabilities with random trials.
        \end{itemize}

  \item \textbf{Reflect and remix.}  
        At the end of the chapter, we check understanding with short
        exercises and a brief ``podcast'' conversation that ties the ideas
        back to the story.
\end{enumerate}

You do \emph{not} need to be a Python expert. The code will be short and
focused, and you can treat it as executable pseudocode if you are still
learning the language.


\section{Python as Our Counting Laboratory}

Here is roughly how Python shows up throughout this unit:

\begin{itemize}
  \item We use \texttt{itertools} to generate permutations, combinations,
        and simple products (like all possible usernames of a given form).
  \item We use the \texttt{math} module for things like \texttt{math.factorial}
        and \texttt{math.comb}.
  \item We use the \texttt{random} module to simulate coin flips, dice rolls,
        and other random experiments.
\end{itemize}

Most scripts will follow this template:

\begin{enumerate}[label=\arabic*.]
  \item Set up a small, concrete version of the problem.
  \item Compute a theoretical count or probability using a formula.
  \item Use Python to:
        \begin{itemize}
          \item either enumerate all possibilities, or
          \item run a large number of random trials.
        \end{itemize}
  \item Compare the result from Python with the formula.
\end{enumerate}

You will be encouraged to modify the scripts:
change parameters, add constraints, or invent your own stories that use the
same combinatorial ideas.


\section{Roadmap for the Five Chapters}

Here is the plan for the rest of this counting unit:

\subsection*{Chapter 1: Seating the Party at the Round Table}

We start with \emph{factorials} and \emph{permutations}. You will:

\begin{itemize}
  \item Count the number of ways to seat people in a row.
  \item Handle restrictions (must sit together, must not sit together, fixed seats).
  \item Use Python to brute-force small seating problems and check your formulas.
\end{itemize}

\subsection*{Chapter 2: Usernames, License Plates, and Other Noisy Strings}

We move to \emph{strings with repetition}. You will:

\begin{itemize}
  \item Model usernames, license plates, and PINs as strings from an alphabet.
  \item Use the product principle to count how many such strings exist.
  \item Compare different username rules and discuss which ones create larger
        (or smaller) universes of names.
\end{itemize}

\subsection*{Chapter 3: How Many Sundaes Can We Build?}

We introduce \emph{stars and bars}. You will:

\begin{itemize}
  \item Count ways to distribute scoops among flavors or points among skills.
  \item Visualize stars-and-bars as a picture and connect it to a formula.
  \item Use Python to generate all small distributions and verify your counts.
\end{itemize}

\subsection*{Chapter 4: Roll the Dice, Check the Math}

We tie counting to \emph{probability} and \emph{simulation}. You will:

\begin{itemize}
  \item Compute exact probabilities using combinatorial reasoning.
  \item Run Monte Carlo simulations in Python to estimate those probabilities.
  \item Watch simulated frequencies drift toward theoretical values as
        the number of trials increases.
\end{itemize}

\subsection*{Chapter 5: Design Your Own Universe}

Finally, you become the world-builder. You will:

\begin{itemize}
  \item Design a small universe of your choice (game, system, or scenario).
  \item Identify at least one permutation, one combination, and one
        stars-and-bars situation inside it.
  \item Analyze your universe with formulas \emph{and} Python, then present
        your findings.
\end{itemize}


\section*{Podcast: Episode 0 --- Trailer for the Counting Worlds}

At the end of this intro, there is a short podcast episode that you can
listen to while walking, commuting, or setting up your Python environment.
The episode can follow this rough script:

\begin{itemize}
  \item Introduce the ``cast'' of the unit (for example, recurring student
        characters or narrators).
  \item Share one real-life story where counting matters
        (picking teams, building a schedule, designing a game).
  \item Tease each chapter in one sentence:
        \begin{itemize}
          \item ``First, we figure out how many ways to seat a chaotic friend group.''
          \item ``Then, we see how many usernames and license plates fit in the system.''
          \item ``Next, we over-engineer sundae bars and skill trees.''
          \item ``After that, we roll dice to see if our math holds up.''
          \item ``Finally, \emph{you} design your own universe and count it.''
        \end{itemize}
  \item Close with a simple challenge:
        \begin{quote}
          Before Chapter 1, try to guess:  
          \emph{How many different outfits can you build from your own closet?}
        \end{quote}
\end{itemize}

This podcast is not graded; it is here to set the tone, tell stories, and
give you a human voice walking you into the math.


