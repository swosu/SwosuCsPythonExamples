\chapter{Math Writeup: Counting Presidents of Virtue}

This chapter connects the story and rules of \emph{Presidents of Virtue} to four core
counting ideas from discrete mathematics:

\begin{itemize}
    \item permutations (when order matters),
    \item combinations (when order does not matter),
    \item stars-and-bars (distributing identical things into labeled boxes),
    \item and probability (using counts to measure how likely an event is).
\end{itemize}

Each section below includes a clearly labeled \textbf{Problem}, the relevant
\textbf{Formula}, and a fully \textbf{Worked Example} in the DeathSpank / Presidents
of Virtue universe.

\section{Permutations: Ordering the Chain of Command}

In each round of \emph{Presidents of Virtue}, players race to empty their hands.
The order in which they finish determines the social ladder for the next round:
Hero of SpankTopia, Deputy of Justice, Adventurers, Minion of AntiSpank, and
finally the AntiSpank.

\subsection*{Problem (Permutations)}

Suppose there are $n$ distinct players at the table. At the end of a round, all
titles are assigned by finishing order: first place becomes Hero of SpankTopia,
second becomes Deputy of Justice, and so on, down to the AntiSpank in last
place.

\begin{quote}
\textbf{Question.} How many different ways can the titles be assigned after one
round, assuming every finishing order is possible?
\end{quote}

\subsection*{Formula (Permutations)}

A \emph{permutation} of $n$ distinct objects is any ordering of them in a line.
The number of permutations of $n$ distinct objects is
\[
n! = n \cdot (n-1) \cdot (n-2) \cdots 3 \cdot 2 \cdot 1,
\]
with the special convention that $0! = 1$.

In words: for the first position we have $n$ choices; for the second, $n-1$
choices; and so on, until only one choice remains. By the product principle, we
multiply these choices together to get $n!$ possible orders.

\subsection*{Worked Example}

Let $n=5$ players sit down to play: DeathSpank, Sparkles, Steve, a Random
Adventurer, and the Mysterious Clerk of Bacon.

At the end of the round, the finishing order determines:

\begin{center}
\begin{tabular}{cl}
1st & Hero of SpankTopia \\
2nd & Deputy of Justice \\
3rd & Adventurer \\
4th & Minion of AntiSpank \\
5th & AntiSpank \\
\end{tabular}
\end{center}

\noindent
\textbf{How many possible ways can these titles be assigned?}

Because all five characters are distinct, this is just the number of permutations
of $5$ distinct players:
\[
5! = 5 \cdot 4 \cdot 3 \cdot 2 \cdot 1 = 120.
\]

So there are $120$ different ways the round could end in terms of titles. Even
with only five players, the space of possible ``political timelines'' for
SpankTopia is already quite large.

\vspace{1em}
\noindent
\textbf{Design note.} If a designer wanted to guarantee that some specific
character (say, DeathSpank) \emph{never} becomes AntiSpank, they would be
forbidding all permutations with that character in last place. That is a
structural change to the game, and permutations give us a precise language for
describing it.

\section{Combinations: Choosing Sets of Thongs of Virtue}

Between rounds, the Hero of SpankTopia may be allowed to equip special artifacts
or Thongs of Virtue that grant small advantages (extra card trades, one-time
Justice Bursts, or fancy decorative glow).

In many situations, it is the set of artifacts that matters, not the order in
which they are chosen.

\subsection*{Problem (Combinations)}

Suppose there are $n$ distinct Thongs of Virtue stored in the Sacred Drawer of
Laundry. At the start of a campaign night, the Hero is allowed to choose $k$ of
them as their \emph{loadout} for the evening.

\begin{quote}
\textbf{Question.} How many different loadouts of $k$ Thongs can the Hero
choose, if the order of selection does not matter?
\end{quote}

\subsection*{Formula (Combinations)}

A \emph{combination} answers the question: ``In how many ways can we choose $k$
objects from $n$ distinct objects when order does not matter?''

The number of such combinations is
\[
\binom{n}{k}
= \frac{n!}{k!(n-k)!},
\]
read as ``$n$ choose $k$.''

The denominator divides out over-counting, since each \emph{set} of $k$ objects
can be listed in $k!$ different orders.

\subsection*{Worked Example}

Assume there are $n=7$ distinct Thongs of Virtue:

\begin{center}
Justice, Stealth, Bacon, Looting, Mana, Friendship, and Mildly Confusing Glow.
\end{center}

Before the first round, the Hero may choose $k=3$ of these to wear (in a safe,
family-friendly way). Only the set matters; wearing Justice--Bacon--Mana is the
same loadout as Mana--Justice--Bacon.

The number of possible loadouts is
\[
\binom{7}{3}
= \frac{7!}{3!\,4!}
= \frac{7 \cdot 6 \cdot 5}{3 \cdot 2 \cdot 1}
= 35.
\]

So there are $35$ distinct sets of three Thongs the Hero could bring into the
session. If you later tweak the balance by adding more artifacts, you can use
the same combination formula to see how quickly the loadout universe grows.

\section{Stars-and-Bars: Distributing Justice Points}

Many role-playing games let a character spread points across several abilities.
In \emph{Presidents of Virtue}, we might imagine DeathSpank distributing
\emph{Justice Points} among different powers before the game starts.

\subsection*{Problem (Stars-and-Bars)}

DeathSpank has $J$ Justice Points to allocate among $k$ special powers:

\begin{itemize}
    \item Smite of Righteousness,
    \item Flaming Bacon Shield,
    \item and Loot Sense.
\end{itemize}

Let $x_1, x_2, x_3$ be the number of points assigned to these three powers,
respectively.

\begin{quote}
\textbf{Question A.} If $J = 6$ and $k=3$, and each power may receive any
nonnegative number of points (including zero), how many different allocations
$(x_1, x_2, x_3)$ are possible?

\textbf{Question B.} Answer the same question, but now require that every power
must receive at least one point.
\end{quote}

\subsection*{Formula (Stars-and-Bars)}

We are counting the number of nonnegative integer solutions to
\[
x_1 + x_2 + \cdots + x_k = J
\]
with $x_i \ge 0$.

The classic stars-and-bars result says that the number of such solutions is
\[
\binom{J + k - 1}{k - 1}.
\]

If instead every power must receive at least one point, we set $y_i = x_i - 1$
so $y_i \ge 0$ and
\[
y_1 + y_2 + \cdots + y_k = J - k.
\]
Then the number of solutions with $x_i \ge 1$ is
\[
\binom{(J - k) + k - 1}{k - 1}
= \binom{J - 1}{k - 1}.
\]

\subsection*{Worked Example}

\paragraph{Question A: Some powers may get zero.}

Here $J = 6$ and $k = 3$, so we count nonnegative integer solutions to
\[
x_1 + x_2 + x_3 = 6.
\]
By stars-and-bars, the number of allocations is
\[
\binom{6 + 3 - 1}{3 - 1}
= \binom{8}{2}
= \frac{8 \cdot 7}{2 \cdot 1}
= 28.
\]

So there are $28$ ways for DeathSpank to distribute $6$ Justice Points among
three powers if some powers are allowed to be left at zero.

\paragraph{Question B: Every power must get at least one point.}

Now we require $x_1, x_2, x_3 \ge 1$ and $x_1 + x_2 + x_3 = 6$.

Set $y_i = x_i - 1$, so $y_i \ge 0$ and
\[
y_1 + y_2 + y_3 = 6 - 3 = 3.
\]
By stars-and-bars, the number of such allocations is
\[
\binom{3 + 3 - 1}{3 - 1}
= \binom{5}{2}
= \frac{5 \cdot 4}{2 \cdot 1}
= 10.
\]

So there are $10$ allocations in which every power gets at least one Justice
Point. In game-design language: enforcing ``no dump stats'' shrinks the build
universe from $28$ to $10$ possibilities.

\section{Probability: Justice Bursts in a Starting Hand}

Counting lets us describe how \emph{big} a universe of possibilities is. Probability
uses those counts to describe how \emph{likely} certain events are, assuming all
hands are equally likely.

\subsection*{Problem (Probability with Combinations)}

In \emph{Presidents of Virtue}, the card \texttt{2} in each suit is treated as a
\emph{Justice Burst}---a very powerful card that often clears the trick.

Suppose we are using a standard $52$-card deck with four suits and four Justice
Bursts (the four \texttt{2}s). A player is dealt a $5$-card starting hand.

\begin{quote}
\textbf{Question.} What is the probability that the player’s starting hand
contains \emph{exactly one} Justice Burst?
\end{quote}

\subsection*{Formula (Probability from Counting)}

When all $5$-card hands are equally likely, we can write
\[
\Pr(\text{exactly one Justice Burst})
= \frac{\text{number of hands with exactly one 2}}{\text{number of all $5$-card hands}}.
\]

\begin{itemize}
    \item The number of all $5$-card hands from a $52$-card deck is
    \[
    \binom{52}{5}.
    \]
    \item To have \emph{exactly one} Justice Burst:
    \begin{itemize}
        \item choose which one of the four \texttt{2}s appears: $\binom{4}{1}$ ways;
        \item choose the remaining $4$ cards from the $48$ non-\texttt{2} cards:
        $\binom{48}{4}$ ways.
    \end{itemize}
    So the number of favorable hands is
    \[
    \binom{4}{1} \binom{48}{4}.
    \]
\end{itemize}

Therefore
\[
\Pr(\text{exactly one Justice Burst})
= \frac{\binom{4}{1} \binom{48}{4}}{\binom{52}{5}}.
\]

\subsection*{Worked Example}

We can either leave the answer in binomial form (which is already meaningful), or
evaluate it numerically.

First, keep it symbolic:
\[
\Pr(\text{exactly one Justice Burst})
= \frac{\binom{4}{1}\binom{48}{4}}{\binom{52}{5}}.
\]

If we expand:
\[
\binom{4}{1} = 4,\qquad
\binom{48}{4} = \frac{48 \cdot 47 \cdot 46 \cdot 45}{4 \cdot 3 \cdot 2 \cdot 1},\qquad
\binom{52}{5} = \frac{52 \cdot 51 \cdot 50 \cdot 49 \cdot 48}{5 \cdot 4 \cdot 3 \cdot 2 \cdot 1}.
\]

Computing these (by hand or with a calculator) gives approximately
\[
\Pr(\text{exactly one Justice Burst})
\approx 0.2995,
\]
or about a $30\%$ chance.

In other words, if you sit down and are repeatedly dealt random $5$-card hands
from a fresh deck, you should expect to see \emph{exactly one} Justice Burst in
roughly $3$ out of every $10$ hands, on average.

\vspace{1em}
\noindent
\textbf{Design note.} If you want Justice Bursts to be more common in starting
hands, you could:

\begin{itemize}
    \item add more special cards to the deck, or
    \item increase the starting hand size.
\end{itemize}

Either way, the combination formulas above let you recompute the exact
probabilities and make data-informed design decisions.

\section*{Summary}

In this chapter we:

\begin{itemize}
    \item used \textbf{permutations} to count possible chains of command at the end of
    a round;
    \item used \textbf{combinations} to count how many artifact loadouts or Thong sets
    a Hero can choose;
    \item used \textbf{stars-and-bars} to count how many Justice Point builds are
    possible for a character;
    \item and used \textbf{probability} to quantify how often special cards (Justice
    Bursts) appear in random hands.
\end{itemize}

These four tools give a mathematical backbone to the \emph{Presidents of Virtue}
universe. They also serve as a template: any time you invent a new mini-game or
variant, you can ask the same questions:

\begin{quote}
How many different configurations are possible, and how likely are the ones I
care about?
\end{quote}

Answering those questions is where discrete mathematics and game design shake
hands.

