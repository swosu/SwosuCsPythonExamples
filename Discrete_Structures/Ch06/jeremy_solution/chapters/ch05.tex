\chapter{Reading the Presidents of Virtue Log Like a Data Scientist}
\label{ch:pov-data}

In the previous chapters we designed the game universe, implemented the
\texttt{PresidentsOfVirtueRound} engine, and wired up strategies and simulations.
Now we do something deliciously modern: we treat the play log as a
\emph{dataset}.

This chapter is about reading the CSV file
\texttt{presidents\_of\_virtue\_plays.csv} and asking questions like a
data scientist:

\begin{itemize}
  \item Which strategies tend to finish earlier?
  \item How often do players trigger a Justice Burst?
  \item What happens to hand sizes over the course of a round?
\end{itemize}

We are not trying to build a full machine learning model (yet), but we \emph{are}
trying to get comfortable with turning a log of events into structured questions
and answers.

\section{Where the CSV Comes From}

When you run the simulation script,
\begin{lstlisting}[language=bash]
cd SwosuCsPythonExamples/Discrete_Structures/Ch06/jeremy_solution
python scripts/simulate_game.py
\end{lstlisting}
it will print several rounds of game play to the console and then write a file:
\begin{center}
  \texttt{presidents\_of\_virtue\_plays.csv}
\end{center}

Each row in this CSV represents \emph{one action} taken by one player during one
trick of one round. It includes both game state and labels that will be useful
for statistics and machine learning.

\section{What the Columns Mean}

Here is a quick tour of the columns in the log; these names come directly from
the engine:

\begin{itemize}
  \item \textbf{\texttt{round}}: which round of the simulation this action
        belongs to (1, 2, 3, \dots).
  \item \textbf{\texttt{trick}}: which trick within the round (starts at 1 each round).
  \item \textbf{\texttt{step}}: the order of actions inside a trick
        (1 = first action in that trick).
  \item \textbf{\texttt{player\_name}}: the seat at the table (\texttt{Cody},
        \texttt{Savannah}, \dots).
  \item \textbf{\texttt{strategy}}: the short label for that player's strategy
        (\texttt{Cautious}, \texttt{Greedy}, \texttt{PairLover}, \texttt{ChaosRevolutionary}, \texttt{Random}, etc.).
  \item \textbf{\texttt{action}}: either \texttt{play} or \texttt{pass}.
  \item \textbf{\texttt{cards\_played}}: a space--separated string naming the
        cards played, for example ``Q of hearts, Q of clubs, Q of diamonds'' or
        ``2 of spades''. This is empty for passes.
  \item \textbf{\texttt{hand\_size\_before}} and \textbf{\texttt{hand\_size\_after}}:
        how many cards the player held before and after that action.
  \item \textbf{\texttt{current\_size\_before}} and
        \textbf{\texttt{current\_rank\_before}}: what the table looked like
        \emph{before} this action (for example, ``there is a pair of 8s out'').
  \item \textbf{\texttt{is\_lead}}: \texttt{True} if this action started a new
        trick (a lead), \texttt{False} otherwise.
  \item \textbf{\texttt{is\_justice\_burst}}: \texttt{True} if the player used a
        2 to slam the table, clear the trick, and take back control.
  \item \textbf{\texttt{is\_revolution\_trigger}}: \texttt{True} if this play
        was a Bacon Revolution (four of a kind) that flipped the rank order.
  \item \textbf{\texttt{revolution\_state\_after}}: \texttt{True} when the
        round is currently in the ``revolution'' rank order, \texttt{False}
        otherwise.
  \item \textbf{\texttt{finish\_position}}: when this row was written we also
        tagged each player with their final finishing place (1 for first, 2 for
        second, etc.). This is constant for a given player within a given round.
  \item \textbf{\texttt{ended\_on\_bomb}}: whether the player ended the round
        by playing a 2 (a bomb).
\end{itemize}

The most important idea is: \textbf{each row is one event in time}. If we sort
by \texttt{(round, trick, step)}, we see the game unfold action by action.

\section{First Contact: Opening the CSV}

There are two very common ways to peek at a CSV:

\begin{enumerate}
  \item In a spreadsheet program (Excel, LibreOffice, Google Sheets).
  \item In Python, using either the built--in \texttt{csv} module or a library
        like \texttt{pandas}.
\end{enumerate}

\subsection{Spreadsheet View}

From your file manager or terminal, you can open the CSV directly in a
spreadsheet tool. This is great for a quick visual scan:

\begin{itemize}
  \item Filter by \texttt{player\_name} to see what one player did.
  \item Filter by \texttt{action==play} to see only plays (no passes).
  \item Sort by \texttt{finish\_position} to see which strategies are winning.
\end{itemize}

For deeper work, though, we want code.

\subsection{Python View with \texttt{csv}}

Here is a small, self--contained example script that just prints the first
few rows and counts them. Save this as \texttt{scripts/pov\_peek.py}:

\begin{lstlisting}[language=Python]
import csv
from pathlib import Path

def main():
    # Assume we run from the jeremy_solution folder
    csv_path = Path("presidents_of_virtue_plays.csv")

    if not csv_path.exists():
        print("CSV not found. Did you run simulate_game.py first?")
        return

    with csv_path.open(newline="", encoding="utf-8") as f:
        reader = csv.DictReader(f)
        rows = list(reader)

    print(f"Loaded {len(rows)} actions from {csv_path}.\n")
    print("First 5 actions:")
    for row in rows[:5]:
        print(
            f"round {row['round']}, trick {row['trick']}, step {row['step']}: "
            f\"{row['player_name']} ({row['strategy']}) "
            f\"{row['action']} {row['cards_played']}\"
        )

if __name__ == "__main__":
    main()
\end{lstlisting}

Run it from the \texttt{jeremy\_solution} directory:

\begin{lstlisting}[language=bash]
python scripts/pov_peek.py
\end{lstlisting}

If that works, you now have a tiny analysis pipeline: simulation $\rightarrow$
CSV $\rightarrow$ Python $\rightarrow$ printed insight.

\section{Exercise 1: Warm--Up Data Slurp}

\textbf{Goal:} practice reading the CSV and extracting basic stats.

\subsection*{Task A: Count Actions by Strategy}

Write a Python script \texttt{pov\_count\_actions.py} that:

\begin{enumerate}
  \item Reads \texttt{presidents\_of\_virtue\_plays.csv} with
        \texttt{csv.DictReader}.
  \item Builds a dictionary mapping:
        \[
        \text{strategy name} \longmapsto \text{number of actions (rows)}
        \]
  \item Prints each strategy and its total number of actions.
\end{enumerate}

You might see output like:
\begin{verbatim}
Cautious: 130 actions
Greedy: 128 actions
PairLover: 125 actions
ChaosRevolutionary: 132 actions
Random: 129 actions
\end{verbatim}

\subsection*{Task B: Count Plays vs Passes}

Extend your script or write a second one that counts how many actions were
\texttt{play} and how many were \texttt{pass}. Then break it down by strategy:
\[
(\text{strategy}, \text{action}) \longmapsto \text{count}.
\]

Questions to reflect on:

\begin{itemize}
  \item Which strategy passes the most?
  \item Does this match your intuition from reading the strategy descriptions?
\end{itemize}

\section{Exercise 2: Strategy Scorecards}

\textbf{Goal:} connect the \texttt{finish\_position} field to the strategies.

Each player appears many times in the log during a round, but their
\texttt{finish\_position} is the same in all those rows for that round. We can
treat \texttt{finish\_position} as a label for that (player, round) combo.

Write a Python script \texttt{pov\_strategy\_scorecard.py} that:

\begin{enumerate}
  \item Reads the CSV with \texttt{DictReader}.
  \item Builds a mapping from:
  \[
    (\text{round}, \text{player\_name}) \longmapsto
       (\text{strategy}, \text{finish\_position})
  \]
     The simplest way is to only record a (round, player) pair the \emph{first}
     time you see it.
  \item After reading all rows, build a table for each strategy:
  \[
     \text{strategy} \longmapsto
     \{\text{finish positions seen}\}.
  \]
  \item For each strategy, compute:
    \begin{itemize}
      \item how many rounds it appeared in,
      \item the average finish position
            (lower is better: 1.0 is best),
      \item how many times it finished first.
    \end{itemize}
\end{enumerate}

For example, your script might print something like:
\begin{verbatim}
Cautious          : avg finish 2.33 over 3 rounds, 1 wins
Greedy            : avg finish 3.00 over 3 rounds, 0 wins
PairLover         : avg finish 1.67 over 3 rounds, 2 wins
ChaosRevolutionary: avg finish 3.67 over 3 rounds, 0 wins
Random            : avg finish 2.33 over 3 rounds, 1 wins
\end{verbatim}

Your numbers will depend on how many rounds you simulated.

\subsection*{Reflection}

\begin{itemize}
  \item Which strategies seem strong in this small sample?
  \item Do you trust the results yet, or do you feel like you need more data?
  \item What would happen if you changed the strategies slightly and re-ran?
\end{itemize}

This is a tiny version of an experiment: we are estimating performance from
sampled games, and sampling error is very real.

\section{Exercise 3: Justice Bursts and Bomb Endings}

\textbf{Goal:} use the special flags \texttt{is\_justice\_burst} and
\texttt{ended\_on\_bomb} to study dramatic events in the game.

\subsection*{Task A: How Often Do Strategies Trigger Justice Bursts?}

Write a script \texttt{pov\_justice\_stats.py} that:

\begin{enumerate}
  \item Reads the CSV.
  \item For each strategy, counts how many actions had
        \texttt{is\_justice\_burst == True}.
  \item Optionally, also compute the fraction:
        \[
           \frac{\text{\# Justice Bursts}}{\text{\# actions for that strategy}}.
        \]
\end{enumerate}

Questions:
\begin{itemize}
  \item Does the Chaos/Revolution strategy really cause more Justice Bursts?
  \item Do cautious players ever use them, or do they mostly avoid them?
\end{itemize}

\subsection*{Task B: Who Ends on a Bomb?}

The column \texttt{ended\_on\_bomb} is associated with the \emph{final play}
that took a player out. Since we carried that label into every row for that
(player, round), we can still extract it from the CSV.

Extend your analysis to compute, for each strategy:

\begin{itemize}
  \item how many rounds that strategy ended on a bomb, and
  \item the fraction of its rounds that ended on a bomb.
\end{itemize}

You might discover that some strategies rarely end on bombs (they burn their 2s
earlier) while others like to hoard them until the final punch.

\section{Optional: Using \texttt{pandas}}

If you are comfortable installing extra libraries, you can also use the
\texttt{pandas} library to treat the CSV as a DataFrame. This often makes
grouping and aggregation more concise.

A minimal example (saved as \texttt{scripts/pov\_pandas\_demo.py}):

\begin{lstlisting}[language=Python]
import pandas as pd

def main():
    df = pd.read_csv("presidents_of_virtue_plays.csv")

    print(df.head())  # first 5 rows

    # Count actions by strategy
    print("\nActions per strategy:")
    print(df["strategy"].value_counts())

    # Justice bursts per strategy
    jb = df[df["is_justice_burst"] == True]
    print("\nJustice bursts per strategy:")
    print(jb["strategy"].value_counts())

if __name__ == "__main__":
    main()
\end{lstlisting}

If you go this route, be sure you can explain \emph{what} each line is doing in
plain language. The point of this chapter is not to memorize pandas, but to get
used to treating a play log as a structured, countable universe.

\section{Big Picture}

By this point you have:

\begin{itemize}
  \item designed a combinatorial universe (the card game rules),
  \item implemented it as a Python engine,
  \item generated play logs as CSV,
  \item and started to ask quantitative questions about strategies and outcomes.
\end{itemize}

This is exactly the kind of pipeline that many real data scientists and machine
learning engineers use:

\begin{center}
  \textbf{Universe / Rules $\rightarrow$ Simulation / Logs $\rightarrow$
  CSV / Tables $\rightarrow$ Models / Decisions}
\end{center}

In the next steps of your journey, you can plug this data into more advanced
models, or design entirely new universes and repeat the experiment. For now,
enjoy the feeling that you can read a game log like a scientist, not just a
spectator.

