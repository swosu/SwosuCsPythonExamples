\chapter{Why Games Work (and Why Monty Hall Still Haunts Our Brains)}

\section{Games as a learning engine}
Games are fun for a simple reason: they turn decision-making into a tight loop.
You learn the rules, take an action, observe the outcome, and adjust. That loop is
fast, emotional, and memorable.

For learning programming and math, games are especially useful because they quietly
force us to think in the same structures we use in computing:
\begin{itemize}
  \item \textbf{State:} what is true right now?
  \item \textbf{Transitions:} what event causes the next step?
  \item \textbf{Rules:} what actions are allowed at each step?
  \item \textbf{Evidence:} how do we know our strategy is good?
\end{itemize}

This project uses a game (the Monty Hall problem) as a friendly doorway into three
serious ideas: finite state machines, simulation-driven evidence, and testable software design.

\section{Why game shows became a television superpower}
Game shows fit television extremely well. A viewer can drop in mid-episode and still
understand the stakes quickly. Compared to many scripted programs, game shows can also
be produced efficiently, and their structure naturally leaves room for advertising.

Television history also includes an important cautionary tale: in the 1950s, several quiz
shows were revealed to have been manipulated, leading to public backlash and investigations.
That period helped shape later expectations around televised contests.\cite{pbs_quizshow_after}

Over time, successful daytime game shows leaned into formats that were simple, repeatable,
and brand-safe. The result was programming that could attract steady audiences and therefore
steady advertising dollars.

\section{Where the money comes from (prizes, sponsors, and the network)}
The financial model behind many game shows is straightforward in \emph{structure} even when the
exact numbers are opaque in \emph{detail}:
\begin{itemize}
  \item Networks sell advertising time against the audience the show brings in.
  \item Production budgets pay for staff, sets, crew, post-production, and distribution.
  \item Prizes may be paid directly by the show/network or supplied/offset by sponsors as part
        of promotional arrangements.\cite{wrapbook_prizemoney}
\end{itemize}

A practical research note for students: you will often see people ask questions like
\emph{“How much did this show gross?”} or \emph{“How much profit did the station make?”}
Those figures are rarely reported cleanly at the single-show level. Networks and studios tend
to report finances at the company, division, or season level. So in this chapter we focus on what
we can support well: the business logic, the longevity of the format, and the way audience attention
translates into revenue.

\section{Monty Hall: the man behind the doors}
Monty Hall was a Canadian-American broadcaster and producer best known as the host and co-creator of
\textit{Let’s Make a Deal}. The Television Academy notes that he appeared in more than 4,700 episodes of the
show.\cite{emmy_monty_hall}

This matters for us because it explains why the Monty Hall puzzle became culturally sticky:
the show was not just a one-off novelty. It was a long-running, high-visibility format that
normalized suspenseful choice as entertainment.

\section{\textit{Let’s Make a Deal}: a marketplace of suspense}
\textit{Let’s Make a Deal} is built around bargaining and uncertainty. Contestants (often called ``traders'')
make choices between known rewards and unknown possibilities hidden behind doors, curtains, or boxes.
Sometimes the unknown is a great prize; sometimes it is a humorous ``zonk.''

The show began in the 1960s and has had multiple runs and revivals across networks and syndication.\cite{emmy_lmadt}
One reason the format has lasted is that it is a reliable machine for emotional moments:
anticipation, risk, regret, surprise, and celebration.

Modern museum commentary emphasizes the importance of the show’s daytime audience and how that audience
connects to advertising value.\cite{strong_lmadt_2023}

\section{The modern revival}
A modern version of \textit{Let’s Make a Deal} has aired on CBS since 2009.\cite{cbs_lmadt}
Even as sets and hosts evolve, the core ingredient remains the same: forced choices under uncertainty.

\section{The three-door game (and the puzzle it inspired)}
The classroom-friendly Monty Hall problem usually appears in a simplified, clean form:

\begin{enumerate}
  \item There are three doors: behind one is a prize, behind the other two are goats.
  \item You pick a door.
  \item The host (who knows where the prize is) opens one of the other doors to reveal a goat.
  \item You are offered a final choice: \textbf{stay} with your original door or \textbf{switch}
        to the remaining closed door.
\end{enumerate}

This becomes the famous Monty Hall problem. Under the usual assumptions, switching wins with probability
$2/3$, while staying wins with probability $1/3$.\cite{britannica_montyhall}

Why does this puzzle matter in a computing course? Because it is a perfect example of how humans can feel
confident and still be wrong. Our intuition tends to treat the final choice as ``50/50,'' but the process
that produced the final two doors is not symmetrical. The host’s action carries information, and the best
strategy depends on modeling that action correctly.

\section{What we are building in this project}
We’ll treat the Monty Hall game as an engineered system:
\begin{itemize}
  \item First, we specify the game precisely using a \textbf{finite state machine} (Chapter 2).
  \item Next, we design a \textbf{data-collection FSM} that formalizes how we will simulate and record outcomes (Chapter 3).
  \item Then we implement the simulation as clean \textbf{object-oriented code} with strong \textbf{unit test coverage},
        and we present results with tables and plots (Chapter 4).
  \item Finally, we summarize what we learned (Chapter 5) and include the full ChatGPT-assisted workflow
        as a transcript with analysis (Chapter 6).
\end{itemize}

\section{Where we go next}
Chapter 2 introduces finite state machines and uses an FSM diagram of Monty Hall as our blueprint.
Once the game is written as states and transitions, it becomes much easier to implement, test, and measure.

