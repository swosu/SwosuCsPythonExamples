\chapter{Why Games Work (and Why Monty Hall Still Haunts Our Brains)}

\section{Why games are fun (and strangely useful)}
Human brains love three things: clear rules, uncertain outcomes, and fast feedback.
Games compress decision-making into a small space where you can try a strategy, see
what happens, and adapt. That loop makes games sticky---and it also makes them a
powerful teaching tool.

In programming, games are especially helpful because they naturally introduce:
state (what is true right now), transitions (what changes next), and rules (what is allowed).
That is exactly the mental model behind finite state machines, which we will use throughout
this project.

\section{Why game shows became a television superpower}
Game shows were built for television because they are:
(1) simple to understand mid-episode,
(2) cheap compared to scripted drama,
and (3) naturally friendly to advertising.

Early TV leaned heavily on sponsors, and the incentives sometimes went badly:
in the 1950s, several quiz shows were revealed to have been manipulated, which led
to backlash, investigations, and legal changes around televised contests.\footnote{PBS,
\textit{American Experience}, “The Aftermath of the Quiz Show Scandal.” \url{https://www.pbs.org/wgbh/americanexperience/features/quizshow-aftermath-quiz-show-scandal/}}

After that era, daytime game shows increasingly emphasized transparency, repeatable formats,
and brand-safe entertainment---a steady engine for ratings and advertising.

\section{Where the money comes from (prizes, sponsors, and the network)}
Most game shows are funded through the same basic machine:
advertising dollars (sold by the network) pay for production, while prizes can come from
either the show/network budget or from sponsors/advertisers as part of promotional deals.\footnote{Wrapbook,
“How to Account for Prize Money in Your Unscripted Game Show.” \url{https://www.wrapbook.com/blog/unscripted-game-show-prize-money}}

Important note: public sources rarely provide clean, audited breakdowns of \emph{profit per show}
(e.g., “this show grossed \$X and netted \$Y profit for the station”). Networks report finances at the
company or division level, and individual show economics are often proprietary. So we’ll focus on what
\emph{is} well supported publicly: the business model, the ratings impact, and the fact that high-audience
daytime programs translate into advertising revenue.

\section{Monty Hall: the man behind the doors}
Monty Hall (born Monte Halparin) was a Canadian-American broadcaster and producer best known as
the host and co-creator of \textit{Let’s Make a Deal}. The Television Academy notes that he appeared in
more than 4,700 episodes of the show.\footnote{Television Academy, “Monty Hall.” \url{https://www.televisionacademy.com/bios/monty-hall}}

\section{\textit{Let’s Make a Deal}: the marketplace of suspense}
\textit{Let’s Make a Deal} began as a daytime show in the 1960s and has had multiple runs and revivals
across networks and syndication.\footnote{Television Academy, “Let’s Make a Deal” show page (channels and original run). \url{https://interviews.televisionacademy.com/shows/lets-make-a-deal}}
Its signature magic trick is not math---it’s \emph{choice under uncertainty}.
Contestants (“traders”) bargain small sure things for unknown outcomes behind doors, curtains, or boxes,
sometimes winning big and sometimes getting a comedic “zonk.”

Historically, the show’s popularity was valuable enough to influence network daytime strategy.
For example, ABC’s 1968 annual report highlighted how its daytime schedule was strengthened by the
addition of \textit{Let’s Make a Deal}.\footnote{ABC, 1968 Annual Report (via WorldRadioHistory). \url{https://www.worldradiohistory.com/Archive-Station-Albums/Networks/ABC/ABC-American-Broadcasting-1968-Annual-Report.pdf}}
Modern commentary from the Strong National Museum of Play likewise emphasizes that the show’s
daytime audience mattered because it connected directly to advertising revenue.\footnote{Strong National Museum of Play,
“\textit{Let’s Make a Deal} Still a Big Deal” (Oct 27, 2023). \url{https://www.museumofplay.org/blog/lets-make-a-deal-still-a-big-deal/}}

\section{The modern revival}
A modern version of \textit{Let’s Make a Deal} has aired on CBS since 2009, hosted by Wayne Brady.\footnote{CBS show page,
“Let’s Make a Deal.” \url{https://www.cbs.com/shows/lets_make_a_deal/}}

\section{The three-door game (and the puzzle it inspired)}
The classroom-friendly version of the game is usually told like this:

\begin{enumerate}
  \item There are three doors: behind one is a prize, behind the other two are goats.
  \item You pick a door.
  \item The host (who knows where the prize is) opens one of the other doors to reveal a goat.
  \item You are offered a choice: stay with your original door or switch to the remaining closed door.
\end{enumerate}

This becomes the famous “Monty Hall problem.” Under standard assumptions, switching wins with
probability $2/3$, while staying wins with probability $1/3$.\footnote{Encyclopaedia Britannica, “Monty Hall problem.” \url{https://www.britannica.com/topic/Monty-Hall-problem}}

\section{Where we go next}
In Chapter 2, we’ll translate the game into a finite state machine: a precise map of states
(what stage the game is in) and transitions (what event causes the next step).
That FSM will become the blueprint for our simulation, testing strategy, and data collection.

