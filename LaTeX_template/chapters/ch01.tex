\chapter{Getting Started}
\label{chap:getting_started}

\section{Overview}
This chapter introduces the project, its goals, and how the rest of the book is organized.

You can briefly explain:
\begin{itemize}
    \item Who this material is for.
    \item What background (if any) is expected.
    \item How the chapters fit together.
\end{itemize}

\section{How to Use This Book}
\label{sec:how_to_use}
Here you can describe how readers should work through the material. For example:
\begin{itemize}
    \item Recommended order for reading chapters.
    \item Whether there are exercises, code, or external resources.
    \item Any conventions used (notation, terminology, etc.).
\end{itemize}

\section{Notation and Conventions}
\label{sec:notation}
Define:
\begin{itemize}
    \item Symbols you will use frequently.
    \item Formatting conventions (e.g., \texttt{code}, \emph{emphasis}).
\end{itemize}

% You can copy this file to create more chapters:
% cp chapters/ch01.tex chapters/ch02.tex
% ...and then update the \chapter{...} title and labels.

