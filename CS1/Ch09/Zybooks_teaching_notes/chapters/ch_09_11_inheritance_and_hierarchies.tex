\chapter{Inheritance and Class Hierarchies}

\section*{9.11 Inheritance and Reuse in Object-Oriented Programming}

Inheritance allows one class (the \textbf{child} or \textbf{subclass}) to reuse and extend the
behavior of another class (the \textbf{parent} or \textbf{base class}). This reduces
duplication and encourages code reuse—similar to how every car model inherits features
like wheels, engines, and doors from the general concept of a vehicle.

\subsection*{Example: A Vehicle Family Tree}

Imagine a \texttt{Vehicle} class that defines attributes and methods shared by all vehicles.
Specific types of vehicles—such as \texttt{Car}, \texttt{Truck}, and \texttt{Motorcycle}—
can inherit from it and add their own unique traits.

\begin{verbatim}
class Vehicle:
    def __init__(self, make, model):
        self.make = make
        self.model = model
    
    def start(self):
        print(f"{self.make} {self.model} engine started.")
    
    def stop(self):
        print(f"{self.make} {self.model} engine stopped.")

class Car(Vehicle):
    def __init__(self, make, model, doors):
        super().__init__(make, model)
        self.doors = doors
    
    def open_trunk(self):
        print(f"{self.make} {self.model}'s trunk is now open.")

class Truck(Vehicle):
    def __init__(self, make, model, towing_capacity):
        super().__init__(make, model)
        self.towing_capacity = towing_capacity
    
    def tow(self, weight):
        print(f"Towing {weight} lbs out of {self.towing_capacity} lbs capacity.")
\end{verbatim}

\subsection*{Discussion}

Each subclass inherits the \texttt{start()} and \texttt{stop()} methods from
\texttt{Vehicle}. They don’t need to be rewritten unless behavior must change.

\begin{verbatim}
my_car = Car("Ford", "Focus", 4)
my_truck = Truck("Nissan", "Titan", 9000)

my_car.start()
my_car.open_trunk()
my_truck.start()
my_truck.tow(6000)
\end{verbatim}

Output:
\begin{verbatim}
Ford Focus engine started.
Ford Focus's trunk is now open.
Nissan Titan engine started.
Towing 6000 lbs out of 9000 lbs capacity.
\end{verbatim}

\subsection*{Using super()}

The \texttt{super()} function lets the subclass call the constructor (or other methods)
of its parent. Without it, the base attributes would not be initialized.

\subsection*{Method Overriding}

A subclass can redefine (override) methods from its parent to create specialized behavior.

\begin{verbatim}
class ElectricCar(Car):
    def start(self):
        print(f"{self.make} {self.model} powers on silently. ⚡")
\end{verbatim}

\begin{verbatim}
tesla = ElectricCar("Tesla", "Model 3", 4)
tesla.start()
\end{verbatim}

Output:
\begin{verbatim}
Tesla Model 3 powers on silently. ⚡
\end{verbatim}

\subsection*{Class Hierarchy Diagram}

\begin{center}
\begin{tikzpicture}[
    node distance=2cm,
    every node/.style={draw, fill=blue!10, rounded corners, minimum width=3cm, align=center}
]
\node (vehicle) {Vehicle};
\node (car) [below left=of vehicle] {Car};
\node (truck) [below right=of vehicle] {Truck};
\node (electriccar) [below=of car] {ElectricCar};

\draw[->, thick] (car) -- (vehicle);
\draw[->, thick] (truck) -- (vehicle);
\draw[->, thick] (electriccar) -- (car);
\end{tikzpicture}
\end{center}

\subsection*{Practice Activity 9.11.1 – Create Your Own Hierarchy}

Create a base class called \texttt{Animal} with attributes \texttt{name} and \texttt{sound}.
Then create subclasses \texttt{Dog} and \texttt{Cat} that each override a \texttt{make\_sound()}
method to print their own noises.

\begin{verbatim}
class Animal:
    def __init__(self, name):
        self.name = name
    def make_sound(self):
        print("Some generic sound")

class Dog(Animal):
    def make_sound(self):
        print(f"{self.name} says Woof!")

class Cat(Animal):
    def make_sound(self):
        print(f"{self.name} says Meow!")
\end{verbatim}

Reflection:
\begin{itemize}
    \item What benefits does inheritance provide for large codebases?
    \item When might it be better to avoid inheritance and use composition instead?
    \item Can a class inherit from more than one parent in Python?
\end{itemize}

\subsection*{Summary}

\begin{itemize}
    \item Inheritance enables code reuse and logical hierarchies.
    \item \texttt{super()} connects child classes to parent initialization.
    \item Method overriding allows specialization.
    \item Real-world analogy: All vehicles share traits (wheels, engines), but each subclass
    adds its own personality.
\end{itemize}

\begin{center}
\textit{“Every subclass stands on the shoulders of its superclass.”}
\end{center}

