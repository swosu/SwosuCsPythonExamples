\documentclass[12pt]{article}
\usepackage[a4paper, margin=1in]{geometry}
\usepackage{fancyhdr}
\usepackage{titlesec}
\usepackage{amsmath, amssymb}
\usepackage{hyperref}
\usepackage{xcolor}
\usepackage{listings}
\usepackage{tcolorbox}
\usepackage{enumitem}

% --------------------------------------------------------------
% Page style
% --------------------------------------------------------------
\pagestyle{fancy}
\fancyhf{}
\rhead{Chapter 11.2: Finding Modules}
\lhead{SWOSU Computer Science}
\rfoot{\thepage}

% --------------------------------------------------------------
% Listings setup
% --------------------------------------------------------------
\definecolor{codegray}{gray}{0.95}
\lstset{
  backgroundcolor=\color{codegray},
  basicstyle=\ttfamily\small,
  frame=single,
  breaklines=true,
  showstringspaces=false,
  keywordstyle=\color{blue},
  commentstyle=\color{gray},
  stringstyle=\color{purple},
  tabsize=4
}

% --------------------------------------------------------------
\titleformat{\section}
  {\normalfont\Large\bfseries\color{blue!60!black}}
  {\thesection}{1em}{}
\titleformat{\subsection}
  {\normalfont\large\bfseries\color{blue!40!black}}
  {\thesubsection}{1em}{}

\newtcolorbox{activitybox}[2][]{colback=blue!5!white,colframe=blue!75!black,title={#2},fonttitle=\bfseries,#1}
\newtcolorbox{conceptbox}[2][]{colback=green!5!white,colframe=green!50!black,title={#2},fonttitle=\bfseries,#1}
\newtcolorbox{reflectionbox}[2][]{colback=orange!5!white,colframe=orange!60!black,title={#2},fonttitle=\bfseries,#1}

% --------------------------------------------------------------
\begin{document}

\begin{center}
  \vspace*{1cm}
  {\Huge \textbf{Chapter 11.2 – Finding Modules}}\\[0.5cm]
  {\Large Teaching Notes and Student Activities}\\[1cm]
  \rule{\textwidth}{0.4pt}\\[1cm]
\end{center}

% ==============================================================
\section{Learning Objectives}
\begin{itemize}
  \item Explain how Python searches for modules when executing an \texttt{import} statement.
  \item Identify the directories contained in the \texttt{sys.path} list.
  \item Understand the use of the environment variable \texttt{PYTHONPATH}.
  \item Avoid naming conflicts between custom and built-in modules.
\end{itemize}

% ==============================================================
\section{How Python Finds Modules}

When an \texttt{import} statement is executed, Python begins searching for a file that matches the module name. The interpreter checks the following in order:

\begin{enumerate}
  \item Built-in modules (e.g., \texttt{sys}, \texttt{math}, \texttt{time}).  
  \item Directories listed in the variable \texttt{sys.path}.
\end{enumerate}

The \texttt{sys.path} variable is a list of directories that Python searches when locating modules.  
It typically contains:
\begin{enumerate}[label=\arabic*.]
  \item The directory of the executing script.  
  \item Any directories specified by the environment variable \texttt{PYTHONPATH}.  
  \item The directory where Python itself is installed.
\end{enumerate}

\begin{conceptbox}{Example – Inspecting \texttt{sys.path}}
\begin{lstlisting}[language=Python]
import sys
for path in sys.path:
    print(path)
\end{lstlisting}
Running this script displays all directories Python searches for modules.  
This can help diagnose “module not found” errors.
\end{conceptbox}

\subsection*{Built-in Modules}
A \textbf{built-in module} comes pre-installed with Python.  
Examples include:
\begin{itemize}
  \item \texttt{math} – mathematical functions
  \item \texttt{sys} – access to system-level variables
  \item \texttt{time} – time and date utilities
\end{itemize}

If a built-in module with the same name exists, it will always be loaded first.  
For example, naming your own file \texttt{math.py} could break imports that expect the standard library version.

\begin{reflectionbox}{Naming Rule of Thumb}
Avoid naming your own modules after existing Python libraries like \texttt{os.py}, \texttt{json.py}, or \texttt{math.py}.  
Python may accidentally load your local file instead of the standard module!
\end{reflectionbox}

% ==============================================================
\section{Environment Variables and PYTHONPATH}

The environment variable \texttt{PYTHONPATH} can be used to tell Python where to look for additional modules.  

On Windows:
\begin{lstlisting}
set PYTHONPATH="C:\myModules;C:\other"
\end{lstlisting}

On Linux or macOS:
\begin{lstlisting}[language=bash]
export PYTHONPATH="/home/student/myModules:/opt/extraModules"
\end{lstlisting}

This allows developers to maintain separate module directories for projects or labs.

% ==============================================================
\section{Student Activity – Where Does Python Look?}

\begin{activitybox}{Participation Activity 11.2.1 – Finding Modules}
\begin{enumerate}
  \item Create a new file called \texttt{hello\_mod.py} that contains:
  \begin{lstlisting}[language=Python]
def say_hello():
    print("Hello from your custom module!")
  \end{lstlisting}

  \item Open a new Python shell in the same directory and run:
  \begin{lstlisting}[language=Python]
import hello_mod
hello_mod.say_hello()
  \end{lstlisting}

  \item Now move the file to another directory. Try importing it again.
  \begin{itemize}
    \item What happens when the file isn’t in the same folder?
    \item Can you use \texttt{PYTHONPATH} or modify \texttt{sys.path} to fix it?
  \end{itemize}

  \item Try running:
  \begin{lstlisting}[language=Python]
import sys
print(sys.path)
  \end{lstlisting}
  and observe which directories Python searches.

  \item Experiment:  
  Create a file called \texttt{math.py} with this content:
  \begin{lstlisting}[language=Python]
def bad_idea():
    print("Oops, I replaced the real math module!")
  \end{lstlisting}

  Then run:
  \begin{lstlisting}[language=Python]
import math
print(dir(math))
  \end{lstlisting}

  Discuss why this is a problem!
\end{enumerate}
\end{activitybox}

% ==============================================================
\section{Quick Quiz}
\begin{enumerate}
  \item When an import statement executes, Python immediately checks the current directory for a matching file. \textbf{(True)}  
  \item The environment variable \texttt{PYTHONPATH} can specify extra module directories. \textbf{(True)}  
  \item \texttt{math.py} is a good name for a new module. \textbf{(False)}  
\end{enumerate}

% ==============================================================
\begin{reflectionbox}{Why This Matters}
Understanding how Python locates modules helps students:
\begin{itemize}
  \item Debug ``ModuleNotFoundError'' problems.
  \item Manage multiple project environments cleanly.
  \item Prevent naming collisions with standard libraries.
\end{itemize}
\end{reflectionbox}

% ==============================================================
\end{document}

