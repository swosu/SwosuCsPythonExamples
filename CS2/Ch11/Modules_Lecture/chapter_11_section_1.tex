\documentclass[12pt]{article}
\usepackage[a4paper, margin=1in]{geometry}
\usepackage{fancyhdr}
\usepackage{titlesec}
\usepackage{amsmath, amssymb}
\usepackage{hyperref}
\usepackage{xcolor}
\usepackage{listings}
\usepackage{tcolorbox}
\usepackage{enumitem}

% --------------------------------------------------------------
% Page style
% --------------------------------------------------------------
\pagestyle{fancy}
\fancyhf{}
\rhead{Chapter 11.1: Modules (Teacher Edition)}
\lhead{SWOSU Computer Science}
\rfoot{\thepage}
\setlength{\headheight}{15pt}

% --------------------------------------------------------------
% Listings setup
% --------------------------------------------------------------
\definecolor{codegray}{gray}{0.95}
\lstset{
  backgroundcolor=\color{codegray},
  basicstyle=\ttfamily\small,
  frame=single,
  breaklines=true,
  showstringspaces=false,
  keywordstyle=\color{blue},
  commentstyle=\color{gray},
  stringstyle=\color{purple},
  tabsize=4
}

% --------------------------------------------------------------
% Box environments
% --------------------------------------------------------------
\newtcolorbox{activitybox}[2][]{colback=blue!5!white,colframe=blue!75!black,
  title={#2},fonttitle=\bfseries,#1,breakable}

\newtcolorbox{conceptbox}[2][]{colback=green!5!white,colframe=green!50!black,
  title={#2},fonttitle=\bfseries,#1,breakable}

\newtcolorbox{reflectionbox}[2][]{colback=orange!5!white,colframe=orange!60!black,
  title={#2},fonttitle=\bfseries,#1,breakable}

\newtcolorbox{teacherbox}[2][]{colback=yellow!10!white,colframe=yellow!60!black,
  title={💡 Instructor Notes – #2},fonttitle=\bfseries\itshape,#1,breakable}

% --------------------------------------------------------------
\begin{document}

\begin{center}
  \vspace*{1cm}
  {\Huge \textbf{Chapter 11.1 – Python Modules}}\\[0.5cm]
  {\Large Teacher Edition with Full Notes and Solutions}\\[1cm]
  \rule{\textwidth}{0.4pt}\\[1cm]
\end{center}

% ==============================================================
\section{Learning Objectives}
\begin{itemize}
  \item Define what a \textbf{module} is in Python.
  \item Explain how to \textbf{import} and use modules.
  \item Demonstrate the \textbf{importing process}.
  \item Distinguish between \textbf{scripts} and \textbf{modules}.
  \item Practice writing and importing custom modules.
\end{itemize}

% ==============================================================
\section{Core Concepts and Vocabulary}

\begin{conceptbox}{Essential Terms}
\begin{itemize}
  \item \textbf{Script}: A file ending in \texttt{.py} that is executed directly to perform a task.
  \item \textbf{Module}: A Python file that defines reusable code (functions, variables, or classes) that can be imported by another script.
  \item \textbf{Import}: The command that brings a module’s definitions into another program.
  \item \textbf{Namespace}: A “labeled space” that keeps identifiers from one module separate from another.
  \item \textbf{Dependency}: A module required by another program to run.
  \item \textbf{sys.modules}: Python’s internal dictionary of all loaded modules.
\end{itemize}
\end{conceptbox}

\begin{teacherbox}{Vocabulary Notes}
Point out that modules are *namespaces*.  
The idea of “where does a name live?” will later connect to classes, packages, and scopes.  
Stress the value of modularity and how this supports collaboration across files.
\end{teacherbox}

% ==============================================================
\section{Introducing Modules}

When you use Python interactively, all variables and functions vanish when you close the interpreter.  
To reuse them, you save the code in a \textbf{script} file.  
But when multiple programs need the same code, it’s better to put those definitions in a \textbf{module}.

\begin{activitybox}{Participation Activity – What Is a Module?}
\textbf{Step 1: Create a file \texttt{math\_utils.py}}
\begin{lstlisting}[language=Python]
def double(x):
    return 2 * x

def triple(x):
    return 3 * x
\end{lstlisting}

\textbf{Step 2: Create a second file \texttt{test\_math.py}}
\begin{lstlisting}[language=Python]
import math_utils

print(math_utils.double(5))
print(math_utils.triple(7))
\end{lstlisting}

\textbf{Expected Output:}
\begin{lstlisting}
10
21
\end{lstlisting}

\textbf{Discussion:}  
Why is this approach better than copying the same functions into every script?
\end{activitybox}

\begin{teacherbox}{Suggested Solution}
Encourage students to say:
\begin{itemize}
  \item “It’s reusable.”  
  \item “If I fix a bug in one place, all scripts benefit.”  
  \item “It helps organize code by responsibility.”  
\end{itemize}
You can demo editing \texttt{math\_utils.py} once and rerunning multiple importing scripts to show consistency.
\end{teacherbox}

% ==============================================================
\section{How Import Works}

When Python sees \texttt{import mymodule}, it executes the following:

\begin{enumerate}
  \item Checks whether \texttt{mymodule} is already loaded in \texttt{sys.modules}.  
  \item If not, creates a \textbf{module object}.  
  \item Executes all the code inside \texttt{mymodule.py}.  
  \item Adds the module object to \texttt{sys.modules}.  
  \item Binds the module’s name into the importer’s namespace.
\end{enumerate}

\begin{activitybox}{Demonstration – The Importing Process}
\textbf{File: \texttt{tools.py}}
\begin{lstlisting}[language=Python]
print("tools.py is running!")

def greet(name):
    print(f"Hello, {name}!")
\end{lstlisting}

\textbf{File: \texttt{main.py}}
\begin{lstlisting}[language=Python]
import tools

print("main.py continues")
tools.greet("Jeremy")
\end{lstlisting}

\textbf{Expected Output:}
\begin{lstlisting}
tools.py is running!
main.py continues
Hello, Jeremy!
\end{lstlisting}
\end{activitybox}

\begin{teacherbox}{Explanation and Solution Discussion}
Explain that when a module is first imported, Python actually *runs* the file.  
Subsequent imports skip execution because the module now exists in \texttt{sys.modules}.  
Have students verify this by adding another \texttt{import tools} line—it won’t print “tools.py is running!” twice.
\end{teacherbox}

% ==============================================================
\section{Using Imported Modules}

Once a module is imported, you can access its contents with dot notation:
\begin{lstlisting}[language=Python]
import math_utils
print(math_utils.double(10))   # 20
\end{lstlisting}

Temporary overwriting is possible:
\begin{lstlisting}[language=Python]
math_utils.double = lambda x: x * 10
print(math_utils.double(2))    # 20
\end{lstlisting}

\begin{teacherbox}{Instructor Notes}
Explain that reassignment like this affects only the in-memory object, not the source file.  
Restarting Python resets everything.
\end{teacherbox}

% ==============================================================
\section{Guided Practice – Importing Multiple Modules}

\begin{activitybox}{Try This!}
\textbf{File: \texttt{weather.py}}
\begin{lstlisting}[language=Python]
def forecast():
    return "Sunny and warm"
\end{lstlisting}

\textbf{File: \texttt{mood.py}}
\begin{lstlisting}[language=Python]
def today():
    return "Feeling great!"
\end{lstlisting}

\textbf{File: \texttt{main.py}}
\begin{lstlisting}[language=Python]
import weather
import mood

print(weather.forecast(), "and", mood.today())
\end{lstlisting}

\textbf{Expected Output:}
\begin{lstlisting}
Sunny and warm and Feeling great!
\end{lstlisting}

\textbf{Follow-up Tasks:}
\begin{enumerate}
  \item Add a \texttt{print()} statement at the top of each module to announce when it’s imported.
  \item Observe which import prints first.
\end{enumerate}
\end{activitybox}

\begin{teacherbox}{Observation Discussion}
If students import in this order:
\begin{lstlisting}
import weather
import mood
\end{lstlisting}
then \texttt{weather} runs first.  
Reversing them flips the print order—proof that Python executes modules as it imports them.  
This becomes relevant later for circular imports.
\end{teacherbox}

% ==============================================================
\section{Student Challenges with Solutions}

\subsection*{Challenge 1 – Module Check}
\textbf{Create a module \texttt{stats\_tools.py}:}
\begin{lstlisting}[language=Python]
def mean(nums):
    return sum(nums) / len(nums)

def median(nums):
    nums.sort()
    mid = len(nums) // 2
    return nums[mid]
\end{lstlisting}

\textbf{Then use it in \texttt{analyze.py}:}
\begin{lstlisting}[language=Python]
import stats_tools

data = [4, 7, 2, 9, 6]
print(stats_tools.mean(data))
print(stats_tools.median(data))
\end{lstlisting}

\textbf{Sample Output:}
\begin{lstlisting}
5.6
6
\end{lstlisting}

\begin{teacherbox}{Teaching Focus}
Discuss how sorting inside \texttt{median()} mutates the list; mention how you might copy it first.  
Encourage testing with both even- and odd-length lists to extend thinking.
\end{teacherbox}

% --------------------------------------------------------------
\subsection*{Challenge 2 – Reimport Experiment}
\begin{lstlisting}[language=Python]
import stats_tools
import stats_tools
\end{lstlisting}

\textbf{Expected Behavior:} The module loads only once; Python reuses the object in \texttt{sys.modules}.  
You can verify by printing:
\begin{lstlisting}[language=Python]
print(id(stats_tools))
\end{lstlisting}
twice and seeing the same memory address.

\begin{teacherbox}{Discussion}
Stress that this caching behavior is efficient and prevents redundant initialization.  
If you *do* need to reload (e.g., during development), you can use:
\begin{lstlisting}[language=Python]
import importlib
importlib.reload(stats_tools)
\end{lstlisting}
\end{teacherbox}

% --------------------------------------------------------------
\subsection*{Challenge 3 – Namespace Play}
\begin{lstlisting}[language=Python]
# cat.py
def speak():
    return "Meow!"

# dog.py
def speak():
    return "Woof!"

# zoo.py
import cat, dog
print(cat.speak(), dog.speak())
\end{lstlisting}

\textbf{Output:}
\begin{lstlisting}
Meow! Woof!
\end{lstlisting}

\begin{teacherbox}{Solution Key}
Use this as a demonstration that each module provides its own namespace—no collisions even though both define \texttt{speak()}.  
Introduce the idea that the dot operator explicitly references the module’s scope.
\end{teacherbox}

% ==============================================================
\section{Reflection – Why Modules Matter}

\begin{reflectionbox}{Key Takeaways}
\begin{itemize}
  \item Modules make code reusable and easier to maintain.
  \item They support collaboration—different people can maintain different files.
  \item Understanding module imports is essential for debugging and larger applications.
\end{itemize}
\end{reflectionbox}

\begin{teacherbox}{Discussion Prompt}
Ask students: “If you were designing a large game or app, how would you break it into modules?”  
Encourage them to name example files (e.g., \texttt{player.py}, \texttt{inventory.py}, \texttt{battle.py}) to see real modular thinking.
\end{teacherbox}

% ==============================================================
\section{Extension Prompts with Solutions}

\begin{itemize}
  \item \textbf{Task:} Create a module with a function \texttt{hello(name)} and import it using  
  \texttt{from module import hello}.  
  \textbf{Solution:}  
  This imports only that function and allows calling \texttt{hello("World")} directly.
  
  \item \textbf{Task:} Investigate \texttt{importlib.reload()}.  
  \textbf{Solution:}  
  \texttt{importlib.reload()} re-runs a module’s code, which is useful when editing modules interactively.

  \item \textbf{Task:} Try putting a module in a subfolder and import using dot notation.  
  \textbf{Solution:}  
  Example:  
  \texttt{from utilities.math\_tools import mean}.  
  Discuss package imports next section.
\end{itemize}

% ==============================================================
\end{document}

