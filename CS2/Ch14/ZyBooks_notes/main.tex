\documentclass[12pt]{book}

% ==========================
% PACKAGES
% ==========================
\usepackage[a4paper, margin=1in]{geometry}
\usepackage{fancyhdr}
\usepackage{graphicx}
\usepackage{amsmath, amssymb}
\usepackage{hyperref}
\usepackage{xcolor}
\usepackage{enumitem}
\usepackage{titlesec}
\usepackage{setspace}
\usepackage{listings}
\usepackage{caption}

% ==========================
% HEADER / FOOTER
% ==========================
\pagestyle{fancy}
\fancyhf{}
\fancyhead[L]{CS2 Recursion Workbook}
\fancyhead[R]{\leftmark}
\fancyfoot[C]{\thepage}

% ==========================
% CODE STYLE
% ==========================
\lstset{
  basicstyle=\ttfamily\footnotesize,
  backgroundcolor=\color{gray!10},
  frame=single,
  breaklines=true,
  postbreak=\mbox{\textcolor{red}{$\hookrightarrow$}\space},
  keywordstyle=\color{blue},
  commentstyle=\color{gray},
  stringstyle=\color{orange}
}

% ==========================
% SECTION STYLE
% ==========================
\titleformat{\chapter}[display]
  {\normalfont\bfseries\Huge}
  {\filleft\thechapter}
  {1ex}
  {\titlerule\vspace{1ex}\filright}

% ==========================
% DOCUMENT START
% ==========================
\begin{document}

\frontmatter
\title{Recursion Workbook for CS 2}
\author{Jeremy Evert}
\date{\today}
\maketitle

\tableofcontents
\newpage

\mainmatter

\chapter{Introduction}
\label{chap:intro}

\section{Overview}
This document presents a solution to the current assignment.
It shows one way the problem could be done and sets some expectations.
You don't have to match it exactly, but it provides a starting point.


% future chapters:
% \chapter{Section 6.1 — The Basics of Counting}

\section{Introduction}

Suppose that a password on a computer system consists of six, seven, or eight characters. 
Each of these characters must be a digit or a letter of the alphabet, and each password must contain at least one digit. 
How many such passwords are there?

Questions like this form the beating heart of discrete mathematics:  
\emph{How many ways can something happen?}  
Counting allows us to measure complexity, probability, and possibility.  
In computer science, every algorithm that loops, branches, or searches is secretly counting something.

\section{Basic Counting Principles}

Two fundamental rules guide our reasoning:

\begin{description}
  \item[\textbf{The Sum Rule}]  
  If a task can be done in $n_1$ ways \emph{or} $n_2$ ways (but not both), then it can be done in $n_1 + n_2$ ways.

  \item[\textbf{The Product Rule}]  
  If a task can be broken into two independent subtasks—first done in $n_1$ ways, then in $n_2$ ways—then the whole procedure can be done in $n_1 \times n_2$ ways.
\end{description}

\section{Worked Examples}

\begin{example}[Assigning Offices]
A new company with two employees, Sanchez and Patel, rents a floor with 12 offices.  
How many ways are there to assign different offices to these two employees?

\textbf{Solution.}  
12 choices for Sanchez, then 11 remaining for Patel.  
By the product rule: $12 \times 11 = 132$ possibilities.
\end{example}

\begin{example}[Labeling Auditorium Chairs]
Each seat is labeled with one uppercase English letter followed by a positive integer not exceeding 100.  
How many unique chair labels exist?

\textbf{Solution.}  
$26$ possible letters $\times$ $100$ integers $= 2600$ labels.
\end{example}

\begin{example}[Counting Ports in a Data Center]
There are 32 computers, each with 24 ports.  
How many total ports exist?

\textbf{Solution.}  
$32 \times 24 = 768$ ports.
\end{example}

\begin{example}[Challenging A: License Plate Combinations]
A region issues license plates with 3 uppercase letters followed by 3 digits (e.g., ABC123).  
How many distinct plates are possible if repetition is allowed?

\textbf{Solution.}  
$26^3 \times 10^3 = 175{,}760{,}000$ possible plates.

If letters and digits cannot repeat, the count becomes  
$26 \times 25 \times 24 \times 10 \times 9 \times 8 = 112{,}320{,}000$.
\end{example}

\begin{example}[Challenging B: Passwords with a Digit Requirement]
A password must be 6, 7, or 8 characters long, each either a letter or digit, 
and must contain at least one digit.  
How many possible passwords exist?

\textbf{Solution.}  
Let $A$ be all possible strings of the given length (letters + digits),  
and $B$ be those with only letters.  

Then valid passwords $= A - B$.

\begin{align*}
|A_6| &= 36^6, & |B_6| &= 26^6 \\
|A_7| &= 36^7, & |B_7| &= 26^7 \\
|A_8| &= 36^8, & |B_8| &= 26^8 \\
\text{Total} &= (36^6 - 26^6) + (36^7 - 26^7) + (36^8 - 26^8)
\end{align*}
\end{example}

\section{Python Demonstrations}

Below is Python code that mirrors the reasoning in these examples.  
Each section prints both the symbolic reasoning and the computed value.  
Students are encouraged to modify the numbers and observe how the results change.

\begin{lstlisting}[language=Python, caption={Demonstrating the Product and Sum Rules in Python}]
"""
COMSC 2043 - Counting Demonstrations
Author: Jeremy Evert
This script demonstrates the basic principles of counting using Python.
"""

from math import comb, perm

def example_offices():
    n1, n2 = 12, 11
    total = n1 * n2
    print(f"Example 1: Assigning offices\n12 * 11 = {total}\n")

def example_chairs():
    total = 26 * 100
    print(f"Example 2: Chair labeling\n26 * 100 = {total}\n")

def example_ports():
    total = 32 * 24
    print(f"Example 3: Data center ports\n32 * 24 = {total}\n")

def example_license_plates():
    allow_repeat = 26**3 * 10**3
    no_repeat = (26*25*24) * (10*9*8)
    print("Example 4: License plates")
    print(f"  With repetition: {allow_repeat:,}")
    print(f"  Without repetition: {no_repeat:,}\n")

def example_passwords():
    # Helper function: A^n - B^n
    def count_valid(length):
        return 36**length - 26**length

    total = sum(count_valid(n) for n in (6, 7, 8))
    print("Example 5: Passwords with at least one digit")
    print(f"  Total valid passwords: {total:,}\n")

def main():
    print("=== Counting Demonstrations ===\n")
    example_offices()
    example_chairs()
    example_ports()
    example_license_plates()
    example_passwords()
    print("Done!")

if __name__ == "__main__":
    main()
\end{lstlisting}

\section*{Discussion and Reflection}

Each of these problems could be solved by intuition, but writing code forces precision.  
Python doesn’t “believe” in the product rule—it performs it.  
This helps students see that the abstract logic of counting translates directly into computation.

Encourage students to:
\begin{itemize}
  \item Change numbers and verify the rules still hold.
  \item Add print statements to trace intermediate steps.
  \item Reflect on where independence between tasks exists—and where it doesn’t.
\end{itemize}

\section*{Next Steps}

In the next section we will apply these rules to cases where choices overlap or restrict one another—  
leading to the powerful and deceptively simple \textbf{Pigeonhole Principle}.

% \chapter{Recursive Algorithm: Search}

\section*{From Repetition to Strategy}
Loops and recursion both repeat work—but recursion lets us do it *intelligently*.  
Instead of plowing through every item one by one, a recursive algorithm can **divide and conquer**.

A recursive algorithm breaks a problem into smaller, self-similar versions of itself until the smallest case (the *base case*) can be solved directly.

\section{A Familiar Analogy: The Guessing Game}
Imagine your friend thinks of a number between 0 and 100.  
Each time you guess, your friend says "higher" or "lower."  

If you always guess halfway between the possible range, you’ll find the number in about $\log_2(100) \approx 7$ guesses.

That’s **binary search**—recursion in action.

\begin{lstlisting}[language=Python, caption={Recursive binary search for a number}]
def binary_search(low, high, target):
    if low > high:
        print("Not found!")
        return
    mid = (low + high) // 2
    print(f"Searching {low}..{high} (mid={mid})")
    if mid == target:
        print("Found it!")
    elif target < mid:
        binary_search(low, mid - 1, target)
    else:
        binary_search(mid + 1, high, target)

binary_search(0, 100, 32)
\end{lstlisting}

\noindent
This algorithm is recursive because it calls itself on smaller subranges each time.  
When the range collapses (\texttt{low > high}), the function ends.

\begin{quote}
Every recursive algorithm is a conversation with smaller versions of itself.
\end{quote}

\section{Recursive Search in a Sorted List}
Now let’s find a name in a list that’s alphabetically sorted.  
This is a textual version of binary search.

\begin{lstlisting}[language=Python, caption={Recursive search in a sorted list}]
def find(lst, item, low, high):
    if low > high:
        return -1  # Not found
    mid = (low + high) // 2
    if lst[mid] == item:
        return mid
    elif item < lst[mid]:
        return find(lst, item, low, mid - 1)
    else:
        return find(lst, item, mid + 1, high)

names = ["Adams, Mary", "Carver, Michael", "Domer, Hugo",
         "Fredericks, Carlo", "Liu, Jie"]

person = input("Enter last, first: ")
pos = find(names, person, 0, len(names) - 1)
if pos >= 0:
    print(f"Found {person} at index {pos}")
else:
    print("Not found.")
\end{lstlisting}

\noindent
Notice how the search range shrinks by half each time.  
Recursive binary search is powerful because it eliminates half the data at every step.

\section{Thinking Recursively}
A recursive algorithm always has these three traits:
\begin{enumerate}
    \item \textbf{A clear goal:} What are we trying to find?
    \item \textbf{A base case:} When do we stop searching?
    \item \textbf{A recursive case:} How do we break the problem into smaller ones?
\end{enumerate}

You’ll see this pattern everywhere—from sorting algorithms (quicksort, mergesort)  
to tree traversal and directory searches.

\section{Visualizing the Divide-and-Conquer Pattern}
Each recursive call creates a branch in the "decision tree."  
Here’s the mental image:

\begin{verbatim}
Search [0..100]
 |- Guess 50 -> too high -> Search [0..49]
 |   |- Guess 25 -> too low -> Search [26..49]
 |   |   |- Guess 37 -> too high -> Search [26..36]
 |   |   |   |- Guess 31 -> too low -> Search [32..36]
 |   |   |   |   |- Guess 34 -> Found!
\end{verbatim}

\noindent
Each level of recursion focuses on a smaller search space.  
The call stack keeps track of where you came from.

\section{Tracing the Call Stack}
When recursion runs, Python keeps track of every unfinished call in the *call stack*.  
Think of it as a trail of sticky notes—each one says, “Come back to me when you’re done.”

\begin{verbatim}
binary_search(0, 100, 32)
  ├── binary_search(0, 49, 32)
  ├── binary_search(25, 49, 32)
  ├── binary_search(32, 36, 32)
  └── Found!
\end{verbatim}

Each time the recursive call returns, Python pops one frame off the stack.  
When the base case is reached, the stack empties gracefully.  
If you forget your base case—well, Python keeps stacking until it crashes! 😅

\section{Try It Yourself}
Write your own recursive search for these problems:
\begin{enumerate}
    \item Search for a letter in a string, returning its index.
    \item Search for the smallest number in a sorted list (without using \texttt{min()}).
    \item Modify \texttt{find()} to print the number of recursive calls made.
\end{enumerate}

\section*{Challenge: Recursive Word Finder}
For extra fun, try writing a recursive function that searches through nested lists:
\begin{lstlisting}[language=Python, caption={Recursive word finder challenge}]
def find_word(nested_list, word):
    for item in nested_list:
        if isinstance(item, list):
            if find_word(item, word):
                return True
        elif item == word:
            return True
    return False

data = [["dog", ["cat", "fish"]], ["hamster", ["parrot", "snake"]]]
print(find_word(data, "snake"))  # True
print(find_word(data, "whale"))  # False
\end{lstlisting}

\noindent
This shows recursion applied to hierarchical data—a natural fit when structures contain smaller versions of themselves.

\section*{Closing Thought}
Recursive search is not just faster—it’s smarter.  
It doesn’t look at everything; it looks *strategically*.  
That’s what separates **repetition** from **recursion**—and **recursion** from **algorithmic thinking**.

\begin{center}
\textit{Next: Debugging recursive calls — the art of seeing the invisible stack.}
\end{center}


% \include{chapters\lab_fibonacci}
% \include{chapters\lab_permutations}

\backmatter
\chapter*{Notes}
Use this space for your own discoveries and recursive experiments.

\end{document}

