\chapter{Recursive Algorithm: Search}

\section*{From Repetition to Strategy}
Loops and recursion both repeat work—but recursion lets us do it *intelligently*.  
Instead of plowing through every item one by one, a recursive algorithm can **divide and conquer**.

A recursive algorithm breaks a problem into smaller, self-similar versions of itself until the smallest case (the *base case*) can be solved directly.

\section{A Familiar Analogy: The Guessing Game}
Imagine your friend thinks of a number between 0 and 100.  
Each time you guess, your friend says "higher" or "lower."  

If you always guess halfway between the possible range, you’ll find the number in about $\log_2(100) \approx 7$ guesses.

That’s **binary search**—recursion in action.

\begin{lstlisting}[language=Python, caption={Recursive binary search for a number}]
def binary_search(low, high, target):
    if low > high:
        print("Not found!")
        return
    mid = (low + high) // 2
    print(f"Searching {low}..{high} (mid={mid})")
    if mid == target:
        print("Found it!")
    elif target < mid:
        binary_search(low, mid - 1, target)
    else:
        binary_search(mid + 1, high, target)

binary_search(0, 100, 32)
\end{lstlisting}

\noindent
This algorithm is recursive because it calls itself on smaller subranges each time.  
When the range collapses (\texttt{low > high}), the function ends.

\begin{quote}
Every recursive algorithm is a conversation with smaller versions of itself.
\end{quote}

\section{Recursive Search in a Sorted List}
Now let’s find a name in a list that’s alphabetically sorted.  
This is a textual version of binary search.

\begin{lstlisting}[language=Python, caption={Recursive search in a sorted list}]
def find(lst, item, low, high):
    if low > high:
        return -1  # Not found
    mid = (low + high) // 2
    if lst[mid] == item:
        return mid
    elif item < lst[mid]:
        return find(lst, item, low, mid - 1)
    else:
        return find(lst, item, mid + 1, high)

names = ["Adams, Mary", "Carver, Michael", "Domer, Hugo",
         "Fredericks, Carlo", "Liu, Jie"]

person = input("Enter last, first: ")
pos = find(names, person, 0, len(names) - 1)
if pos >= 0:
    print(f"Found {person} at index {pos}")
else:
    print("Not found.")
\end{lstlisting}

\noindent
Notice how the search range shrinks by half each time.  
Recursive binary search is powerful because it eliminates half the data at every step.

\section{Thinking Recursively}
A recursive algorithm always has these three traits:
\begin{enumerate}
    \item \textbf{A clear goal:} What are we trying to find?
    \item \textbf{A base case:} When do we stop searching?
    \item \textbf{A recursive case:} How do we break the problem into smaller ones?
\end{enumerate}

You’ll see this pattern everywhere—from sorting algorithms (quicksort, mergesort)  
to tree traversal and directory searches.

\section{Visualizing the Divide-and-Conquer Pattern}
Each recursive call creates a branch in the "decision tree."  
Here’s the mental image:

\begin{verbatim}
Search [0..100]
 |- Guess 50 -> too high -> Search [0..49]
 |   |- Guess 25 -> too low -> Search [26..49]
 |   |   |- Guess 37 -> too high -> Search [26..36]
 |   |   |   |- Guess 31 -> too low -> Search [32..36]
 |   |   |   |   |- Guess 34 -> Found!
\end{verbatim}

\noindent
Each level of recursion focuses on a smaller search space.  
The call stack keeps track of where you came from.

\section{Tracing the Call Stack}
When recursion runs, Python keeps track of every unfinished call in the *call stack*.  
Think of it as a trail of sticky notes—each one says, “Come back to me when you’re done.”

\begin{verbatim}
binary_search(0, 100, 32)
  ├── binary_search(0, 49, 32)
  ├── binary_search(25, 49, 32)
  ├── binary_search(32, 36, 32)
  └── Found!
\end{verbatim}

Each time the recursive call returns, Python pops one frame off the stack.  
When the base case is reached, the stack empties gracefully.  
If you forget your base case—well, Python keeps stacking until it crashes! 😅

\section{Try It Yourself}
Write your own recursive search for these problems:
\begin{enumerate}
    \item Search for a letter in a string, returning its index.
    \item Search for the smallest number in a sorted list (without using \texttt{min()}).
    \item Modify \texttt{find()} to print the number of recursive calls made.
\end{enumerate}

\section*{Challenge: Recursive Word Finder}
For extra fun, try writing a recursive function that searches through nested lists:
\begin{lstlisting}[language=Python, caption={Recursive word finder challenge}]
def find_word(nested_list, word):
    for item in nested_list:
        if isinstance(item, list):
            if find_word(item, word):
                return True
        elif item == word:
            return True
    return False

data = [["dog", ["cat", "fish"]], ["hamster", ["parrot", "snake"]]]
print(find_word(data, "snake"))  # True
print(find_word(data, "whale"))  # False
\end{lstlisting}

\noindent
This shows recursion applied to hierarchical data—a natural fit when structures contain smaller versions of themselves.

\section*{Closing Thought}
Recursive search is not just faster—it’s smarter.  
It doesn’t look at everything; it looks *strategically*.  
That’s what separates **repetition** from **recursion**—and **recursion** from **algorithmic thinking**.

\begin{center}
\textit{Next: Debugging recursive calls — the art of seeing the invisible stack.}
\end{center}

