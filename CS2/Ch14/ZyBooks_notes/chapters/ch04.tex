\chapter{Adding Output Statements for Debugging}

\section*{Why Debugging Recursion Feels Weird}
When loops misbehave, we can print a single counter and know exactly where we are.  
When recursion misbehaves, it’s like yelling down a canyon and getting ten voices back at once.  

Recursion errors can hide in the call stack, so debugging means learning to *see the depth* of the function calls.

\begin{quote}
Each recursive call is like a new mirror added to a hall of reflections.  
If you can see the pattern of echoes, you can find the bug.
\end{quote}

\section{Adding Output for Insight}
One of the best debugging tools is a simple \texttt{print()} statement.  
By printing the current range or step, and indenting based on recursion depth, we can visualize what’s happening.

\begin{lstlisting}[language=Python, caption={Recursive search with indentation for debugging}]
def find(lst, item, low, high, indent=""):
    """Finds index of item in a sorted list, else returns -1."""
    print(f"{indent}Searching {low}..{high}")

    # Base case: Not found
    if low > high:
        print(f"{indent}Not found.")
        return -1

    mid = (low + high) // 2
    print(f"{indent}Checking index {mid}: {lst[mid]}")

    # Base case: Found
    if lst[mid] == item:
        print(f"{indent}Found {item} at index {mid}")
        return mid

    # Recursive case: Search smaller half
    elif item < lst[mid]:
        return find(lst, item, low, mid - 1, indent + "   ")

    # Recursive case: Search larger half
    else:
        return find(lst, item, mid + 1, high, indent + "   ")

# Example usage
names = ["Adams, Mary", "Carver, Michael", "Domer, Hugo",
         "Fredericks, Carlo", "Liu, Jie"]

print("Running debug search for 'Carver, Michael'...\n")
find(names, "Carver, Michael", 0, len(names) - 1)
\end{lstlisting}

\noindent
Notice how the indentation grows deeper with each recursive call.  
The result is a visual "map" of the recursion depth — a living breadcrumb trail.

\begin{quote}
Indentation isn’t just about code style — it can become a debugging superpower.
\end{quote}

\section{What the Output Teaches Us}
When you run this, you’ll see the function’s “journey”:

\begin{verbatim}
Searching 0..4
Checking index 2: Domer, Hugo
   Searching 0..1
   Checking index 0: Adams, Mary
      Searching 1..1
      Checking index 1: Carver, Michael
      Found Carver, Michael at index 1
\end{verbatim}

\noindent
We can instantly see:
\begin{itemize}
    \item The recursive narrowing of the search range.
    \item Which values are checked each time.
    \item How indentation reflects call depth.
\end{itemize}

\section{Common Pitfalls to Watch For}
Recursion is a mirror that shows both clarity and chaos.  
Here are the usual suspects that break your reflection:

\begin{enumerate}
    \item \textbf{No Base Case:} The recursion never ends, causing a stack overflow.
    \item \textbf{No Return on Recursive Call:} The answer is found, but lost on the way back up.
    \item \textbf{Misplaced Print:} Messages print at the wrong depth, confusing the trace.
\end{enumerate}

\begin{quote}
A well-placed print can save an hour of staring at the screen.
\end{quote}

\section{Try It Yourself}
\begin{enumerate}
    \item Modify \texttt{find()} to count how many recursive calls were made.
    \item Write a recursive function that prints a pyramid of indentation up to depth 5.
    \item Create a recursive version of \texttt{sum()} that prints its partial results on the way up and down.
    \item Comment out each print one at a time to see which ones are most useful for debugging.
\end{enumerate}

\section{Bonus Challenge: The “Echo Locator”}
Write a recursive function called \texttt{echo()} that takes a message and depth:
\begin{lstlisting}[language=Python]
def echo(message, depth):
    if depth == 0:
        print(message)
    else:
        print(" " * depth + f"Echoing ({depth})...")
        echo(message, depth - 1)
        print(" " * depth + "Returning.")
\end{lstlisting}

\noindent
Run \texttt{echo("Recursion rocks!", 4)} and watch the indentation reveal how recursion dives and returns.

\section*{Closing Thought}
Debugging recursion is about listening to the rhythm of the calls.  
Add prints, indent the echoes, and soon you’ll be able to \emph{see} your program think.

\begin{center}
\textit{Next: Creating Recursive Functions — from reflection to design.}
\end{center}

