\chapter{12.1 Reading Files}

\section*{Overview}
In this section, students learn how to read from files using Python's built-in \texttt{open()} function.  
Files allow programs to save data permanently and retrieve it later---a sort of long-term memory for your code.  
Instead of typing data every time the program runs, we can read it directly from disk like civilized programmers.

\begin{quote}
\textbf{Learning Goals:}
\begin{itemize}
  \item Understand how to open, read, and close text files safely.
  \item Differentiate between \texttt{read()}, \texttt{readline()}, and \texttt{readlines()}.
  \item Process and analyze file data (for example, computing averages).
  \item Use best practices that prevent data loss and mysterious crashes.
\end{itemize}
\end{quote}

---

\section{Reading from a File}

The simplest way to read data from a file is to use \texttt{open()} together with \texttt{read()}.
This reads the entire file at once into a single string.

\begin{lstlisting}[language=Python, caption={Reading an entire file safely.}]
# Example 1: Reading the entire contents of a file

from pathlib import Path

file_path = Path("journal.txt")

if file_path.exists():
    with open(file_path, "r", encoding="utf-8") as journal:
        contents = journal.read()
        print("[Info] Successfully opened journal.txt!\n")
        print(contents)
else:
    print("[Warning] File 'journal.txt' not found.")
\end{lstlisting}

\noindent
\textbf{Key points:}
\begin{itemize}
  \item Always use \texttt{with open(...)} so Python closes the file automatically.
  \item Always specify \texttt{encoding="utf-8"} to avoid surprises on Windows.
  \item Check that the file exists before trying to read it---it's less embarrassing that way.
\end{itemize}

\begin{quote}
\textbf{Sample File:} \texttt{journal.txt}
\begin{verbatim}
Dear Journal,

Today I wrestled with a wild bug named SyntaxError.
After three print statements and one sigh, I prevailed.

Moral: Always close your parentheses before closing your laptop.
\end{verbatim}
\end{quote}

---

\section{A More Complete Example}

Let’s take it up a notch: we’ll add clear print messages, read the contents,  
and even perform a quick word analysis---because data deserves compliments too.

\begin{lstlisting}[language=Python, caption={Reading and analyzing file contents.}]
# Example 2: Read a file and analyze it

from pathlib import Path

path = Path("myfile.txt")

if path.exists():
    with open(path, "r", encoding="utf-8") as f:
        print("[Info] Opening myfile.txt...")
        contents = f.read()
        print("[Success] File read successfully!\n")

    print("--- File Contents ---")
    print(contents)
    print("----------------------\n")

    words = contents.split()
    print(f"[Stats] Word count: {len(words)}")
    print(f"[Stats] Longest word: {max(words, key=len)}")
else:
    print("[Warning] File 'myfile.txt' not found.")
\end{lstlisting}

\begin{quote}
\textbf{Sample File:} \texttt{myfile.txt}
\begin{verbatim}
Python is elegant.
It reads like poetry.
Sometimes, the bugs are haikus.
\end{verbatim}
\end{quote}

---

\section{Reading Line by Line}

The \texttt{readlines()} method reads a file into a list, where each element is one line of text.
This makes it easy to loop through lines individually.

\begin{lstlisting}[language=Python, caption={Reading all lines into a list.}]
# Example 3: Read lines from a file

from pathlib import Path

file_path = Path("readme.txt")

if file_path.exists():
    with open(file_path, "r", encoding="utf-8") as f:
        lines = f.readlines()

    print(f"[Info] Found {len(lines)} lines in readme.txt.\n")

    for i, line in enumerate(lines, start=1):
        print(f"Line {i:>2}: {line.strip()}")

    all_words = " ".join(lines).split()
    print(f"\n[Result] Longest word: {max(all_words, key=len)}")
else:
    print("[Warning] File 'readme.txt' not found.")
\end{lstlisting}

\begin{quote}
\textbf{Sample File:} \texttt{readme.txt}
\begin{verbatim}
Welcome to the File Reading Zone.
Line 1: Preparation.
Line 2: Curiosity.
Line 3: Revelation.
Line 4: Triumph.
\end{verbatim}
\end{quote}

---

\section{Processing Data from a File}

Files often hold numbers, and it’s common to process them to compute things like sums or averages.  
Here’s a calm, methodical way to do that without blowing up your CPU.

\begin{lstlisting}[language=Python, caption={Computing an average from file data.}]
# Example 4: Calculate the average of numbers stored in a file

from pathlib import Path

data_file = Path("mydata.txt")

if data_file.exists():
    with open(data_file, "r", encoding="utf-8") as f:
        numbers = [int(line.strip()) for line in f if line.strip().isdigit()]

    average = sum(numbers) / len(numbers)
    print(f"[Result] Average value: {average:.2f}")
else:
    print("[Warning] File 'mydata.txt' not found.")
\end{lstlisting}

\begin{quote}
\textbf{Sample File:} \texttt{mydata.txt}
\begin{verbatim}
10
15
25
20
30
\end{verbatim}
\end{quote}

---

\section{Iterating Directly Over a File Object}

For very large files, it’s better to read one line at a time instead of loading the whole file.
This approach uses minimal memory and maximum patience.

\begin{lstlisting}[language=Python, caption={Memory-efficient iteration through a file.}]
# Example 5: The efficient way to read a file

with open("myfile.txt", "r", encoding="utf-8") as f:
    for line_number, line in enumerate(f, start=1):
        print(f"Line {line_number}: {line.strip()}")
\end{lstlisting}

This method works beautifully on files of any size,  
even the ones so large you can hear your hard drive whisper, “Are you sure about this?”

---

\section{Practice Exercise}

\textbf{Challenge:}  
Write a program that asks the user for a filename, reads its contents,  
and prints them in uppercase. Bonus points if you make it sound enthusiastic.

\begin{lstlisting}[language=Python, caption={Challenge Activity: Transform file contents.}]
# Example 6: Read and transform a file

filename = input("Enter filename to shoutify: ")

try:
    with open(filename, "r", encoding="utf-8") as f:
        contents = f.read()
    print("\n[Result] SHOUTING MODE ACTIVATED:\n")
    print(contents.upper())
except FileNotFoundError:
    print("[Warning] File not found. Please try again.")
\end{lstlisting}

---

\section{Explore More}

For more detailed tutorials and examples:
\begin{itemize}
  \item \href{https://docs.python.org/3/tutorial/inputoutput.html#reading-and-writing-files}{Python Docs: Reading and Writing Files}
  \item \href{https://realpython.com/read-write-files-python/}{Real Python: Working with Files}
  \item \href{https://www.w3schools.com/python/python_file_handling.asp}{W3Schools: File Handling in Python}
\end{itemize}

---

\section*{Summary}
\begin{itemize}
  \item Use \texttt{with open()} for clean and automatic file handling.
  \item \texttt{read()}, \texttt{readline()}, and \texttt{readlines()} each serve different use cases.
  \item Always specify \texttt{encoding="utf-8"}.
  \item Check file existence before opening it.
  \item Never fear the file system—treat it like a friend that occasionally misplaces your stuff.
\end{itemize}
